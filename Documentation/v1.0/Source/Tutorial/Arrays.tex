\section{Computations on Arrays%
\label{sec:array}}
\cutname{array.html}

This section explains how to write common parallel patterns for array
processing, using DPJ's features for supporting arrays:
%
\begin{itemize}
\item \S~\ref{sec:array:disjoint}, Disjoint Array Update, shows how to
  create an array whose cells all point to different objects, then
  traverse the array in parallel and update the objects.
%
\item \S~\ref{sec:array:blocked}, Blocked Array Update, shows how to
  divide an array into segments and process the segments in parallel
  using a loop. 
%
\item \S~\ref{sec:array:dandc}, Divide-and-Conquer Array Update, shows
  how to partition and recurse on arrays in parallel.
\end{itemize}

\subsection{Disjoint Array Update%
\label{sec:array:disjoint}}

Here we explain how to use DPJ to create an array of objects, put a
different object into every element of the array, and then iterate
over the array in parallel and update the objects.  We call this
pattern Disjoint Array Update.  Because objects are stored as
references in Java, in general, multiple array cells could point to
the same object, causing a race.  We'll show how to use the DPJ type
and effect system to ensure that no such race can occur.

%\subsubsection{Writing the Pattern%
%\label{sec:array:disjoint:pattern}}

\bfhead{How to write the pattern:} Figure ~\ref{fig:array:disjoint}
illustrates the Disjoint Array Update pattern in DPJ, for a simple
computation (creating an array of objects, then incrementing a field
of each one).

\begin{figure}
\input{Listings/DisjointArrayUpdate}
\caption{Disjoint Array Update.}
\label{fig:array:disjoint}
\end{figure}


Class \kwd{Data} (lines 2--4) has an integer field \kwd{x} and one
region parameter \kwd{R} (Reference Manual, \S~2.4.1).  The field
\kwd{x} is in region \kwd{R}; that means that when \kwd{Data} is used
as a type, the field of the object instance will be in whatever region
is supplied as an argument to the type.  See Reference Manual,
\S~4.1.1.

The rest of the code consists of an array \kwd{arr} of \kwd{Data}
objects and three methods that operate on the array:
%
\begin{enumerate}
%
\item \kwd{initialize} assigns a new array to \kwd{arr} and uses a
  \kwd{foreach} loop (Reference Manual, \S~6.2) to assign a fresh
  \kwd{Data} object to every element of \kwd{arr} in parallel.
%
\item \kwd{compute} increments the \kwd{x} field of each \kwd{Data}
  object in \kwd{arr} in parallel.
%
\item \kwd{main} creates a new \kwd{DisjointArrayUpdate} object, calls
  \kwd{initialize} on it, then calls \kwd{compute} on it, prints out
  the array, and exits.
%
\end{enumerate}
%
The array \kwd{arr} has an index-parameterized type (Reference Manual,
\S~4.2.2) stating that cell \kwd{i} of the array is in region
\kwd{[i]} and has type \kwd{Data<[i]>}.  The compiler can prove
that the parallelism is safe because:
%
\begin{itemize}
%
\item In line 10, the update to \kwd{arr[i].x} has effect \kwd{writes
  [i]}.  The value of \kwd{i} is different in each loop iteration, so
  the updates are all to disjoint regions, i.e., there are no
  conflicting writes.
%
\item In line 14, each iteration of the loop reads cell \kwd{arr[i]}
  (in region \kwd{[i]}) and writes field \kwd{arr[i].x}.  Again, the
  operations are to disjoint regions in different iterations, so there
  is no interference.
%
\end{itemize}

\bfhead{Further examples:} This pattern appears in the parallel
Barnes-Hut force computation in method \kwd{computegrav} of file
\kwd{Tree.java} in directory
\kwd{Benchmarks/Applications/BarnesHut/dpj} of the DPJ release.  The
force computation iterates in parallel over a disjoint array of
\kwd{Body} objects and writes into the \kwd{force} field of each one.


\begin{comment}
\subsubsection{Algorithm Example:  Barnes-Hut Force Computation%
\label{sec:array:disjoint:bh}}

TODO

\begin{figure}
\begin{numbereddpjlisting}
TODO
\end{numbereddpjlisting}
\caption{Disjoint Array Update in the Barnes-Hut force computation.}
\label{fig:array:bh}
\end{figure}
\end{comment}

\subsection{Blocked Array Update%
\label{sec:array:blocked}}

When you write a \kwd{foreach} loop in DPJ, the runtime repeatedly
divides the iteration space in half, until a programmer-specified
depth is reached.  See Reference Manual, \S~6.2.2.  In conjunction
with an index-parameterized array that has each cell in its own region
(Reference Manual, \S~4.2.2), this is usually the easiest and best way
to write a loop that traverses an array in parallel and updates its
elements.  However, sometimes you may want to control the partition
directly, by dividing the array into blocks (or tiles) and having each
task operate on a block.  We call this pattern Blocked Array Update.

%\subsubsection{Writing the Pattern%
%\label{sec:array:blocked:writing}}

\bfhead{How to write the pattern:} Figure ~\ref{fig:array:blocked}
illustrates the Blocked Array Update pattern in DPJ, for a simple
computation (initializing an array of integers so that cell \kwd{i}
contains the value \kwd{i}).

\begin{figure}
\input{Listings/BlockedArrayUpdate}
\caption{Blocked Array Update.}
\label{fig:array:blocked}
\end{figure}

The code uses the \kwd{DPJArrayInt} and \kwd{DPJPartitionInt} classes
discussed in ReferenceManual, \S~7, and the DPJ runtime documentation.
(\kwd{DPJArrayInt} and \kwd{DPJPartitionInt} are specializations of
the \kwd{DPJArray} and \kwd{DPJPartition} classes to the primitive
type \kwd{int}.)  Line 5 creates a fresh \kwd{DPJArrayInt} called
\kwd{array} with 100 elements.  A \kwd{DPJArrayInt} wraps an ordinary
Java array (of type \kwd{int[]}) and behaves much like a
\kwd{java.util.ArrayList}: for example, it has \kwd{put} and \kwd{get}
methods for accessing the array elements.  Lines 6--7 create a
\emph{strided partition} of \kwd{array} called \kwd{segs}.  The
partition has stride 10, which means that the array is divided into 10
segments of length 10 each.  The variable \kwd{segs} is declared
\kwd{final} so that it can be used as a region (Reference Manual,
\S~3.1.5).

Lines 8--13 represent the parallel computation.  The outer parallel
\kwd{foreach} loop iterates over the segments in \kwd{segs}.  Line 4
pulls segment \kwd{i} out of the partition.  Segment \kwd{i} is itself
a \kwd{DPJArray}, representing the segment consisting of indices
$10*i$ through $10*i+9$ of the original array.  Its type is
\kwd{DPJArray<segs:[i]:*>}.  As explained in Reference Manual, \S~7.2,
the \kwd{[i]} in the type allows different segments of the same array
to be treated as disjoint, and the variable \kwd{segs} in the type
ensures that different partitions of the same array are not treated as
disjoint.  For now, all you really need to know to write the pattern
is that you have to declare the partition variable \kwd{segs}
\kwd{final}, as shown in line 6, and you have to write the type of a
segment as shown in line 9.

The inner sequential loop iterates over the elements of a segment and
writes values into its elements.  Notice that in iteration \kwd{i} of
the outer loop, the inner loop is zero-indexed, even though it is
accessing elements $10*i, 10*i+1, \ldots$ of the underlying array.
That's because when you create a \kwd{DPJPartition} out of a
\kwd{DPJArray}, each segment provides a zero-indexed \emph{view} of
some segment of the original \kwd{DPJArray}.  The \kwd{DPJArray} class
takes care of the index translation for you, which is handy.

\bfhead{Further examples:} This pattern appears in the International
Data Encryption Algorithm (IDEA), in method \kwd{Do} of file
\kwd{IDEATest.java} in directory
\kwd{Benchmarks/Applications/IDEA/dpj} of the DPJ release.  The
pattern is useful for IDEA, because the IDEA algorithm operates on
fixed-size blocks of the input data.

This pattern is also useful for implementing reductions.  See
\S~\ref{sec:reductions} of this tutorial for more details.

\begin{comment}
\subsubsection{Algorithm Example:  International Data Encryption Algorithm (IDEA)%
\label{sec:array:blocked:idea}}

Now we show how the Blocked Array Update pattern can be used to
parallelize the International Data Encryption Algorithm (IDEA).  IDEA
is a block cipher code that was designed as a replacement for the Data
Encryption Standard (DES).  It operates on 64 bit blocks using a 128
bit key using a series of eight transformations.  We use Blocked Array
Update because the algorithm requires that the blocking structure be
explicit in the code, and not hidden in the runtime.

TODO: EXPLAIN FIGURE

\begin{figure}
\begin{numbereddpjlisting}
// slice has the number of blocks.
  DPJPartitionByte<R1> inSegs  = DPJPartitionByte.stridedPartition(plain1, slice);
  DPJPartitionByte<R2> outSegs = DPJPartitionByte.stridedPartition(crypt1, slice);
  foreach (int i in 0, inSegs.length)
  {
    DPJArrayByte<inSegs:[i]:*>inSeg = inSegs.get(i);
    DPJArrayByte<outSegs:[i]:*>outSeg = outSegs.get(i);
    run1(0, inSeg.length, inSeg, outSeg, Z); // do the encryption
  }

\end{numbereddpjlisting}
\caption{Blocked Array Update in IDEA.}
\label{fig:array:idea}
\end{figure}
\end{comment}

\subsection{Divide-and-Conquer Array Update%
\label{sec:array:dandc}}

A common pattern in parallel array algorithms is to divide an array in
pieces, and work independently on the pieces, then recursively divide
again, etc., until a base case is reached.  This pattern is often
called \emph{divide and conquer}, because it divides the problem into
smaller subproblems, then ``conquers'' the subproblems by recursively
dividing them.

%\subsubsection{Writing the Pattern%
%\label{sec:array:dandc:writing}}

\bfhead{How to write the pattern:} Figure~\ref{fig:array:dandc}
illustrates the Divide-And-Conquer Array Update pattern in DPJ, for a
simple computation (initializing an array of integers so that every
cell contains the same value).  

\begin{figure}
\input{Listings/DivideAndConquerArrayUpdate}
\caption{Divide-And-Conquer Array Update.}
\label{fig:array:dandc}
\end{figure}

The code uses the \kwd{DPJArrayInt} and \kwd{DPJPartitionInt} classes
also used for Blocked Array Update (\S~\ref{sec:array:blocked}).  The
method \kwd{recursiveInit} (line 4) takes as input a \kwd{DPJArrayInt}
parameterized by a method region parameter \kwd{R} (Reference Manual,
\S~2.4.2).  Its effect is \kwd{writes R:*}, i.e., it reads and writes
regions under the parameter.  If the array is sufficiently small, the
initialization is done sequentially (lines 8--12).  Otherwise the
program creates a \kwd{DPJPartition} that splits the array in half.
Lines 17--22 recursively and in parallel call \kwd{recursiveInit} on
the halves.  The effects of the recursive calls are as shown, for the
reasons explained in Reference Manual, \S~7.2, and they are disjoint
for the two parallel tasks (see Reference Manual, \S~3.6.4).

\bfhead{Further examples:} For a more realistic example of this
pattern, see the implementation of parallel merge sort in file
\kwd{MergeSort4.java} in directory \kwd{Benchmarks/Kernels/dpj} of the
DPJ release.


