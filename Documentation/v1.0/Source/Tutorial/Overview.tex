\section{Overview of DPJ%
\label{sec:overview}}
\cutname{overview.html}

This section provides an overview of DPJ concepts, to get you started
quickly using the language.  References to \refmanual\ are provided
throughout.

\subsection{Basic DPJ Concepts%
\label{sec:overview:basic}}

DPJ is based on Java.  If you know Java, then you know most of DPJ
already.  In fact, every legal Java program is a legal DPJ program!
However, the reverse is not true, because DPJ adds several features to
Java to guarantee determinism for parallel programs.  We'll start with
the DPJ equivalent of ``Hello, world'' to explain the most basic DPJ
features.

\subsubsection{A Simple DPJ \kwd{Point} Class%
\label{sec:overview:basic:simple}}

At the most fundamental level, DPJ adds just three concepts to Java:
explicit fork-join parallelism, regions, and effects.
Figure~\ref{fig:overview} shows a simple DPJ class \kwd{Point} that
illustrates the concepts.

\begin{figure}
\input{Listings/Point}
\caption{A simple DPJ class to illustrate basic concepts.}
\label{fig:overview}
\end{figure}

\bfhead{Explicit fork-join parallelism:} The DPJ language syntax shows
explicitly where a parallel code section begins (a fork point) and
ends (the join point).  At a join point, execution is suspended until
all the parallel tasks created at the corresponding fork point have
finished executing.  In Figure~\ref{fig:overview}, the \kwd{cobegin}
statement (lines 8--11) executes its component statements (lines 9 and
10) in parallel.  The fork point is at the start of the \kwd{cobegin}
statement, and the join point is at the end of the statement.
Execution does not continue after the \kwd{cobegin} statement until
lines 9 and 10 have both completed execution.

DPJ's fork-join model makes the structure of the parallelism explicit,
in contrast to an unstructured model such as Java threads.  Together
with the determinism guarantee, this structure allows the programmer
to reason about the behavior of the parallel program as if it were the
equivalent sequential program with the parallel constructs elided.
For example, DPJ guarantees that the code shown in
Figure~\ref{fig:overview} will compute the same results as the code
in which the \kwd{cobegin} statement in lines 8--11 were replaced with
an ordinary sequential statement block.  Thus we say that DPJ has a
sequential \emph{correctness model} and a parallel \emph{performance
  model}.

\bfhead{Regions:} To facilitate the analysis of reads and writes to
the heap, DPJ groups memory locations into named sets called
\emph{regions}.  The programmer explicitly partitions the heap into
regions by marking class fields and array cells with region
information.  In Figure~\ref{fig:overview}, line 2 declares region
names \kwd{X} and \kwd{Y}.  Lines 3 and 4 say that field \kwd{x} is
located in region \kwd{X} and field \kwd{y} is located in region
\kwd{Y}.  Two or more variables may be grouped in the same region.
For example, it would be permissible to add a field \kwd{double z in
  Y} to class \kwd{Point}.  However, every variable always resides in
exactly one region.

\bfhead{Effects:} The DPJ compiler infers the read and write effects
of each statement in the program in terms of regions.  For example, in
line 5 of Figure~\ref{fig:overview}, the effect of the statement
\kwd{this.x = x} is \kwd{writes X}, because the statement assigns a
value to variable \kwd{x}, which is declared to be in region \kwd{X}
(line 3).

To make the compiler analysis simple and modular, and to document the
effects at API boundaries, the DPJ programmer writes \emph{effect
  summaries} on methods that specify their read and write effects in
terms of regions.  Examples of effect summaries are shown in lines 5,
6, and 7 of Figure~\ref{fig:overview}.  The compiler checks that every
method's summary includes all the actual effects of the method body
(including the effects of methods called in the body, using those
methods' summaries to compute the effects of the invocations).  For
example, in line 7, it would be a compile-time error to omit the
effect \kwd{writes Y} from the summary of \kwd{setXY}, because of the
write to region \kwd{Y} in line 10.  

The compiler uses the inferred statement effects and method effect
summaries to compute the effects of parallel tasks, and check that all
pairs of parallel tasks are mutually noninterfering.  For example, the
effect of line 9 is \kwd{writes X}, and the effect of line 10 is
\kwd{writes Y}.  Because \kwd{X} and \kwd{Y} name different regions,
the two writes are to disjoint memory locations, so the parallel tasks
in the \kwd{cobegin} statement are safe to run in parallel.  

Some effects don't need to be reported in a method summary.  These
are:
%
\begin{itemize}
%
\item Any read or write of a local variable (i.e., a variable declared
  inside method scope, including method formal parameters).  For
  instance, the read of parameter \kwd{x} in line 9 doesn't generate
  any effect that needs to be reported.  That's because Java doesn't
  allow you to take the reference of a local variable, so the compiler
  can confirm noninterference for local variables without any help
  from the effect system.
%
\item Any effect on a \kwd{final} variable.  That's because the value
  of a \kwd{final} variable never changes, so reading it cannot cause
  interference.
%
\item Any initialization effect on fields of \kwd{this} by a
  constructor.  That's because in DPJ, other parallel tasks can never
  see the constructed object until all the initialization is done.
%
\end{itemize}

\subsubsection{Compiling the \kwd{Point} Class%
\label{sec:overview:basic:compiling}}

Try entering and compiling the \kwd{Point} class shown in
Figure~\ref{fig:overview}.  If you've entered the code correctly, and
your DPJ environment is set up correctly, it should compile with no
errors or warnings.

Now try modifying the example a bit to see how DPJ responds to the
changes.  Playing around with programs like this is one of the best
ways to learn new language features.

First, delete \kwd{in X} at the end of line 3, and recompile.  You
should get an error that the effect summary \kwd{writes X} in line 5
doesn't cover the effects of the method \kwd{setX}.  That's because
when you take out the explicit declaration \kwd{in X}, the variable
\kwd{x} gets put in a default region called \kwd{Root}.  Now replace
\kwd{writes X} in line 5 with \kwd{writes Root}.  The code should
compile again without errors.

Now delete the effect summary in line 5 entirely.  This time the
code should compile, but it gives a warning.  Since you didn't give
any explicit effect summary for \kwd{setX}, the compiler assumes the
most conservative effect for that method --- in effect, ``writes the
entire heap.''  The code compiles, because the assumed effect
covers the effects of \kwd{setX}.  However, the assumed effect
conflicts with the effect \kwd{writes Y} of \kwd{setY}.  So the
compiler is warning you that there is potential interference between
lines 9 and 10 in the \kwd{cobegin}.

Please notice three things about this example.  First, there really
isn't any interference; the compiler is just warning you because you
haven't satisfied all its rules for \emph{proving} there is no
interference.  This may seem like a minor point for this trivial
class, but for a large and complex program, proving determinism can be
quite tricky.  So following the rules is important, because it gives a
\emph{guarantee} of correctness that would otherwise be hard to show.

Second, notice that the DPJ compiler gives an error when your declared
method effects are too small, but a warning when your parallel effects
are interfering.  This behavior is so that you can write a DPJ program
incrementally, without getting all the annotations right on the first
go.  For example, you could have written the class in
Figure~\ref{fig:overview} by putting in the \kwd{cobegin} first,
leaving out the region and effect annotations entirely.  That code
would compile, but it would give a warning.  Then you could add the
region and effect annotations incrementally, until the warnings go
away.  This is usually a good strategy for writing programs in DPJ.
Writing bad method effects causes an error (not just a warning)
because it is both easier to deal with (if you write nothing you get
the default effect, which always works) and less localized --- a bad
method effect in one place could cause interference to be
underreported in a different one.

Third, the defaults are designed so that you don't have to worry about
region and effect annotations for code that is never called inside a
parallel task.  You only need to annotate code that is going to run in
parallel.

\subsubsection{Where To Learn More%
\label{sec:overview:basic:where-tolearn}}

The Reference Manual has more detailed information on the topics
discussed above.  In addition to \kwd{cobegin} for parallel
statements, DPJ has a \kwd{foreach} construct for parallel loop
iterations.  See Reference Manual, \S~6, for more details.  Reference
Manual, \S\S~2.1 and 2.2, have more information on declaring and using
regions.  For further information about summarizing and checking
effects, see the following sections of the Reference Manual:
%
\begin{itemize}
%
\item 
\S\S~2.3.1, 2.3.2, and 5.2 explain how to write method effect summaries.
%
\item
\S~5.4 explains how the compiler computes a method's actual effects.
%
\item
\S~5.6 explains how the compiler compares a method's actual effects to
its summarized effects.
%
\item
\S~5.7 explains noninterference of effect.
\end{itemize}
%
It may be best to read at least the rest of this section of this
tutorial before consulting these sections of the reference manual,
because the full explanation of these features requires concepts (such
as region path lists and region parameters) introduced in later
subsections here.  However, the Reference Manual contains plenty of
cross references (hyperlinked in the HTML version), so you can also
just start reading the Reference Manual, following cross references
when you don't understand something, if you prefer learning that way.

\subsection{Region Path Lists%
\label{sec:overview:rpls}}

Every region in DPJ is described by a \emph{region path list}, or RPL.
A basic region name, shown in \S~\ref{sec:overview:basic:simple}, is a
particular kind of RPL.  More generally, a \emph{sequence of names}
separated by colons is an RPL.  For example, if \kwd{A} and \kwd{B}
are declared region names, then \kwd{A}, \kwd{B}, \kwd{A:B},
\kwd{A:A:B}, etc., are RPLs.  RPLs can be \emph{partially specified},
by writing a \kwd{*} in place of some sequence of names.  For example,
\kwd{A:*:B} stands in for \kwd{A:B}, \kwd{A:A:B}, \kwd{A:B:B},
\kwd{A:A:A:B}, etc.  Nesting and partial specification are useful for
describing \emph{sets of regions} in effects.  For example, in
conjunction with region parameters (\S~\ref{sec:overview:params}),
RPLs let you write an effect that says ``writes all nodes in the left
subtree.''

Here is a simple example of RPLs in action:
%
\input{Listings/RPLExample}
%
We have declared region names \kwd{A}, \kwd{B}, and \kwd{C} (line 2),
and we use them to construct the RPLs \kwd{A:B} and \kwd{A:C}.  We put
field \kwd{x} in region \kwd{A:B} and field \kwd{y} in region
\kwd{A:C} (lines 3--4).  In line 5, we use the partially specified RPL
\kwd{A:*} to cover both \kwd{A:B} and \kwd{A:C}.  Note in particular
that \emph{\kwd{A} and \kwd{A:B} are different regions}.  So
\kwd{writes A} does \emph{not} cover a write to \kwd{A:B}; the \kwd{*}
is needed to get the inclusion.  This is because to get more precision
with RPLs, DPJ separates the concepts of \emph{nesting} (\kwd{A:B} is
nested under \kwd{A}) and \emph{inclusion} (\kwd{A:B} is included in
\kwd{A:*} but not in \kwd{A}).  It may help to think of RPLs like a
file system.  For example, given the UNIX path \kwd{foo/bar}, the file
system is a tree with \kwd{bar} nested under \kwd{foo}; but \kwd{rm
  foo} won't remove \kwd{bar}; you have to say either \kwd{rm foo/bar}
or \kwd{rm foo/*}.

RPLs are useful in several ways:
%
\begin{itemize}
%
\item Covering effects in method effect summaries, as shown in the
  example above.
%
\item Distinguishing different sets of regions from each other.  For
  example, \kwd{A:*} is distinct from \kwd{B:*}, and \kwd{*:A} is
  distinct from \kwd{*:B}.  This technique is useful for proving
  noninterference.
%
\item Describing partially specified types, to allow flexible
  assignments to variables with region arguments in their types.  See
  the next section of this tutorial for more information.
%
\end{itemize}
%
Later sections of this tutorial give more examples of using RPLs.  For
the full details, see \S\S~3 (RPLs), 4 (Types), and 5 (Effects) of the
Reference Manual.

\subsection{Class and Method Region Parameters%
\label{sec:overview:params}}

\bfhead{Class region parameters:} DPJ allows you to write classes with
region parameters.  This works similarly to Java generic type
parameters and allows you to have different instances of the same
class with the fields in different regions.

Here is a simple example to illustrate class region parameters:
%
\input{Listings/ClassRegionParams}
%
Lines 1--3 define a simple \kwd{Data} class with an integer field
\kwd{x}.  The class has one \emph{region parameter}, \kwd{R}, and
field \kwd{x} is located in that class.  That means that when an
object of the \kwd{Data} class created, its \kwd{x} field will reside
in the region supplied as an argument to the type.

The rest of the program is a simple class \kwd{DataPair} which stores
two \kwd{Data} objects in fields \kwd{first} and \kwd{second} (lines
6--7).  Note that we have made the fields \kwd{final}.  That is
because we are not going to assign into the fields themselves, as in
Figure~\ref{fig:overview}; instead we are going to read the fields and
assign into the objects they refer to.  We do this in method
\kwd{updateBoth}.  In line 15, the effect is \kwd{writes First}.
That's because the write is to field \kwd{x}, \kwd{x} is in region
\kwd{R} (line 2), and \kwd{R=First} in the type of the selector
\kwd{first}.  See Reference Manual, \S~2.4.1, for more information on
defining class region parameters, and Reference Manual, \S~4, for more
information on using parametric classes as types.  See Reference
Manual, \S~5.4 for more information on how the compiler computes the
effect of a field access through a parametric type.

\bfhead{Method region parameters:} Methods can also have region
parameters in DPJ.  Method region parameters allow the same method to
be invoked with different regions.  See Reference Manual, \S~2.4.2.

\bfhead{Disjointness constraints on parameters:} You can put
disjointness constraints on class and method parameters.  This allows
you, for example, to require that two parameters coming into a class
or method be different names.  The requirement is checked by the
compiler at the point where the region arguments are given to the
parameters.  Inside the class or method definition, the compiler can
use the disjointness constraint to prove noninterference.  For
example, if two parameters are constrained to be disjoint, then
effects on them are disjoint.  See Reference Manual, \S~2.4.3, for
more details.

\bfhead{Assignment restrictions:} In order to ensure sound reasoning
about types, some assignments aren't allowed when two types with the
same class have different region arguments.  The rules are in
Reference Manual, \S~4.1.2.  Sometimes this means you can't do an
assignment you want to do.  For example, if variable \kwd{a} has type
\kwd{C<A>} and \kwd{b} has type \kwd{C<B>}, you can't say \kwd{b=a}.
In this case, you have three options:
%
\begin{enumerate}
%
\item Use partially specified RPLs (Reference Manual, \S~3.4) and/or
  additional region parameters to express the assignment.  DPJ's
  subtyping features are fairly powerful, so often you can do this.
  For example, if you give \kwd{b} the type \kwd{C<*>}, now you can
  assign \kwd{a} to it, as well as \kwd{C<B>} objects.  However, the
  \kwd{*} causes the compiler to infer coarser information about
  effects.  So this strategy doesn't always work.
%
\item Clone an object with the new type.  For example, \kwd{b =
  a.<B>clone()}. Here \kwd{clone} is a generic method that takes as an
  arguments the region parameter arguments of the outgoing object, and
  clones \kwd{this} with the new type:
\begin{dpjlisting}
<region Out>C<Out>clone() {
    C<Out> result = new C<Out>();
    // Copy fields from this to result
    ...
    return result;
}
\end{dpjlisting}
%
This strategy adds some overhead, which is a price of DPJ's
determinism guarantee in this case.
%
\item As a last resort, use a type cast (Reference Manual,
  \S~4.1.3). For example, \kwd{b = (C<B>) a}.  This is usually not a
  good option, because it breaks DPJ's determinism guarantee.
%
\end{enumerate}

\subsection{Arrays}

Arrays are important in parallel programming, so DPJ has several
features to support them.

\subsubsection{Arrays and Regions}

In DPJ, array types can have region arguments.  This is similar to
giving a region argument to a class type
(\S~\ref{sec:overview:params}).  For example, the following array type
describes an array of \kwd{int} cells all located in region \kwd{R}:
%
\begin{dpjlisting}
int[]<R>
\end{dpjlisting}
%
This technique is useful to distinguish effects on whole arrays from
effect on other whole arrays or objects.  See Reference Manual,
\S~4.2.1, for more information about regions in array types.

Sometimes it's useful to put different array cells in different
regions, in particular if you want to update different parts of the
array in parallel.  To do this, you use a DPJ feature called an
\emph{index-parameterized array}.  For example, the following array
type declares an array of \kwd{int} such that cell \kwd{i} is in
region \kwd{[i]}:
%
\begin{dpjlisting}
int[]<[_]>
\end{dpjlisting}
%
An RPL \kwd{[$e$]}, where $e$ is an integer expression, is called an
\emph{array index RPL}; it refers to the array index given by the
runtime value of $e$.  The variable \kwd{\_} is always in scope in an
array type, and refers to the index of a cell.

The two strategies can also be mixed.  For example, you can give array
\kwd{A} type \kwd{int[]<rA:[\_]>} and array \kwd{B} type
\kwd{int[]<rB:[\_]>}.  Then if you want to write in parallel to the
two arrays, you can express the effects on \kwd{A} as \kwd{writes
  rA:[?]}  and the effects on \kwd{B} as \kwd{writes rB:[?]}.  The
\kwd{[?]} is an RPL element that stands in for any array index element
(see Reference Manual, \S~3.4.2).  However, you can also express
disjoint effects on different parts of \kwd{A} or \kwd{B}.  For
example, writing to cells 0 and 1 of \kwd{A} generates effects
\kwd{writes rA:[0]} and \kwd{writes rA:[1]}, which are disjoint.

Index-parameterized arrays are also useful with arrays of class
objects and arrays of arrays.  See Reference Manual, \S~4.2.2, for
more information.  Later sections of this tutorial also provide
concrete examples.

\subsubsection{DPJ Runtime Array Classes}

The DPJ runtime provides classes \kwd{DPJArray} and \kwd{DPJPartition}
for manipulating arrays.  These classes are useful for divide and
conquer traversals that update disjoint parts of the same array.
Reference Manual, \S~7, gives an overview of how these classes work.
They are fully documented in the HTML documentation included with the
DPJ release.  Later sections of this tutorial also give concrete
examples for how to use the classes.

Please note that, as described in \S~7 of the Reference Manual, it is
not sufficient to put the DPJ runtime classes in your class path when
compiling DPJ code that uses the runtime classes.  Instead, you must
compile the runtime together with your code that depends on the
runtime.  This is because the DPJ bytecode does not yet properly
support separate compilation of DPJ annotations, so the compiler needs
all the source code to process the annotations.
%

