\section{Introduction%
\label{sec:introduction}}
\cutname{introduction.html}

Deterministic Parallel Java (DPJ) is an extension to the Java
programming language that guarantees \emph{determinism} for programs
written with fork-join parallelism.  That is, if a DPJ program
compiles without errors or warnings, then it will produce the same
output on every execution of the program.  DPJ achieves its
determinism guarantee by enforcing \emph{noninterference}: DPJ
guarantees that for any two parallel tasks, there are no pairs of
\emph{conflicting accesses} (i.e., accesses to the same memory
location, with at least one a write) performed by the tasks.

DPJ's determinism guarantee provides a significant benefit over
traditional models such as Java threads.  In such models, conflicting
memory accesses and even data races can occur, so the behavior of the
program can vary depending on the interleaving of operations from
different threads on each execution.  Such nondeterminism makes many
parallel programs difficult to reason about and hard to debug.

The purpose of this tutorial is to bring you up to speed quickly using
DPJ.  Section~\ref{sec:overview} gives an overview of major DPJ
concepts, so you can start writing programs as quickly as possible.
It provides links to \refmanual\ (e.g., Reference Manual, \S~1) so you
can look up particular features there to get more detail.  The rest of
the sections show how to write common parallel patterns in DPJ.  We
believe that a very good way to get started with DPJ is to (1) look
for patterns in this guide that are the same as or similar to patterns
occurring in your parallel algorithm; (2) study and understand the
relevant pattern or patterns in this guide; and (3) adapt the pattern
or patterns to your needs.

Throughout this tutorial are DPJ code examples.  Some are fragments of
larger programs, but many are short, self-contained programs.  We
recommend that, before reading this tutorial, you download and
install DPJ.  See \installmanual\ for instructions on how to do it.
Then, as you read this tutorial, you can compile and run the examples.

