\section{Associative Reductions%
\label{sec:reductions}}
\cutname{reductions.html}

This section discusses how to implement associative reductions in DPJ.
A reduction is a common parallel pattern for computing a single datum
from a collection of data.  The transformation from the collection of
data to the single datum is called ``reducing'' the collection.  For
example, a sum reduction of a collection of integers sums all the
elements in the collection to produce a single integer representing
the sum.

A reduction is \emph{associative} if it can be represented as an
associative binary operation on the elements.  For example, a sum
reduction is associative because it is represented by the binary
operation $+$, which is associative; that means that for any integers
$a$, $b$, and $c$, $a+(b+c)=(a+b)+c$.  

\bfhead{How to write the pattern:} Associative reductions can be
written efficiently as parallel algorithms.  There are at least two
ways to do it, both supported in DPJ:
%
\begin{enumerate}
%
\item \emph{Recursive reduction:} Use divide-and-conquer recursion.
%
\item \emph{Iterative reduction:} Divide the input collection up into
  pieces, loop over all the pieces in parallel and reduce each piece
  to a partial result, then reduce the partial results to the final
  result, either sequentially or in parallel.
%
\end{enumerate}
%
Below we illustrate method (2) (iterative reduction).
\S~\ref{sec:local:method} shows an example of a divide-and-conquer sum
reduction.

\begin{figure}
\input{Listings/IntegerSumReduction.tex}
\caption{Iterative sum reduction in DPJ.}
\label{fig:reductions:pattern}
\end{figure}

Figure~\ref{fig:reductions:pattern} lists a simple DPJ program that
implements an iterative integer sum reduction.  Line 4 declares a
private region \kwd{AccumRgn} for holding the accumulated results.
Line 5 declares an accumulation variable \kwd{sum} in \kwd{AccumRgn}.
Lines 6--8 give the prototype for the public method \kwd{reduce},
which takes a \kwd{DPJArrayInt} and a tile size (i.e., the number of
array elements to process in each task) and does the reduction.
Method \kwd{reduce} has a region parameter \kwd{R}, so a
\kwd{DPJArrayInt} with any RPL argument can be passed to its \kwd{arr}
parameter.  The region parameter is declared so that binding
\kwd{R=$r$} at a call site must satisfy \kwd{$r$:* \# AccumRgn}
(Reference Manual, \S~2.4.3).  The method is declared to read under
\kwd{R} (because it is reading the array), and to write \kwd{AccumRgn}
(because it uses \kwd{AccumRgn} to compute the sum)

The body of the \kwd{reduce} method sets \kwd{sum} to zero and uses a
blocked array to do the partial sums.  The array blocking is similar
to Blocked Array Update (\S~\ref{sec:array:blocked}), except the array
is only read, and not written.  Inside the \kwd{foreach} loop (lines
13--19), there is a sequential partial sum computation, followed by a
global accumulation.  The accumulation invokes the private method
\kwd{updateSum} (defined at line 22).  That method is declared
\kwd{commutative} (Reference Manual, \S~2.3.4), because it is properly
synchronized to allow concurrent execution, and the operation it
performs commutes with itself.  This is a typical way to write an
accumulation operation in DPJ.  If the number of parallel tasks is
small, one could also have each task store its partial result into an
array cell, and then sum the elements of that array in a separate
sequential phase.

Lines 26 and following show a \kwd{main} method that creates a new
integer array of size 1,000,000, initializes element 42 of the array
with the value 42, and calls \kwd{reduce} with a tile size of 1000.
Since Java initializes the other array cells to 0, hopefully the
answer is 42.

\bfhead{Further examples:} The following benchmarks in the DPJ release
implement reductions:
%
\begin{itemize}
%
\item \kwd{SumReduce.java} in \kwd{Benchmarks/Kernels/dpj} is a simple
  sum reduction kernel, using the divide-and-conquer strategy.
%
\item The Monte Carlo benchmark computes results for many tasks, then
  reduces the partial results to a single answer using an iterative
  reduction.  The final accumulation step is sequential.  See method
  \kwd{processResults} in file \kwd{AppDemo.java} in directory
  \descr{\kwd{Benchmarks/Applications/MonteCarlo/dpj}.}
%
\item The k-means clustering benchmark uses an iterative reduction to
  compute a histogram.  See method \kwd{work} in file
  \kwd{Normal.java} in directory
  \kwd{Benchmarks/Applications/KMeans/dpj}.
%
\end{itemize}
