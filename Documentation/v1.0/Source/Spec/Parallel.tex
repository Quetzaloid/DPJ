\section{Parallel Control Flow}
\label{section:parallel}

DPJ extends Java's control flow constructs by adding two parallel
statements: \kwd{cobegin} and \kwd{foreach}.  \kwd{cobegin} followed
by a statement block executes each statement of the block as a
parallel task.  \kwd{foreach} defines a loop that executes sets of
iterations as parallel tasks.

\subsection{Cobegin}


The \kwd{cobegin} statement consists of the keyword \kwd{cobegin},
followed by a statement $S$ to execute.  If $S$ is any statement but a
block enclosed in curly braces $\kwd{\{} \ldots \kwd{\}}$, or if $S$
is a block consisting of a single statement, then the \kwd{cobegin}
just executes the statement $S$.  Otherwise, $S$ is a block containing
multiple statements, and the component statements of $S$ are run as
parallel tasks.  There is an implicit barrier (join) at the end of the
\kwd{cobegin} statement, so that all the component tasks must finish
execution before the parent task executes the statement after the
\kwd{cobegin}.

The compiler computes the inferred effect set
(\S~\ref{section:effects:inference}) of each statement in the block and checks
the effect sets for pairwise noninterference
(\S~\ref{section:effects:noninterference}).  The compiler emits a warning if
interference is discovered.  Both parallel and sequential code
generation of \kwd{cobegin} blocks are supported.

\subsection{Foreach}
\label{section:parallel:foreach}

The \kwd{foreach} statement has the syntax
\kwd{foreach(}\emph{index-var}\kwd{ in }\emph{start}\kwd{,
}\emph{length}\kwd{, }\emph{stride}\kwd{)} $S$, where 
%
\begin{itemize}
\item \emph{index-var} is an integer variable declaration.
%
\item \emph{start}, \emph{length}, and \emph{stride} are integer
  expressions.  The \emph{stride} expression is optional, and the
  default is 1.
%
\item $S$ is a statement representing the loop body.  It may be either
  a simple statement or a block.
%
\end{itemize}
%
The \kwd{foreach} statement causes one instance of $S$ to execute for
each value of \emph{index-var} in the iteration space
$\{\nonterm{start} + \nonterm{stride} \cdot i | i \in [0,
  \nonterm{length}-1]\}$.  \emph{index-var} is treated as a
\kwd{final} variable in $S$, so that any assignment to the index
variable inside the loop causes an error.

The compiler performs the following noninterference check for indexed
\kwd{foreach} loops:
\begin{enumerate}
\item Infer the effect set of the body of the \kwd{foreach}
  (\S~\ref{section:effects:inference}).
\item Create a copy of the effect set generated in (1), but replace
  every occurrence of \emph{index-var} with the negation expression
  (\S~\ref{section:rpls:relations:disjoint}) of \emph{index-var}.
\end{enumerate}
If interference is detected, the compiler generates a warning.

Both parallel and sequential code generation are supported.  For
sequential code generation, the \kwd{foreach} iterations are executed
in sequence, from lowest to highest in the iteration space.  For
parallel code generation, the compiler repeatedly halves the iteration
space, recursively forking off pairs of parallel tasks, until a
programmer-specified cutoff.  The precise mechanism for specifying the
cutoff is implementation dependent.

