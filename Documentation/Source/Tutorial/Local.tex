\section{Computing with Local Objects%
\label{sec:local}}

This section discusses a common parallel pattern that we call Local
Objects.  Often an object is created by some task, and not seen by any
other task running in parallel with that one.  Either it is used only
by the creating task, and its lifetime ends with that task; or it is
stored into global memory, but no reference to it is seen until after
all the parallel tasks have completed.  We call this kind of object a
\emph{local object}.  Local objects are useful, because effects on
them by the creating task are hidden from other tasks running in
parallel, so the effects cannot cause interference.

This section discusses two kinds of local objects:
%
\begin{enumerate}
%
\item \S~\ref{sec:local:iteration} discusses \emph{iteration-local
  objects}.  These objects are created inside a \kwd{foreach}
  iteration, and are local to that iteration.
%
\item \S~\ref{sec:local:method} discusses \emph{method-local objects}.
  These objects are created inside a method, and are local to that
  method scope and its callees.
%
\end{enumerate}
%


\subsection{Iteration-Local Objects%
\label{sec:local:iteration}}

There are two ways to write the Iteration-Local Objects pattern in
DPJ: using local regions, and using array regions.

\begin{figure}
\input{Listings/IterationLocalObjects}
\caption{Iteration-Local Objects.}
\label{fig:local:iteration:pattern}
\end{figure}

\bfhead{How to write the pattern with local regions:} Method
\kwd{usingLocalRegions} (line 10 of
Figure~\ref{fig:local:iteration:pattern}) illustrates the
Iteration-Local Objects pattern using local regions.  Class
\kwd{LocalObject} (lines 2--8) has an integer field called \kwd{value}
and a method \kwd{produceValue}.  The \kwd{produceValue} method
accepts an integer value, stores it to the object's \kwd{value} field,
and returns it.  The effect of \kwd{produceValue} is to write to
\kwd{R}, the region of the \kwd{LocalObject} type.  The rest of the
program (lines 9 and following) creates an array and writes to it in
parallel.  Line 9 creates an array of integers called \kwd{results}.
The array is index-parameterized (Reference Manual, \S~4.2.2), so that
every cell is in its own region.

Line 16 produces two effects, both disjoint for distinct iterations of
the \kwd{foreach} loop at line 11.  First, because \kwd{results} is
index-parameterized, the effect of the assignment to \kwd{results[i]}
is \kwd{writes [i]}; the write is to a different region for each value
of \kwd{i} in each distinct iteration.  Second, the effect of invoking
\kwd{produceValue} is \kwd{writes R} with $\kwd{R}=\kwd{LocalRegion}$,
that is, \kwd{writes LocalRegion}.  Since the scope of
\kwd{LocalRegion} is local to an iteration of the \kwd{foreach}, this
effect does not cause any interference across \kwd{foreach}
iterations.


\bfhead{How to write the pattern with array regions:} Method
\kwd{usingArrayRegions} (line 19 of
Figure~\ref{fig:local:iteration:pattern} illustrates the
Iteration-Local Objects pattern using array regions.  The code is the
same as in method \kwd{usingLocalRegions}, except that instead of an
iteration-local region, it binds the array region \kwd{[i]} to the RPL
parameter of the \kwd{LocalObject} class in iteration \kwd{i} of the
\kwd{foreach} loop.  The effect of invoking \kwd{produceValue} in line
14 is \kwd{writes [i]}, which is a distinct region for each iteration
of the loop.

\bfhead{Further examples:} The Monte Carlo benchmark in the DPJ
release uses the array region version of this pattern.  Monte Carlo
operates in parallel on a set of independent tasks, and uses a local
object called \kwd{PriceStock}, parameterized by an iteration index
region, to store temporary data for processing the task.  See the file
\kwd{AppDemo.java} in the directory
\kwd{Benchmarks/Applications/MonteCarlo/dpj}.  


\begin{comment}
\subsubsection{Algorithm Example:  Monte Carlo Path Computation%
\label{sec:local:iteration:mc}}

TODO

\begin{figure}
\begin{numbereddpjlisting}
TODO
\end{numbereddpjlisting}
\caption{Iteration-Local Objects in the Monte Carlo path computation.}
\label{fig:local:iteration:mc}
\end{figure}
\end{comment}

\subsection{Method-Local Objects%
\label{sec:local:method}}

\begin{figure}
\input{Listings/MethodLocalObjects}
\caption{Method-Local Objects.}
\label{fig:local:method:pattern}
\end{figure}

\bfhead{How to write the pattern:}
Figure~\ref{fig:local:method:pattern} illustrates the Method-Local
Objects pattern with a simple sum reduction that also performs an
incidental computation on local data.  Class \kwd{LocalObject} (lines
4--10) is the same as in \S~\ref{sec:local:iteration}.  Method
\kwd{sumReduce} (lines 11 and following) takes as input a
\kwd{DPJArrayInt} (Reference Manual, \S~7.1).  In the base cases it
either does nothing (empty array) or returns the only value (array
with one element).  Otherwise, it divides the array in half and
processes each half recursively and in parallel.  This part of the
computation is similar to the Divide-and-Conquer Array Update pattern
(\S~\ref{sec:array:dandc}), except that the array is only read, not
written.

The rest of the computation (lines 20 and following) declares a local
region (Reference Manual, \S~2.3.3) called \kwd{LocalRegion}, and
creates a \kwd{LocalObject} with the local region as its RPL argument.
The effect of invoking \kwd{localObject.produceValue} in line 24 is
\kwd{writes LocalRegion}, as shown in line 23.  This effect is a local
effect (Reference Manual, \S~5.3), so it doesn't have to be reported
in the declared effect of \kwd{sumReduce} (Reference Manual,
\S~2.3.1).  The write effect on the \kwd{LocalObject} instance in each
invocation of \kwd{sumReduce} is therefore hidden, or \emph{masked},
from the callee, and doesn't cause any interference at the
\kwd{cobegin} in line 16.


\bfhead{Further examples:} The Collision Tree benchmark in the DPJ
release uses method-local objects.  Collision Tree computes whether
two trees representing objects in space have any intersection, by
recursively traversing the trees in parallel.  The computation is
read-only on the trees, but the computation on each tree node passes
results (a list of intersecting triangles) up to its parent.  Before
recursively calling itself in parallel on two subtrees, the
\kwd{intersect} method creates fresh objects to hold the results
from one of the subtrees, placing their data in a local region.
The computation on the other subtree reuses the data structures 
used for the parent node, so the computations on the two subtrees
write their results to different regions and can be safely run in
parallel.  See the file \kwd{CollisionTree.java} in the directory
\kwd{Benchmarks/Applications/CollisionTree/src/com/jme/bounding}.

Method-local regions are also useful for algorithms that explore a
tree-shaped search space: for example, a recursive solver (such as a
tic-tac-toe or n queens solver).  In such algorithms, typically some
local state (like a board position) is copied and sent to all possible
next moves.  Each next move does its computation on its private copy
of the state, then returns its result to the parent.  The private copy
of the state is a method-local object.

