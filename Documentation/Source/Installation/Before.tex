\section{Before You Begin%
\label{sec:build}}
\cutname{Building.html}

Thank you for you interest in DPJ.  This section covers the
information you need to know before installing DPJ on your system.

\subsection{Requirements%
\label{sec:requirements}}
\cutname{requirements.html}

DPJ is based on the Java language and \kwd{javac} compiler from Sun
Microsystems. This manual assumes that the reader is familiar with
Java, including compiling and running Java programs.  You will need at
least a Java Development Environment (JDK) to build and run DPJ codes.  If you wish
to work on the compiler sources you will need a Java Development Kit
(JDK) to build the compiler from source.  We recommend the latest Sun
JDK from \kwd{java.sun.com}.

\subsection{Developer or User?}

Instructions and requirements are slightly different depending on
whether you want to use DPJ \emph{(user install)} or develop DPJ
\emph{(developer install)}.  A user install of DPJ includes the
bytecode versions of the compiler and runtime, documentation, and the
DPJ source code for the benchmarks.  A developer install of DPJ
includes everything from the user install, and adds the compiler and
runtime sources and build environment.

Performing a user install is easier, but there are two reasons to do a
developer install:
%
\begin{enumerate}
%
\item You are interested in reading or modifying the source code for
  the DPJ compiler and/or runtime; or
%
\item You want access to the latest updates to the DPJ compiler,
  without waiting for the next bytecode release.
\end{enumerate}


\subsubsection{Requirements for User Install%
\label{sec:runDPJ}}
%
For a user install of DPJ, do the following:
\begin{enumerate}
\item Make sure you have a Java Devlopment Kit (JDK) installed.  As of
  this writing we have tested with JDK 1.6.0\_20.
\item Follow the instructions in \S~\ref{sec:user-install} to do
  the installation and start using DPJ.
\end{enumerate}

\subsubsection{Requirements for Developer Install%
\label{sec:buildDPJ}}
For a developer install of DPJ, do the following:
\begin{enumerate}
\item Make sure you have a JDK (at least Java 6) installed.
\item Make sure that you have a working installation of Apache
  \kwd{ant}, the Java build tool.
\item Make sure you have a working installation of \kwd{git}, the
  version control system.
%
\begin{description}
\item  \kwd{http://git.com}.
\end{description}
\item Make sure you have a working \kwd{make}.
\item The Eclipse IDE is invaluable for studying and modifying large
  Java programs, so we recommend installing it.  See
%
\begin{description}
\item \kwd{http://www.eclipse.org}
\end{description}
\item Follow the instructions in \S~\ref{sec:dev-install} to do the
  installation.  Then read \S\S~\ref{sec:invoking}
  and~\ref{sec:running} to learn how to use DPJ.
\end{enumerate}

\subsection{What's in the Release}
The release directory structure contains the following directories:
\begin{itemize}
\item \kwd{Documentation }: Manuals and instructions for using and/or
  building DPJ.
\item \kwd{Implementation}: The DPJ compiler and runtime.
\item \kwd{Benchmarks}: Example code kernels and applications written
  for DPJ.
\end{itemize}

 \subsection{DPJ Resources% 
\label{sec:getDPJ}}
You should keep the following resources in mind as you work with DPJ:
\begin{enumerate}
\item The DPJ home page: \kwd{dpj.cs.illinois.edu}.
\item The DPJ public code repository:
  \kwd{http://github.com/dpj/DPJ.git}.
\item The DPJ development mailing list: \kwd{dpjdev@cs.uiuc.edu}.
  Joining the list allows you to follow major announcements, news, and
  technical discussions regarding DPJ.  To subscribe to the list,
  please visit
%
\begin{description}
\item \kwd{http://lists.cs.uiuc.edu/mailman/listinfo/dpjdev}.
\end{description}

\item DPJ documentation is located in the \kwd{Documentation}
  directory of the DPJ release and is also available on the DPJ web
  site at \kwd{http://dpj.cs.illinois.edu/DPJ/Download.html}.
\end{enumerate}

The DPJ development team appreciates your feedback and questions.
Please check the email list archives before submitting a question or
bug report.  We prefer using the list for submissions so that the
entire DPJ community can benefit from the growing knowledge base of
questions and answers.


\subsection{DPJ License}

The DPJ software is subject to the following licenses:
%
\begin{itemize}
%
\item The DPJ compiler is based on Sun's \kwd{javac} compiler and is
  covered by the GNU General Public License version 2.
%
\item The programs in the \kwd{Benchmarks/Applications} directory are
  based on codes written by various third parties, and are subject to
  their licenses.
%
\item The rest of the code is by the University of Illinois and is
  released under the University of Illinois/NCSA Open Source License.
%
\end{itemize}
%
See the file \kwd{LICENSE.TXT} in the top-level directory of the DPJ
software for further license information.


