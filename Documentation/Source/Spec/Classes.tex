\section{Class and Interface Definitions}
\label{section:class-def}

This section describes DPJ's extensions to Java class and interface
definitions.

\subsection{Class Region Name Declarations}
\label{section:class-def:class-rgn-decl}

A region name declaration as defined immediately below may appear as a
class member (JLS \S~8.1.6).

A region name declaration has the form \kwd{region}
\emph{region-decls}, where \emph{region-decls} is a comma-separated
list of identifiers.  The declared names are available within the
enclosing class scope and via inheritance and/or class selection (as
for an ordinary Java static member, JLS \S~8.2), subject to visibility
restrictions.  The declared names may be used as name RPL elements
(\S~\ref{section:rpls:elts:name-elts}) wherever they are available.

Every region name declaration appearing in a particular class must
have a unique name.  Otherwise, if two region declarations with the
same name appear in the same scope, the innermost declaration hides
all the others.

\subsection{Fields}
\label{section:class-def:fields}

Any field declaration may include an \emph{RPL specifier} with the
syntax \kwd{in} \emph{rpl}, where \emph{rpl} is a valid RPL
(\S~\ref{section:rpls:valid}). If the RPL specifier appears in a field
declaration, it must appear after the field name and before the
initializer expression, if any.  The RPL specifier is meaningful only
for non-\kwd{final} fields; if an RPL specifier appears in a
\kwd{final} field, it is allowed but silently ignored.  RPL specifiers
are not allowed for local variable declarations or formal method
parameter definitions, because those variables reside in stack regions
(see \S~\ref{section:effects:inference}).


\subsection{Methods}
\label{section:class-def:methods}

\subsubsection{Method Effect Summaries}
\label{section:class-def:methods:effect-summaries}

Every method may optionally declare its effects via an \emph{effect
  summary} (\S~\ref{section:effects:summaries}).  If an effect summary
is present, it must appear after the method arguments, before the
method body, and before the \kwd{throws} clause, if any.  The compiler
internally converts each method effect summary to an effect set
(\S~\ref{section:effects:basic}), called the \emph{declared effect set}
of the method.  If there is no explicit effect summary, then the
compiler inserts a \emph{default effect set} containing \kwd{writes
  Root:*}.  In particular, an ordinary Java 1.6 method (with no DPJ
annotations) is a valid DPJ method, whose effect summary is
\kwd{writes Root:*}.

The compiler infers the effects of the method body
(\S~\ref{section:effects:inference}) and checks the following, after
deleting all local effects (\S~\ref{section:effects:basic}):
%
\begin{itemize}
%
\item The inferred effects are a subeffect
  (\S~\ref{section:effects:subeffects}) of the declared effects.
%
\item The declared effects are a subeffect
  (\S~\ref{section:effects:subeffects}) of the declared effects of all
  methods overridden by this one.
%
\end{itemize}
%
If either of these requirements is not met, the compiler generates an
error.

\bfhead{Constructors:} The following rules apply to the method effect
summaries of constructors:
%
\begin{itemize}
%
\item The effect of writing into a field of the object being
  constructed does not have to be reported in the constructor's effect
  summary, even if the field is not declared \kwd{final}.  This is
  because in DPJ, the effect of initializing an object in a task is
  not seen until after all concurrent tasks have completed.
%
\item All instance initializer effects for a class must be included in
  the method summaries of all constructors of the class.
%
\end{itemize}

\subsubsection{Local Region Name Declarations}
\label{section:class-def:methods:local-rgn-decl}

A region declaration may appear as a statement in a method body (JLS
\S~ 8.4.7).  The declaration has the same form as for class region
name declarations (\S~\ref{section:class-def:class-rgn-decl}).  The
declared names are available as name RPL elements in the innermost
enclosing statement block.

\subsubsection{Commutativity Annotations}
\label{section:class-def:methods:commutativity}

The keyword \kwd{commutative} may optionally appear as a method
qualifier, before the return type and before the method parameters, if
any.  The \kwd{commutative} annotation is a programmer-specified
guarantee that the method commutes with itself, i.e., any pair of
\kwd{invokes} effects (\S~\ref{section:effects:basic}) using this
method are treated as noninterfering.  If \kwd{commutative} appears in
the definition of a method $M$, then it must also appear in the
definitions of all methods overriding $M$.

When using \kwd{commutative}, the programmer is responsible for
ensuring two things:
%
\begin{enumerate}
%
\item The method is properly synchronized so that concurrent
  invocations of the method behave as though they have occurred in
  sequence (i.e., the invocations have serializable
  semantics~\cite{Papadimitriou:Database}).

\item The semantics of the method is such that either order of a pair
  of invocations produces the same result.
%
\end{enumerate}
%
Note that ``same result'' in condition (2) generally has an
application-specific meaning.  For instance, two insert operations on
a set may commute with each other, because they produce the ``same''
set (i.e., the two sets are indistinguishable as far as future set
operations like find are concerned), even though the sets may have
different internal representations.  However, two append operations on
an ordered list generally do not commute.

\subsection{Type and RPL Parameters}
\label{section:class-def:params}


\subsubsection{Class Parameters}
\label{section:class-def:params:class}

As part of a class or interface definition, a list of generic type
and/or RPL parameter declarations may appear directly after the class
or interface name, which follows the keyword \kwd{class} or
\kwd{interface}.  A parameter declaration extends the generic
parameter declaration of Java 1.6 and has the form
\kwd{<}\emph{params}\kwd{>} or
\kwd{<}\emph{params}\kwd{|}\emph{constraints}\kwd{>}, where
\begin{itemize}
\item \emph{params} is a comma-separated nonempty list of identifiers
  stating the parameters being declared.  The generic type parameters,
  if any, must come first; followed by the RPL parameters, if any.
  Any of the type parameters may be preceded by the keyword
  \kwd{type}.  At least the first RPL parameter must be preceded by
  the keyword \kwd{region}, and the other RPL parameters may be
  preceded by the keyword \kwd{region}.
\item \emph{constraints} is a comma-separated list of \emph{RPL
  constraints}.  An RPL constraint has the form \emph{rpl-1} \kwd{\#}
  \emph{rpl-2}, where \emph{rpl-1} and \emph{rpl-2} are legal RPLs
  (\S~\ref{section:rpls:valid}) in the scope of the class definition.
\end{itemize}
The type parameters function exactly as in Java 1.6 (notice that if
only type parameters appear, and the optional \kwd{type} keyword is
not used, the syntax corresponds exactly to Java 1.6).  The RPL
parameters may be used as RPL elements (\S~\ref{section:rpls:elts}) in
the scope of the class definition, including in any type constraints
appearing in any generic parameter declarations.  As with type
parameters in Java 1.6, the type and RPL parameters are available only
in an instance context (i.e., in any scope where \kwd{this} is in
scope).

The RPL parameter constraints, if present, impose disjointness
constraints (\S~\ref{section:rpls:relations:disjoint}) that are
enforced when RPL arguments are bound to the parameters in a type that
instantiates the class (\S~\ref{section:types:class:args}).  Each
constraint \emph{rpl-1}\kwd{ \# }\emph{rpl-2} is also recorded in the
environment, so that the compiler can prove that \emph{rpl-3} and
\emph{rpl-4} are disjoint, if \emph{rpl-3} is included in
(\S~\ref{section:rpls:relations:inclusion}) \emph{rpl-1} and
\emph{rpl-4} is included in \emph{rpl-2}.  It is an error to write a
constraint \emph{rpl-1}\kwd{ \# }\emph{rpl-2} such that \emph{rpl-1}
is included in \emph{rpl-2} or vice versa.

\subsubsection{Method Parameters}
\label{section:class-def:params:method}

\noindent
\textbf{Declaring method parameters:} As part of a method definition
(including constructors), a list of type and RPL parameter
declarations may appear inside angle brackets \kwd{<...>} in the same
position where Java 1.6 allows type parameters, i.e., after the method
qualifiers (\kwd{public}, \kwd{static}, etc.), if any, and before the
method return type (for non-constructor methods) or class name (for
constructors).  A method parameter declaration has the same form as a
class parameter declaration (\S~\ref{section:class-def:params}).  The
declared RPL parameters may be used as RPL elements in the scope of
the method definition, including the formal parameters and return
type.  The constraints, if present, impose disjointness constraints
that are enforced when the parameters become bound in a method
invocation.  The constraints are also recorded in the method
environment, as described in \S~\ref{section:class-def:params} for
class parameter constraints.

\noindent
\textbf{Invoking parametric methods:} When a method defined with type
and/or RPL parameters is invoked, the invocation (at the programmer's
option, and depending upon the support for region inference in the DPJ
compiler) may use either \emph{explicit type and RPL arguments} or
\emph{inferred type and RPL arguments}.  This approach corresponds to
the way that Java 1.6 handles generic method parameters.

\noindent
\textit{Explicit Type and RPL Arguments:} As in Java 1.6, if type
and/or RPL arguments are present, then there must be an explicit
selector (a static class selector or an expression of class or
interface type) followed by a dot.  The arguments must be enclosed in
angle brackets \kwd{<...>}, after the dot and before the method name.
For constructors, the arguments must follow the keyword \kwd{new} and
precede the class name.  In either case, the RPL arguments have the
same form as for class and interface RPL arguments
(\S~\ref{section:types:class:args}), except that the keyword
\kwd{region} must precede at least the first RPL argument.  (This
requirement tells the compiler which arguments are types and which
ones are RPLs.  Because of Java's method overloading, this information
is not available from the method name, as it is for classes from the
class name.)  The number of type and RPL arguments must exactly match
the number of type and RPL parameters; otherwise, the compiler reports
an error.

\noindent
\textit{Inferred Type and RPL Arguments:} If a method is defined with
type and/or RPL parameters, then the method invocation may be written
without explicit type or RPL arguments.  The compiler \emph{may}
attempt to infer the RPL arguments from the types of the value
arguments, using an algorithm similar to the one that Java 1.6 uses
for inferring generic parameters.  This specification does not require
any particular inference algorithm; in particular no inference is a
correct implementation of the language.  If the inference is not
possible for a given RPL argument, either because inference is not
supported or it is not possible (for instance, because the parameter
does not appear in any of the actual argument types), then the
compiler must use \kwd{Root:*} as the RPL for that argument.

\noindent
\textit{RPL Constraints:} The compiler checks the RPL arguments
against the constraints, as discussed in
\S~\ref{section:types:class:args} for class parameter constraints.  If
any constraint is violated, the compiler issues a warning.

