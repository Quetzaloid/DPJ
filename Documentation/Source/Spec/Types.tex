\section{Types}
\label{section:types}

\subsection{Class and Interface Types}
\label{section:types:class}

\subsubsection{Type and RPL Arguments}
\label{section:types:class:args}

When a class or interface defined with type and/or RPL parameters
(\S~\ref{section:class-def:params}) is used to name a type, the type
may optionally specify type and/or RPL arguments.

\noindent
\textbf{Syntax:} The arguments have the form
\kwd{<}\emph{args}\kwd{>}, where \emph{args} is a comma-separated list
of valid class or interface types (this section) and/or RPLs
(\S~\ref{section:rpls}).  This syntax, if it appears, must immediately
follow the class or interface name. The types, if any, must appear
first; followed by the RPLs, if any.  Any of the types may be preceded
by the keyword \kwd{type}, and any of the RPLs may be preceded by the
keyword \kwd{region}; but the keywords are optional, as the compiler
can infer which are the types and which are the RPLs from the class or
interface definition.

\noindent
\textbf{Binding RPL arguments:} It is an error to specify more RPL
arguments than there are RPL parameters.  If fewer RPL arguments than
parameters are specified, the remaining arguments (from left to right)
become bound to \kwd{Root}.  For each RPL parameter constraint
\emph{rpl-1}\kwd{ \# }\emph{rpl-2} specified in the parameter
declaration (\S~\ref{section:class-def:params}), the compiler checks
that the RPLs \emph{rpl-1} and \emph{rpl-2} are disjoint
(\S~\ref{section:rpls:relations:disjoint}), after substituting the RPL
arguments for the corresponding parameters in the RPLs.  If any
disjointness constraint is violated, the compiler issues a warning.

\subsubsection{DPJ Erasure of Class Types}
\label{section:types:class:dpjerasure}

The \emph{DPJ erasure} of a class type is the type obtained by
ignoring all RPL parameters and arguments.  For example, suppose class
\kwd{C} has a type parameter \kwd{T} and a region parameter \kwd{R}.
Then the DPJ erasure of \kwd{C<Object,Root>} is \kwd{C<Object>}.

\subsubsection{Equivalent Types}
\label{section:types:class:equiv}

Class or interface types $T_1$ and $T_2$ with no type or RPL arguments
are equivalent if they are equivalent in ordinary Java.  Otherwise,
$T_1$ and $T_2$ are equivalent if (1) their DPJ erasures
(\S~\ref{section:types:class:dpjerasure}) are equivalent according to
this section; and (2) for every RPL parameter $P$ of $C$, the RPL
arguments bound to $P$ in $T_1$ and $T_2$ are equivalent
(\S~\ref{section:rpls:relations:equiv}).

\subsubsection{Subtypes}
\label{section:types:class:subtypes}

\bfhead{Types that instantiate the same class or interface :} $T_1$ is
a subtype of $T_2$ in DPJ if (1) the two types instantiate the same
class or interface; (2) the two types have no type or RPL arguments;
and (3) $T_1$ is a subtype of $T_2$ in ordinary Java.

$T_1$ is also a subtype of of $T_2$ in DPJ if (1) the two types
instantiate the same class or interface; (2) the DPJ erasure
(\S~\ref{section:types:class:dpjerasure}) of $T_1$ is a subtype of the
DPJ erasure of $T_2$ according to this section; and (3) the RPL
arguments of $T_1$ are included
(\S~\ref{section:rpls:relations:inclusion}) in the corresponding
arguments of $T_2$.

\bfhead{Types that instantiate different classes:} $T_1$ is a subtype
of $T_2$ if $T_2$ is a \emph{direct supertype} or \emph{indirect
  supertype} of $T_1$.

\ithead{Direct supertypes:} $T_2$ is a direct supertype of $T_1$ if a
type with $T_2$'s class or interface appears in an \kwd{extends} or
\kwd{implements} clause of $T_1$'s class or interface, and that type
is a subtype of $T_2$ after substituting arguments for parameters
according to $T_1$.

\ithead{Indirect supertypes:} $T_2$ is an indirect supertype of $T_1$
if there is a chain of direct supertypes connecting $T_1$ to $T_2$.

\subsubsection{Casts}
\label{section:types:class:casts}

A cast from one class or interface type to another is legal if the
cast would be legal for the DPJ erasures
(\S~\ref{section:types:class:dpjerasure}) of the types.  Therefore
casts that violate type preservation are allowed.  Any unsafe cast
will generate the usual ``unchecked or unsafe operations'' warning
that Java 1.6 gives for unsound generic casts.

\subsubsection{Owner RPLs}
\label{section:types:class:owner-rpls}

If \emph{class-type} is a class type, then the \emph{owner RPL} of
\emph{class-type} is (1) the RPL bound to the first parameter of
\emph{class-type}, if \emph{class-type} has parameters; otherwise (2)
\kwd{Root}.

\subsubsection{Captured Types}
\label{section:types:class:capture}

The compiler extends Java 1.6 type capture (JLS \S~5.1.10) to capture
RPL parameters as well as generic type parameters.  If \emph{type} is
a class type with \emph{rpls} as its RPL arguments, then the capture
of \emph{type} is defined as follows:
\begin{enumerate}
\item Take the Java 1.6 capture of \emph{type}, possibly
  substituting for the generic type arguments but keeping the same RPL
  arguments.
\item In the result of step 1, for each RPL \emph{rpl} in \emph{rpls}
  that is partially specified (i.e., contains \kwd{*} or \kwd{[?]}),
  replace that RPL with a fresh RPL parameter constrained to be
  included in \emph{rpl}.
\end{enumerate}


\subsection{Array Types}

\subsubsection{Valid Array Types}
\label{arrays:valid}

In DPJ, the array type is given by \emph{base-type} followed by one or
more instances of \emph{brackets}, where \emph{base-type} is a
primitive, class, or interface type; and each instance of
\emph{brackets} is a pair of brackets (\kwd{[]}) optionally followed
by an RPL argument \kwd{<}\emph{rpl}\kwd{>}, optionally followed by
\kwd{\#}\emph{index-var}, where \emph{index-var} is an identifier that
declares an index variable.  An array type is valid if (1)
\emph{base-type} is valid, treating all index variables appearing in
the type as integer variables in scope; and (2) the RPL of every
\emph{brackets} is valid, treating all index variables appearing in
that \emph{brackets} and all \emph{brackets} to the left of that one
as integer variables in scope.

\noindent
\textbf{Default index variable declaration:} If no index variable
declaration \kwd{\#}\emph{index-var} appears in a particular instance
of \emph{brackets}, then that instance of \emph{brackets} is treated
as if it had the declaration \kwd{\#\_} (pound underscore).  For
example, \kwd{int[]<[\_]>} is a valid array type; it is equivalent to
\kwd{int[]<[\_]>\#\_}.

\subsubsection{New Arrays}
\label{arrays:new}

A new array expression is \kwd{new} followed by \emph{base-type} and
one or more instances of \emph{new-brackets}, where \emph{base-type}
is a primitive, class, or interface type; and \emph{new-brackets} is
the same as \emph{brackets} as described in
\S~\ref{arrays:valid}, except that an integer length expression
\emph{expr} may optionally appear between the brackets.  A new array
expression is valid if (1) the array type generated by deleting the
length expressions is valid (\S~\ref{arrays:valid}); and (2) the
first $n$ instances of \emph{new-brackets} contain a length
expression, for some $n \geq 1$, while the rest do not.

\subsubsection{Subtypes}

If $T_1$ and $T_2$ are array types, then the compiler
checks whether $T_1$ is a subtype of $T_2$ as follows:
\begin{enumerate}
\item If the RPL specified in the leftmost \emph{brackets} of
  $T_1$ is not included in
  (\S~\ref{section:rpls:relations:inclusion}) the RPL specified in the
  leftmost \emph{brackets} of $T_2$, then the answer is no.
\item Otherwise, check whether the element type of $T_1$ is a subtype
  of the element type of $T_2$ (\S~\ref{section:types:class:subtypes}
  or this section), but require equivalence of corresponding RPL
  arguments (\S~\ref{section:rpls:relations:equiv}), instead of just
  inclusion, in doing this check.
\end{enumerate}
We require equivalence of RPL arguments for all but the leftmost RPL
because it is not sound in general to allow subtype assignments into
array cells.

\subsubsection{Casts}

Array type casts work in the same way as class type casts
(\S~\ref{section:types:class:casts}).

\subsection{Typing Expressions}

\subsubsection{Field Access}
\label{section:types:exp:field}

The compiler computes the type of a field access expression
\emph{selector-exp}\kwd{.}\emph{field-name} as follows: (1) compute
the type \emph{selector-type} of \emph{selector-exp}; (2) capture the
type (\S~\ref{section:types:class:capture}) to generate type
\emph{captured-selector-type}; (3) look up the type \emph{field-type}
of \emph{field} based on the class $C$ named in
\emph{captured-selector-type}; (4) make the following substitutions in
\emph{field-type}: (a) the type and RPL arguments of
\emph{captured-selector-type} for the type RPL parameters of $C$; and
(b) \emph{selector-rpl} for \kwd{this}.  If \emph{selector-exp} is a
\kwd{final} local variable \emph{var}, then \emph{selector-rpl} is
\emph{var}; otherwise, \emph{selector-rpl} is a fresh capture
parameter (\S~\ref{section:types:class:capture}) constrained to be
included in \emph{owner-rpl}\kwd{:*}, where \emph{owner-rpl} is the
owner RPL (\S~\ref{section:types:class:owner-rpls}) of
\emph{captured-selector-type}.

\subsubsection{Array Access}
\label{section:types:exp:array}

If expression $e$ has array type $T$ (\S~\ref{arrays:valid}), then we
construct the type and region of array access expression
\kwd{$e$[$e'$]} as follows:
\begin{itemize}
\item To construct the type, do the following: (a) concatenate
  \emph{base-type} of $T$ with all instances of \emph{brackets} in $T$
  but the leftmost to form \emph{elt-type}; and (b) if an index
  variable $i$ is declared in the leftmost instance of
  \emph{brackets}, then substitute $e'$ for all instances of $i$
  appearing in \emph{elt-type}.
\item To construct the RPL, do the following: (a) let \emph{rpl} be
  the RPL appearing in the leftmost instance of \emph{brackets} in
  $T$, or \kwd{Root} if no RPL appears there; and (b) if an index
  variable $i$ appears in the leftmost instance of \emph{brackets},
  then substitute $e'$ for all instances of $i$ appearing in
  \emph{rpl}.
\end{itemize}

\subsubsection{Method Invocation}
\label{section:types:exp:invocation}

\textbf{Explicit Type and RPL Arguments:} The compiler computes the
type of a method invocation expression
%
\begin{description}
%
\item
  \emph{selector-exp}\kwd{.<}\emph{type-args}\kwd{,}\emph{rpl-args}\kwd{>}\emph{method-name}\kwd{(}\emph{args}\kwd{)}
%
\end{description}
%
as follows: (1) compute the type \emph{selector-type} of
\emph{selector-exp} and the types \emph{arg-types} of \emph{args}; (2)
capture \emph{selector-type} (\S~\ref{section:types:class:capture}) to
generate type \emph{captured-selector-type}; (3) look up the method
symbol based on the method name, \emph{selector-type}, and
\emph{arg-types}, and find the return type \emph{return-type} of the
method; (4) make the following substitutions in \emph{return-type}:
(a) the type and RPL arguments of \emph{captured-selector-type} for
the type and RPL parameters of \emph{return-type}; (b)
\emph{type-args} and \emph{rpl-args} for the type and RPL parameters
of the method; (c) for every expression \emph{int-exp} in \emph{args}
of a type assignable to \kwd{int}, \emph{int-exp} for the
corresponding formal parameter of the method; and (d)
\emph{selector-rpl} for \kwd{this}, where if \emph{selector-exp} is a
\kwd{final} local variable \emph{var}, then \emph{selector-rpl} is
\emph{var}, and otherwise \emph{selector-rpl} is a fresh capture
parameter (\S~\ref{section:types:class:capture}) constrained to be
included in \emph{owner-rpl}\kwd{:*}, where \emph{owner-rpl} is the
owner RPL (\S~\ref{section:types:class:owner-rpls}) of
\emph{selector-type}.

\noindent
\textbf{Inferred or Missing Type Arguments, RPL Arguments, and/or
  \emph{selector-exp}:} If there are no explicit type or RPL
arguments, the compiler infers them (if necessary) as stated in
\S~\ref{section:class-def:params:method} and proceeds as stated above
using the inferred arguments in step (4)(b).  If there is no
\emph{selector-exp}, then as in ordinary Java, the implied selector is
\kwd{this}.

