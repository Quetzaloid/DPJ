\section{Parallel Control Flow%
\label{sec:parallel}}
\cutname{parallel.html}

DPJ employs a fork-join model of parallelism.  That means that a task
may launch several parallel tasks (the fork) and all must complete
(the join) before the launching task can continue.  Recursive forking
is supported, i.e., tasks can launch other tasks to arbitrary depth.
There are two ways to fork tasks: \kwd{cobegin}, which forks several
statements as parallel tasks, and \kwd{foreach}, which forks groups of
consecutive loop iterations as parallel tasks.

\subsection{\kwd{cobegin}}
\label{sec:parallel:cobegin}
\cutname{parallel.cobegin.html}

The syntax of the \kwd{cobegin} statement is as follows, where $S$ is
a DPJ statement:
%
\begin{description}
\item \kwd{cobegin} $S$
\end{description}
%
If $S$ is any statement but a block enclosed in curly braces $\kwd{\{}
\ldots \kwd{\}}$, or if $S$ is a block consisting of a single
statement, then the \kwd{cobegin} just executes the statement $S$.
Otherwise, the component statements of $S$ are run as parallel tasks.
There is an implicit barrier (join) at the end of the \kwd{cobegin}
statement, so that all the component tasks must finish execution
before the parent task executes the statement after the \kwd{cobegin}.

For example, the following code executes statements \kwd{S1} and
\kwd{S2} in parallel:
%
\begin{dpjlisting}
cobegin {
  S1;
  S2;
}
S3;
\end{dpjlisting}
%
Because \kwd{S3} appears after the \kwd{cobegin}, both \kwd{S1} and
\kwd{S2} are guaranteed to finish before \kwd{S3} is executed.

In order to guarantee deterministic execution, the compiler checks the
component statements of a \kwd{cobegin} for noninterference
(Section~\ref{sec:effects:nonint}).  If interference is discovered,
then the compiler issues a warning.  For example, the following code
would cause a warning, because of the interfering writes to variable
\kwd{x} in the parallel tasks:
%
\begin{dpjlisting}
class C {
  void m() {
    int x;
    cobegin {
      x = 0;
      x = 1;
    }
  }
}
\end{dpjlisting}
%
On the other hand, the following code would compile with no warning:
%
\begin{dpjlisting}
class C {
  void m() {
    int x, y;
    cobegin {
      x = 0;
      y = 1;
    }
  }
}
\end{dpjlisting}
%

The \kwd{cobegin} statement is most often used with recursion.  The
following pattern is typical:
%
\begin{dpjlisting}
void recursiveMethod(...) {
  if (...) {
    // do base case sequentially
  } else {
    cobegin {
      recursiveMethod(...);
      recursiveMethod(...);
    }
  }
}
\end{dpjlisting}
%
Note that the ``recursion cutoff'' (i.e., when the base case takes
over) has to be programmed manually.  For example, in a parallel
recursive sort, the condition might say to do the sort sequentially
when the input array reaches a certain minimum size.  The minimum
should be chosen so that (1) there are enough parallel tasks for the
scheduler to balance the computation, but (2) the task creation
overhead is not unduly large.  See \tutorial\ for more
examples of \kwd{cobegin}.

\subsection{\kwd{foreach}%
\label{sec:parallel:foreach}}
\cutname{parallel.foreach.html}

\kwd{foreach} operates similarly to \kwd{cobegin}, except that the
parallel tasks are the iterations of a loop, instead of the component
statements of a block.  The granularity of parallelism (i.e., how many
loop iterations to execute in a task) is controllable by the
programmer.

\subsubsection{Writing the \kwd{foreach} loop%
\label{sec:parallel:foreach:writing}}
\cutname{parallel.foreach.writing}

The syntax of the \kwd{foreach} loop is as follows:
%
\begin{description}
\item \kwd{foreach (int
    }\emph{index-var}\kwd{ in }\emph{start}\kwd{, }\emph{length}\kwd{,
    }\emph{stride}\kwd{) }\emph{body}\kwd{;}
\end{description}
%
\kwd{index-var} is an identifier, \emph{start}, \emph{length}, and
\emph{stride} are integer expressions, and \emph{body} is a statement.
The \emph{stride} expression is optional, and the default is 1. If it
appears, the \emph{stride} expression must evaluate to an integer
greater that 0.  The \kwd{foreach} loop executes the loop body once
for each element of an iteration space given by all integers
$\nonterm{stride} \cdot i$ such that $i$ ranges between $0$ and
$\nonterm{length}-1$, inclusive.  The variable \emph{index-var} may
not be modified in the loop body.

For example, the following code sets the cells of array \kwd{A} with
even indices to 0:

\begin{dpjlisting}
foreach (int i in 0, A.length, 2) {
  A[i] = 0;
}
\end{dpjlisting}

The compiler performs the following noninterference check for indexed
\kwd{foreach} loops:
\begin{enumerate}
\item Infer the effect set of \emph{body}
  (Section~\ref{sec:effects:stmt-exp}).
\item Create a copy of the effect set generated in (1), but replace
  every occurrence of \emph{index-var} with a fresh variable that is
  known to be unequal to \emph{index-var}.  This simulates the effects
  generated by two distinct iterations of the loop, for which
  \emph{index-var} will have distinct values.
\item Check whether the effect sets generated in (1) and (2) are
  noninterfering (Section~\ref{sec:effects:nonint}).
\end{enumerate}
If interference is detected, the compiler issues a warning.

The DPJ runtime divides the \kwd{foreach} iterations into parallel
work according to the programmer-specified granularity
(Section~\ref{sec:parallel:foreach:granularity}).  As in the case of
\kwd{cobegin}, there is an implicit barrier at the end of the
\kwd{foreach}; i.e., all tasks created by the \kwd{foreach} must
complete before any code following the \kwd{foreach} is executed.

\subsubsection{Controlling the granularity of parallelism%
\label{sec:parallel:foreach:granularity}}
\cutname{parallel.foreach.granularity.html}

In most cases, it would be inefficient to issue every loop iteration
as a parallel task.  For example, consider a loop iterating over an
array with 100,000 elements on a machine with 10 cores.  If 100,000
tasks were issued to do this computation, the task creation overhead
would swamp the parallelism.

Instead, the DPJ compiler allows the user to control the granularity
of task creation.  The user can specify a minimum task size, in terms
of the number of iterations.  The compiler recursively splits the
iteration space, until the minimum size is reached.  For example, if
the cutoff were specified to be 1000 in the example above, then the
compiler would divide the loop into 100 tasks of 1000 iterations
each.

If the \kwd{foreach} computation is perfecly balanced (i.e., each
iteration does exactly the same amount of work), then it makes sense
to make the cutoff $I/T$, where $I$ is the number of loop iterations
and $T$ is the number of threads available to the DPJ runtime (usually
equal to the number of cores on the host machine).  Using this
strategy, in the example with 100,000 iterations and 10 cores, the
cutoff should be 10,000.

If the computation is not perfectly balanced, then a better strategy
is to \emph{over-decompose} the computation, specifying more tasks
than the number of available threads.  This allows the scheduler to
schedule multiple tasks of varying size per thread, in a such a way
that the aggregate amount of work per thread is roughly balanced.  As
in the case of the \kwd{cobegin} cutoff
(Section~\ref{sec:parallel:cobegin}), the programmer should choose a
minimum task size that creates enough parallel work without incurring
too much task creation overhead.

The precise mechanism for specifying the number of DPJ threads and the
\kwd{foreach} cutoff is implementation-dependent.
\installmanual\ explains how this mechanism works for the \kwd{dpjc}
compiler.


