\section{Classes and Interfaces%
\label{sec:classes}}
\cutname{classes.html}

DPJ classes and interfaces are identical to classes and interfaces in
Java, with the following additional features:
%
\begin{itemize}
%
\item New members called \emph{class region name declarations}
  (\S~\ref{sec:classes:region-names}) declare region names at class or
  interface scope that can be associated with class fields
  (\S~\ref{sec:classes:field-region-spec}).
%
\item Methods have several new features to support regions and effects
  (\S~\ref{sec:classes:methods}).
%
\item Classes, interfaces, and methods support \emph{region
  parameters}, so that different objects of the same class or
  interface, and different invocations of the same method, can use
  different regions (\S~\ref{sec:classes:params}).
%
\end{itemize}
%
Throughout this section, we use the term ``class'' with the
understanding that the concepts apply identically to interfaces,
unless otherwise noted.

\subsection{Class Region Name Declarations%
\label{sec:classes:region-names}}
\cutname{classes.region-names.html}

A \emph{class region name declaration} may appear as a class or
interface member.  A class region name declaration consists of the
keyword \kwd{region} followed by one or more identifiers separated by
commas. For example, the following declaration declares a region name
\kwd{Data} within the class \kwd{Element}:

\begin{dpjlisting}
class Element {
	region Data;
	...
}
\end{dpjlisting}

Class region name declarations function like static class members in
Java (though the keyword \kwd{static} need not be used with a class
region declaration --- if it is, it has no effect).  In particular,
subject to visibilty restrictions, other packages and classes can
refer to the declared names by prepending the proper package and class
qualifiers.  For example, outside of class \kwd{Element}, region
\kwd{Data} could be referred to as \kwd{Element.Data}.

As with Java fields and methods, the programmer may control the access
to the declared names with the qualifiers \kwd{public}, \kwd{private},
or \kwd{protected}.  For example, the following code declares two
regions that can be referred to from anywhere in the program:

\begin{dpjlisting}
class Node {
	public region Left, Right;
	...
}
\end{dpjlisting}

Class region name declarations are available for use in RPLs
(\S~\ref{sec:rpls}) in the scope where they are visible.  A class
region name standing alone is a particular case of an RPL (see
\S~\ref{sec:rpls:basic:class}).

\subsection{Field RPL Specifiers%
\label{sec:classes:field-region-spec}}
\cutname{classes.field-region-spec.html}

Every class field in DPJ resides in a \emph{region}, named with a
region path list (\S~\ref{sec:rpls}).  This is a fundamental aspect of
DPJ versus plain Java; it allows the specification and checking of
effects (\S~\ref{sec:effects}). There are two ways to specify the RPL
of a field: (1) with an explicit \emph{field RPL specifier}; and (2)
by using the \emph{default field RPL}.

\bfhead{Explicit field RPL specifiers} A field RPL specifier has the
following form:
%
\begin{description}
\item \kwd{in} \emph{rpl}
\end{description}
%
\emph{rpl} is an RPL, as defined in \S~\ref{sec:rpls}.  If the field
RPL specifier is present, then it must appear immediately after the
field name (and before the field initializer expression, if there is
one).  For example:
%
\begin{dpjlisting}
class FieldRPLSpecifiers {
    region X, Y;             // X and Y name regions
    int x in X;              // field x is in region X
    boolean y in Y = false;  // field y is in region Y
}
\end{dpjlisting}
%
Notice that because a class region name declaration functions like a
static class member (\S~\ref{sec:classes:region-names}), it creates a
single region name \emph{per class}, not per object.  Therefore, a
field RPL specifier such as \kwd{int x in X} assigns the same
region \kwd{FieldRPLSpecifiers.X} to the field of \emph{every object
  instance} of class \kwd{FieldRPLSpecifiers} created at runtime.
\S~\ref{sec:classes:params} explains how to use \emph{region
  parameters} to assign different regions to the fields of different
objects of the same class.


\emph{Final fields} Effects on \kwd{final} fields are ignored
(\S~\ref{sec:effects:stmt-exp}), because their values never change
after initialization.  Therefore, field RPL specifiers are not
meaningful for \kwd{final} fields.  If a field is declared
\kwd{final}, then an RPL specifier may be given for the field; but if
so, it is ignored.

\bfhead{Default field RPL} If a field has no explicit RPL specifier,
then its RPL is \kwd{Root}.  The name \kwd{Root} is always in scope
(\S~\ref{sec:rpls:basic:root}).  For example, in the following code,
field \kwd{x} is in region \kwd{Root}:
%
\begin{dpjlisting}
class DefaultRegionSpecifier {
    // Equivalent to int x in Root = 5
    int x = 5;
}
\end{dpjlisting}


\subsection{Methods%
\label{sec:classes:methods}}
\cutname{classes.methods.html}

DPJ adds the following features to Java methods:
%
\begin{itemize}
%
\item Every method (\S~\ref{sec:classes:methods:summaries}) and
  constructor (\S~\ref{sec:classes:methods:summaries-constructor})
  summarizes its effects, either explicitly or through a default
  effect summary.
%
\item The programmer may declare and use \emph{local region names}
  inside a method body (\S~\ref{sec:classes:methods:local-regions}),
  to express effects that do not escape the method scope.
%
\item Methods may be marked \kwd{commutative} to indicate that their
  effects commute, even though they have conflicting reads and writes
  (\S~\ref{sec:classes:methods:commutative}).
%
\end{itemize}

\subsubsection{Effect Summaries (Non-Constructor Methods)%
\label{sec:classes:methods:summaries}}
\cutname{classes.methods.summaries}

Every DPJ method must summarize its effects.  The compiler uses the
summaries to check noninterference of effect
(\S~\ref{sec:effects:nonint}).  This section discusses effect
summaries for non-constructor methods.
\S~\ref{sec:classes:methods:summaries-constructor} describes effects
summaries for constructors.  There are two ways to summarize a
method's effects: (1) with an explicit method effect summary; or (2)
with the default method effect summary.

\bfhead{Explicit method effect summaries} An explicit method effect
summary appears immediately after the method's value parameters and
before the \kwd{throws} clause, if any.  It has the form given in
\S~\ref{sec:effects:summaries}.  For example:
%
\begin{dpjlisting}
class Summaries {
  region X, Y;
  int x in X;
  int y in Y;
  // pureMethod has no effect on the heap
  int pureMethod(int y) pure { return y+1; }
  // throwsMethod reads X and throws an exception
  void throwsMethod() reads X throws Exception {
    if (x != 0) throw new Exception();
  }
  // readWriteMethod reads Y and writes X
  void readWriteMethod() reads Y writes X {
    x = y;
  }
}
\end{dpjlisting}
%

A method effect summary must represent all the effects of the method
body.  More precisely, the actual method effects must be a
\emph{subeffect} (\S~\ref{sec:effects:subeffects}) of the summarized
effects.  The actual effects are computed as described in
\S~\ref{sec:effects:stmt-exp}, \emph{except} that any local effect
(\S~\ref{sec:effects:local}) can be omitted from the summary.  The
representation can be conservative, i.e., it is permissible to include
effects in the summary that can never occur in executing the method.
But if any effect of any possible execution of the method is omitted
from the summary, it is a compile-time error.  

For example:
%
\begin{dpjlisting}
class MoreSummaries {
  region X;
  int x in X;
  // OK, summary is conservative
  int readsMethod() writes X {
    // Write effects cover reads
    return x;
  }
  // Error!  Read effect must be reported
  int pureMethod() pure {
    return x;
  }
}
\end{dpjlisting}
%

A method effect summary must also represent all the effects of any
method that overrides it.  If it does not, then there is a
compile-time error.  To see why, consider the following classes:
%
\begin{dpjlisting}
class SuperClass {
  region X;
  int x in X;
  void method(int x) writes X {
    this.x = x;
  }
}
class SubClass extends SuperClass {
  // Compile-time error:  pure does not cover writes X
  void method(int x) pure {
    // Do nothing
  }
}
\end{dpjlisting}
%
The method \kwd{method} defined in \kwd{SubClass} overrides the method
\kwd{method} defined in \kwd{SuperClass}.  There is an error in the
code, because the effect \kwd{pure} of the subclass method does not
cover the effect \kwd{writes X} of the superclass method.  While this
may look innocuous (after all, the \kwd{SubClass} version of method
really has no effect!), suppose we allowed the code given above and
then wrote the following method:
%
\begin{dpjlisting}
void callMethod(SubClass subClass) pure {
  subClass.method();
}
\end{dpjlisting}
%
Based on the previous code, this looks fine: we are calling
\kwd{subClass.method}, which has effect \kwd{pure}.  But does it?
Because of polymorphic dispatch, the \emph{runtime} type of the object
bound to \kwd{subClass} could be \kwd{SuperClass}.  And in that case,
the call to \kwd{method} would invoke the \emph{superclass} version of
\kwd{method}, which has the effect \kwd{writes X}.  The write to
region \kwd{X} would be ``hidden'' by the polymorphic dispatch.  To
prevent this kind of hiding, DPJ requires that superclass effects
cover subclass effects.  Notice that if we change the effect summary
in the \kwd{SubClass} version of \kwd{method} to \kwd{writes X}, as
DPJ requires, then this problem goes away.

\bfhead{Default effect summary} Any DPJ method may be written with no
explicit method effect summary.  In particular, an ordinary Java
method is always a valid DPJ method.

If a method lacks an explicit effect summary, then the compiler
assigns it the default summary \kwd{writes Root:*}.  This is the most
conservative possible effect summary; it says that the method may read
or write to any globally-visible memory location.  The default effect
summary is always valid, because it covers all possible effects of any
method.  For example, the following code is valid:
%
\begin{dpjlisting}
class DefaultEffectSummary {
    // Equivalent to 'void method() writes Root:*'
    // OK, because effects may be overreported
    void defaultSummary() {
        // No actual effect!
    }
    void parallel() {
        // Reports interference, even though defaultSummary has no
        // actual effect!
        cobegin {
           defaultSummary();
           defaultSummary();
        }
    }
}
\end{dpjlisting}

In general, the default effect summary is too coarse-grained for
methods that are called either directly or indirectly inside a
parallel task: as shown in the example, it causes DPJ to detect and
warn about interfering writes, even if the method's actual effects are
not interfering.  However, for methods that are only ever used in
sequential parts of the program, the default effect summary saves
effort, because the effects of those methods are not important.

\subsubsection{Effect Summaries (Constructors)%
\label{sec:classes:methods:summaries-constructor}}
\cutname{classes.methods.summaries-constructor}

Constructors are special methods; therefore they must also summarize
their effects.  All the rules given in
\S~\ref{sec:classes:methods:summaries} apply, with one important
exception: the effect summary of a constructor does \emph{not} have to
report any initialization effects on fields of the object being
constructed.  For example, the following code is legal:
%
\begin{dpjlisting}
class ConstructorExample {
    region X, Y;
    int x in X, y in Y;
    // Effect 'pure' is valid, because initialization effects
    // on x and y don't have to be reported
    ConstructorExample(int x, int y) pure {
        this.x = x;
        this.y = y;
    }
}
\end{dpjlisting}
%
This works because the DPJ type and effect system guarantees that if
one parallel task calls a constructor, then the fields of a
constructed object are never read by any other parallel task until the
first task is finished.  In particular, that means that no other
parallel task can ever see object fields in an uninitialized state.
So there can be no interference due to initialization effects of
constructors.

However, if we wrote the same initializer as a non-constructor method,
then the rules in \S~\ref{sec:classes:methods:summaries} would apply,
and we would have to write the effects:
%
\begin{dpjlisting}
class NonConstructorExample {
    region X, Y;
    int x in X, y in Y;
    // Effect 'pure' would cause a compile error here
    static void initialize(int x, int y) writes X, Y {
        this.x = x;
        this.y = y;
    }
}
\end{dpjlisting}

\subsubsection{Local Region Name Declarations%
\label{sec:classes:methods:local-regions}}
\cutname{classes.methods.local-regions}

A region name declaration may appear as a statement in the body of a
method.  This kind of declaration is called a \emph{local region name
  declaration}.  Like a class region name declaration
(\S~\ref{sec:classes:region-names}), a local region name declaration
consists of the keyword \kwd{region} followed by a comma-separated
list of identifiers.  The declared names are available for use in RPLs
(\S~\ref{sec:rpls}) in the scope of the enclosing statement block.  In
particular, a local region name standing alone is a valid RPL
(\S~\ref{sec:rpls:basic:local}).

For example, the following code declares
region names \kwd{A} and \kwd{B} that are available for use as region
names in the scope of method \kwd{localRegionNames}:
%
\begin{dpjlisting}
void localRegionNames() {
	region A, B;
	...
}
\end{dpjlisting}
%

Because of their limited scope, local names (and RPLs constructed from
them) cannot appear in field RPL specifiers
(\S~\ref{sec:classes:field-region-spec}).  They can be used
\emph{only} in arguments to class or method region parameters
(\S~\ref{sec:classes:params}).  Their purpose is to indicate that the
effects of operating on the class object, or invoking the method, are
local to the enclosing method, and need not to be reported in the
effect summary of the enclosing method or constructor
(\S\S~\ref{sec:classes:methods:summaries},~\ref{sec:classes:methods:summaries-constructor}).
This technique is called \emph{effect masking}.  For example:
%
\begin{dpjlisting}
class EffectMasking<region R> {
  int x in R;
  // method has no globally-visible effect
  void method() pure {
    // Declare region r local to method
    region r;
    // Use local region to create new EffectMasking object
    // 'masking' cannot escape method
    EffectMasking<r> masking = new EffectMasking<r>();
    // Effect 'writes r' is masked from callees
    masking.x = 0;
  }
}
\end{dpjlisting}
%
In this example the masked effect is somewhat useless.  But there are
plenty of realistic cases where objects are created and assigned to
temporarily to support some computation, then thrown away because only
the final result is needed by the callee.  By creating the temporary
objects with local regions, the callee can reduce its effect
signature, minimizing potential interference and making it more useful
inside parallel tasks.  See \tutorial\ for more details.

Like Java local variables, a local region name declared in a statement
block is in scope only in that block; its scope ends when the block
ends.  For example, the following code would cause a compile error,
because the name \kwd{OutOfScopeRegion} is not in scope where it is
used:
%
\begin{dpjlisting}
class ScopeExample<region R> {
    void method() {
        {
            region OutOfScopeRegion;
            // Scope ends here
        }
        // Error:  OutOfScopeRegion is no longer in scope
        ScopeExample<OutOfScopeRegion> x = null;
    }
}
\end{dpjlisting}

\subsubsection{Commutative Methods%
\label{sec:classes:methods:commutative}}
\cutname{classes.methods.commutative}

The keyword \kwd{commutative} may appear as a method qualifier, before
the return type and before the type and/or region parameters of the
method, if any:

\begin{dpjlisting}
commutative int m(...) { ... }
\end{dpjlisting}

The \kwd{commutative} qualifier is a programmer-specified guarantee
that any invocation of the method commutes with itself.  It is
typically used for commutative operations on concurrent data
structures such as counter updates, set inserts, histograms, and
reductions.  These operations write to shared data in a way that
``looks interfering'' to the DPJ effect system, but due to the
semantics of the data structure (which the effect system does not know
about), still preserves determinism.  For example, here is a use of
the \kwd{commutative} qualifier to write a simple counter class:

\begin{dpjlisting}
class Counter<region R> {
  private int count in R;
  void clear() writes R { count = 0; }
  commutative synchronized void increment() writes R {
    ++count;
  }
  int getCount() reads R { return count; }
}
\end{dpjlisting}

When a method $m$ is labeled \kwd{commutative}, the DPJ compiler
treats the effects of multiple invocations of $m$ as noninterfering,
even if the read and write effects by themselves, without the
\kwd{commutative}, would be interfering.  For example, in the case of
the counter class, concurrent invocations of \kwd{increment} have
interfering writes to region \kwd{R}.  However, the \kwd{commutative}
annotation tells the DPJ compiler to ``ignore'' that interference.
So, for example, the following code compiles with no errors or
warnings:

\begin{dpjlisting}
Counter<Root> counter = new Counter<Root>();
foreach(int i in 0, 10)
  counter.increment();
\end{dpjlisting}

Note that the \kwd{commutative} qualifier \emph{only} tells the DPJ
compiler that it is safe to ignore interference.  It does \emph{not}
introduce any special synchronization or other concurrency control.
In particular, it is the programmer's responsibility when using
\kwd{commutative} to ensure two things:
%
\begin{enumerate}
\item The method is properly synchronized so that concurrent accesses
  to the method behave as though they have occurred in sequence (i.e.,
  the accesses have \emph{serializable} semantics).  In the example
  above, this property is enforced by making method \kwd{increment}
  \kwd{synchronized}.  Without the \kwd{synchronized} keyword,
  concurrent invocations of \kwd{increment} would have a
  read-modify-write race, and some of the updates could be lost.

\item The method behaves so that either order of a pair of invocations
  produces the same result.  In the example above, this property holds
  because incrementing a counter twice, in either order, has the same
  result (i.e., the final counter value is 2 more than the starting
  value).
\end{enumerate}
%
Typically, in user code, the \kwd{commutative} annotation is used for
simple commutative read-modify-write operations like the one
illustrated above.  More complicated operations, e.g. set or tree
inserts, where the commutativity property is more subtle to verify,
would typically be provided by library or framework code.  Only very
skilled programmers should attempt to ``roll their own''
implementation of such commutative operations, as the potential for
subtle bugs is high, and the DPJ effect system provides no assistance
in checking for such bugs.

\subsection{Region Parameters%
\label{sec:classes:params}}
\cutname{classes.params.html}

DPJ extends Java's generic type parameters by allowing \emph{region
  parameters} in class and method definitions.  Class region
parameters become bound to actual regions when the class is
instantiated to a type (\S~\ref{sec:types:class:instant}).  Method
region parameters become bound to actual regions when the method is
invoked (\S~\ref{sec:classes:params:method}).

\subsubsection{Class Region Parameters%
\label{sec:classes:params:class}}
\cutname{classes.params.class.html}

\bfhead{Declaring class region parameters} As in ordinary generic
Java, the class parameters are given as a comma-separated list of
identifiers enclosed in angle brackets (\kwd{<>}) immediately after
the class name.  The new features are as follows:
%
\begin{itemize}
%
\item Both type and region parameters may appear.  If both appear, all
  the region parameters must follow the type parameters.
%
\item The first region parameter \emph{must} be preceded by the
  keyword \kwd{region}.  Any of the other region parameters \emph{may}
  be preceded by the keyword \kwd{region}.
%
\item Any of the type parameters \emph{may} be preceded by the keyword
  \kwd{type}.
%
\end{itemize}
%
The keyword \kwd{type} is provided for convenience in distinguishing
type from region parameters; but to preserve compatibility with
ordinary Java syntax, it is never required.

For example, the following are valid class declarations:
%
\begin{dpjlisting}
// Type parameter T
class A<T> {}            
class B<type T> {}
// Region parameter R
class C<region R> {}
// Type parameter T, region parameter R
class D<T, region R> {}
class E<type T, region R> {}
// Type parameters T1, T2, region parameters R1, R2
class F<type T1, T2, region R1, R2> {}
\end{dpjlisting}

\bfhead{Using class region parameters} A class region parameter is
available for use in RPLs (\S~\ref{sec:rpls:param}) within the class
definition.  In particular, a class region parameter standing alone is
a valid RPL.  For example, the following code declares a class
\kwd{ParamInField} with a region parameter \kwd{R} and uses \kwd{R} in
a field RPL specifier (\S~\ref{sec:classes:field-region-spec}):
%
\begin{dpjlisting}
class ParamInField<region R> {
    public int x in R;
}
\end{dpjlisting}
%
This code says that the region of field \kwd{x} of an object
instantiated from the \kwd{ParamInField} class is given by the RPL
provided as an argument to \kwd{R} in the type of the object.  For
example, this code creates a fresh \kwd{ParamInField} object called
\kwd{p}, such that the field \kwd{p.x} is in region \kwd{Root}:
%
\begin{dpjlisting}
ParamInField<Root> p = new ParamInField<Root>();
\end{dpjlisting}

As another example of using parameters, the following code declares a
class region parameter \kwd{R} and uses it to instantiate a type
(\S~\ref{sec:types:class:instant}):
%
\begin{dpjlisting}
class ParamInType<region R> {
    ParamInType<R> x in R;
}
\end{dpjlisting}
%
And this code creates a fresh \kwd{ParamInType} object whose field
\kwd{x} has type \kwd{ParamInType<Root>}:
%
\begin{dpjlisting}
ParamInType<Root> p = new ParamInType<Root>();
\end{dpjlisting}
%
This pattern is very useful for creating chains or graphs of objects,
all in the same region.  For example, without the region parameter
\kwd{R}, there would be no way to get the object pointed to by \kwd{x}
in the same region as the parent object.  The RPL specifier \kwd{x in
  R} is insufficient, because that says only that the field \kwd{x} is
in \kwd{R}, not the object that it points to.  To get both the field
and the object in the same region, you have to write
%
\begin{dpjlisting}
ParamInType<R> x in R
\end{dpjlisting}
%
as shown.  Moreover, DPJ provides the flexibility to assign different
regions to the field pointing to the object and the object itself, for
example:
%
\begin{dpjlisting}
ParamInType<r1> x in r2
\end{dpjlisting}
%
This allows for finer-grain partitioning of memory than if the field
and the object always had the same region.

%
As in generic Java, a new declaration of a parameter named \kwd{R}
hides any parameter named \kwd{R} that is already in scope.  For
example, the following two definitions of the class \kwd{Hiding} are
equivalent:
%
\begin{dpjlisting}
// First version
class Hiding<region R1> {
    class Inner<region R2> {
        Hiding<R2> x;
    }
}
// Second version
class Hiding<region R> {
    class Inner<region R> {
        Hiding<R> x;
    }
}
\end{dpjlisting}
%
In both versions we are declaring a parameter for the outer class,
declaring a parameter for the inner class, and using the inner class
parameter to instantiate a type inside the inner class.  In the first
version we have given distinct names to the two parameters; while in
the second version we have used the same name.

Also as with Java generic type parameters, a class region parameter
may not be used in a static context.  This is because DPJ region
information (like Java generic information) is \emph{erased} during
the compilation process.  In particular, only one set of code is
generated for all types instantiating a given class.  If region
parameters could be used in a static context, then different code
would have to be generated for each different type (as with C++
templates).
%
For example, the following code is invalid, because it attemps to use
region \kwd{R} in a static context:
%
\begin{dpjlisting}
class Outer<region R> {
    public static class Inner {
        // Not allowed!
        public int x in R;
    }
}
\end{dpjlisting}

Fortunately, this limitation is easy to work around.  The standard way
(again, as with Java generics) is to introduce a new parameter for the
inner class, and write \kwd{Inner<R1>} to place \kwd{x} in region
\kwd{R1}:

\begin{dpjlisting}
class Outer<region R1> {
    public static class Inner<region R2> {
        public int x in R2;
    }
}
\end{dpjlisting}


\subsubsection{Method Region Parameters%
\label{sec:classes:params:method}}
\cutname{classes.params.method.html}

\bfhead{Declaring method region parameters} As in ordinary generic
Java, the method region parameters are given as a comma-separated list
of identifiers enclosed in angle brackets (\kwd{<>}) immediately
before the method return type (for non-constructor methods) or class
name (for constructors).  A method parameter declaration has the same
form as a class parameter declaration
(\S~\ref{sec:classes:params:class}).
%
For example, the following code defines a method with type parameter
\kwd{T} and region parameter \kwd{R}:
%
\begin{dpjlisting}
interface MethodParams {
    public <type T, region R> void method();
}
\end{dpjlisting}
%
As for class parameters, the \kwd{type} keyword is always optional,
and the \kwd{region} keyword is optional except as to the first region
parameter.

\bfhead{Using method region parameters} A method region parameter is
available for use in RPLs (\S~\ref{sec:rpls:param}) within the method
definition.  In particular, a method region parameter standing alone
is a valid RPL.  For example, the following code declares a method
\kwd{method} with a region parameter \kwd{R} and uses \kwd{R} in the
formal (value) parameter and return type of the method:
%
\begin{dpjlisting}
interface MethodParamUses<region R> {
    <region R1>void method(MethodParamUses<R1> arg);
}
\end{dpjlisting}
%
As in generic Java, a new declaration of a parameter named \kwd{R}
hides any parameter named \kwd{R} that is already in scope.  For
example, the following code is equivalent to the code above:
%
\begin{dpjlisting}
interface MethodParamUses<region R> {
    <region R>void method(MethodParamUses<R> arg);
}
\end{dpjlisting}
%

\bfhead{Invoking methods defined with region parameters} Subject to
the restrictions stated below, RPL arguments to method parameters may
be provided explicitly by the programmer, or they may be inferred from
the types of the method arguments.  In either case, the arguments must
obey any disjointness constraints on the method's region parameters
(\S~\ref{sec:classes:params:disjoint}).

\ithead{Explicit RPL arguments (fully specified)} If the RPL arguments
to the method are fully specified (i.e., they contain no \kwd{*} or
\kwd{?}), then the programmer may supply the RPL arguments explicitly,
using an extension of the Java syntax for generic method arguments.
For this form, as for Java generic methods, there must be an explicit
qualifier and a dot preceding the method name; and the arguments must
appear in angle brackets after the dot, and before the method name.
For example:
%
\begin{description}
\item \kwd{this.<}\emph{args}\kwd{>}\emph{meth-name}\kwd{()}
\end{description}
%
Here \emph{args} are type and/or RPL arguments (defined below), and
\emph{meth-name} is the name of the method being invoked.  Notice that
to invoke a method defined in the enclosing class with explicit RPL
arguments, \kwd{this} (or the class name, for a \kwd{static} method)
must be used as a qualifier.

The new DPJ features are as follows:
%
\begin{itemize}
%
\item Both type and region parameters may appear.  The number of type
  and RPL arguments must exactly match the number of type and region
  parameters (so all type arguments come first, because that is true
  for the parameters).
%
\item The first region parameter \emph{must} be preceded by the
  keyword \kwd{region}.  Any of the other region parameters \emph{may}
  be preceded by the keyword \kwd{region}.
%
\item Any of the type parameters \emph{may} be preceded by the keyword
  \kwd{type}.
%
\end{itemize}
%

The following code example illustrates the use of explicit RPL
arguments to a method invocation:
%
\begin{dpjlisting}
abstract class ExplicitArgs<region R> {
    abstract <type T, region R>T callee(ExplicitArgs<R> param);
    ExplicitArgs<Root> caller() {
        ExplicitArgs<Root> arg = new ExplicitArgs<Root>();
        return this.<ExplicitArgs<Root>,Root>callee(arg);
    }
}
\end{dpjlisting}
%
The abstract method \kwd{callee} is generic in type \kwd{T} and
region \kwd{R}.  It takes an \kwd{ExplicitArgs<R>} object as a value
argument and returns type \kwd{T}.  In \kwd{caller}, we create
a new \kwd{ExplicitArgs<Root>} and pass it to \kwd{callee}.  We
pass \kwd{ExplicitArgs<Root>} as the type argument to \kwd{T} and
\kwd{Root} as the RPL argument to \kwd{R}, using explicit RPL
arguments.  Notice that the type of \kwd{arg} matches the type of
\kwd{param}, after substituting \kwd{Root} for \kwd{R} in \kwd{param}.
If the types did not match, there would be a compile error.  See
\S~\ref{sec:types:exp:invoke} for more information about how typing
works for method invocations in the presence of region parameters.

Note that the keyword \kwd{region} must be present to identify the
first RPL argument in the list, unlike the case of RPL arguments to
classes (\S~\ref{sec:types:class:instant}), where the \kwd{region}
keyword is optional.  The reason for this rule is Java's method
overloading: because multiple methods can be declared with the same
name but different parameters, information about which arguments are
types and which are regions is not always available from the method
name, as it is from the class name.

\ithead{Explicit RPL arguments (partially specified)} Explicit RPL
arguments to a method invocation may contain partially specified RPLs
(\S~\ref{sec:rpls:partial}).  However, if a partially specified RPL is
bound to a method region parameter \kwd{R}, then \kwd{R} may not
appear in any of the types of the method arguments.  For example, the
following code causes a compile error:
%
\begin{dpjlisting}
abstract class PartiallySpecifiedBad<region R> {
    region r1, r2;
    abstract <region R>void callee(PartiallySpecifiedBad<R> param1,
                                          PartiallySpecifiedBad<R> param2);
    void caller() {
        PartiallySpecifiedBad<r1> arg1 = new PartiallySpecifiedBad<r1>();
        PartiallySpecifiedBad<r2> arg2 = new PartiallySpecifiedBad<r2>();
        return this.<*>callee(arg1,arg2);
    }
}
\end{dpjlisting}
%
If this code were allowed, then both \kwd{r1} and \kwd{r2} would be
bound to \kwd{R} inside the scope of \kwd{callee}.  This would violate
a key invariant of the system, namely that in any scope a parameter
\kwd{R} always refers consistently to a single region.  Technically,
the compiler avoids this problem by \emph{capturing} any fully
specified RPLs provided as an explicit argument to a method RPL
parameter (see \S~\ref{sec:types:exp:capture}).  The capture operation
declares a fresh parameter which stands in for the runtime region
represented by the \kwd{*}.

In practice, this means that explicit partially-specified arguments to
method RPLs are rarely used.  Instead, an inferred argument would be
used (see immediately below).  However, an explicit
partially-specified argument may be used to specify an effect.  For
example, this code is legal:
%
\begin{dpjlisting}
abstract class PartiallySpecifiedOK<region R> {
    abstract <region R>void callee() writes R;
    void caller() writes * {
        // Effect is writes *
        return this.<*>callee(arg1,arg2);
    }
}
\end{dpjlisting}
%


\ithead{Inferred RPL arguments} As in generic Java, a method with type
and/or region parameters may be written without any explicit generic
arguments.  In this case the compiler will attempt to infer the type
and/or region arguments from the types of the value arguments supplied
to the method.  For example, the \kwd{dpjc} compiler accepts the
following code, because it is able to infer from the type
\kwd{InferredArguments<Root>} of \kwd{arg} that the argument to
\kwd{R} is \kwd{Root}:
%
\begin{dpjlisting}
abstract class InferredArguments<region R> {
    abstract <region R>void callee(InferredArguments<R> param);
    void caller() {
        InferredArguments<Root> arg = new InferredArguments<Root>();
        callee(arg);
    }
}
\end{dpjlisting}
%
This code is equivalent to the following:
%
\begin{dpjlisting}
abstract class InferredArguments<region R> {
    abstract <region R>void callee(InferredArguments<R> param);
    void caller() {
        InferredArguments<Root> arg = new InferredArguments<Root>();
        this.<Root>callee(arg);
    }
}
\end{dpjlisting}

While the compiler can infer region arguments to methods in many
cases, sometimes it cannot, either because the inference algorithm is
insufficiently powerful, or because the information is simply not
available from the types of the value parameters.  The compiler uses
\kwd{Root:*} as the argument to any region parameters in the method
that it cannot infer.  For example, the compiler would infer
\kwd{Root:*} as the argument to \kwd{R} in the following code:
%
\begin{dpjlisting}
abstract class InferredArguments<region R> {
    abstract <region R>void callee() writes R;
    void caller() {
        // There are no value arguments, so there is no way to infer
        // the argument to R!  This is equivalent to this.<Root:*>callee();
        callee();
    }
}
\end{dpjlisting}

Finally, because of the restriction on partially-specified RPLs in
explicit arguments mentioned above, it is possible to write certain
method invocations using inferred arguments that it is \emph{not}
possible to write using explicit arguments.  For example:
%
\begin{dpjlisting}
abstract class InferredArguments<region R> {
    abstract <region R>void callee(InferredArguments<R> param) writes R;
    void caller() {
        InferredArguments<*> arg = new InferredArguments<Root>();
        callee(arg);
    }
}
\end{dpjlisting}
%
Here the compiler infers that at the invocation of \kwd{callee} a
fresh region parameter should be bound to both the parameter \kwd{R}
in the method and the parameter \kwd{R} in the type.

Notice that because the invocation of \kwd{callee} binds \kwd{*} to
\kwd{R}, and \kwd{R} appears in the type of \kwd{param}, it would not
be legal to provide the RPL argument explicitly:
%
\begin{dpjlisting}
this.<*>callee(arg); // illegal!
\end{dpjlisting}
%
That is because this code provides no assurance that the region bound
to the method parameter is the same as the region bound to the type of
\kwd{arg}.  In DPJ, region inference is the only way to declare a
fresh RPL parameter and bind it to multiple region arguments (this is
sometimes called ``opening an existential'').  Generic Java has a
similar mechanism for handling wildcard types in method arguments.

\subsubsection{Disjointness Constraints%
\label{sec:classes:params:disjoint}}
\cutname{classes.params.disjoint.html}

It is sometimes necessary to require that two or more region
parameters (or a region parameter and some other region) be disjoint.
For example, the following code can have a data race unless regions
\kwd{R1} and \kwd{R2} are bound to disjoint regions:
%
\begin{dpjlisting}
class Unsafe<region R> {
    int x in R;
    <region R1, R2>void method(Unsafe<R1> o1, Unsafe<R2> o2) {
        cobegin {
            ++o1.x; // writes R1
            ++o2.x; // writes R2
        }
    }
}
\end{dpjlisting}
%
To support this reasoning, DPJ provides optional \emph{disjointness
  constraints} for region parameters.  If used, the constraints must
appear after the region parameters, and be separated from them by a
vertical bar:
%
\begin{description}
\item \kwd{<region R1, R2, ... |} \emph{constraints} \kwd{>}
\end{description}
%
\emph{constraints} is a comma-separated list of constraints, where a
constraint has the form \emph{rpl1} \kwd{\#} \emph{rpl2}, and
\emph{rpl1} and \emph{rpl2} are valid RPLs (\S~\ref{sec:rpls}).  The
constraint states that \emph{rpl1} and \emph{rpl2} are disjoint
regions (\S~\ref{sec:rpls:comparing:disjoint}).  For class region
parameters, the requirement is enforced when the class is instantiated
to a type by substituting RPLs for parameters
(\S~\ref{sec:types:class:instant}).  For method region parameters, the
requirement is enforced when RPL arguments are provided to an
invocation of the method (\S~\ref{sec:classes:params:method}).

For example, the code above could be rewritten as follows to make it
safe:
%
\begin{dpjlisting}
class Safe<region R> {
    int x in R;
    <region R1, R2 | R1 # R2> void(Safe<R1> o1, Safe<R2> o2) {
        cobegin {
            ++o1.x; // writes R1
            ++o2.x; // writes R2
        }
    }
}
\end{dpjlisting}
%
There is no interference between the statements of the \kwd{cobegin},
because \kwd{R1} and \kwd{R2} are guaranteed to be disjoint regions.
Therefore, the effects of \kwd{++o1.x} and \kwd{++o2.x} are to
disjoint memory locations.

We can also use constraints to make parameters disjoint from region
names, not just other parameters.  For example, we can write the
following:
%
\begin{dpjlisting}
class Safe2<region R> {
    region r;
    int x in R;
    <region R | R # r>void(Safe2<R> o1, Safe2<r> o2) {
        cobegin {
            ++o1.x; // writes R
            ++o2.x; // writes r
        }
    }
}
\end{dpjlisting}
%
This technique is useful when methods need to operate disjointly in
parallel on global data and data passed in as an an argument.  Of
course, the region of the global data could also be passed in as an
RPL argument, but using the global region directly in the constraint
saves writing parameters and arguments.

It is an error to write a constraint that cannot be satisfied, because
the RPLs given cannot be disjoint.  For instance, this class
declaration generates a compile error:
%
\begin{dpjlisting}
class BadConstraint<region R | R # R> {}
\end{dpjlisting}

It is also an error to provide RPL arguments to a class or method that
do not satisfy the constraints.  For example:
%
\begin{dpjlisting}
abstract class BadArgs<region R1, R2 | R1 # R2> {
    // Error:  Instantiating BadArgs with R1=Root, R2=Root
    BadArgs<Root, Root> x;
    abstract <region R3, R4 | R3 # R4>void callee();
    // Error:  Invoking callee with R3=Root, R4=Root
    void caller() {
        this.<Root,Root>callee();
    }
}
\end{dpjlisting}
%
\kwd{BadArgs<Root, Root>} is an invalid type, because the RPLs
supplied as arguments to \kwd{R1} and \kwd{R2} are not disjoint, as
required by the definition of class \kwd{BadArgs}.  Similarly,
\kwd{this.<Root,Root>callee()} is an invalid invocation of
\kwd{callee}, because the RPLs supplied as arguments to \kwd{R3} and
\kwd{R4} are not disjoint, as required by the definition of method
\kwd{callee}.

