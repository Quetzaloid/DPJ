\section{Exception Behavior%
\label{sec:exceptions}}
\cutname{exceptions.html}

In DPJ, an exception thrown outside any parallel construct
(\kwd{cobegin} or \kwd{foreach}) behaves exactly as in sequential
Java.  An exception thrown inside a parallel construct and caught
inside that same parallel construct also behaves as in sequential
Java.  An exception thrown inside a parallel construct and not caught
inside that parallel construct has the following behavior:

\begin{enumerate}
\item If an exception $E$ is thrown in branch $B$ of a \kwd{cobegin}
  or \kwd{foreach}, then branch $B$ behaves as if it were executed in
  isolation, starting with the state that existed at the start of the
  \kwd{cobegin} or \kwd{foreach}.  For example, consider this code,
  where \kwd{loopBody(int)} is a method:

\begin{dpjlisting}
foreach (int i in 0, 10) {
    loopBody(i);
}
\end{dpjlisting}

If iteration $I$ throws exception $E$, then replacing the entire
\kwd{foreach} with \kwd{loopBody(I)} would also cause $E$ to be
thrown, at the same point in the execution of method \kwd{loopBody} as
in the parallel case.

\item If multiple branches of a \kwd{cobegin} or \kwd{foreach}, each
  run in isolation, would throw an exception, then one of those
  exceptions is guaranteed to be thrown by the entire \kwd{cobegin} or
  \kwd{foreach}.  Which one is thrown is scheduler dependent (i.e.,
  different ones may be thrown on different executions).

\item Methods annotated \kwd{commutative}
  (Section~\ref{sec:classes:methods:commutative}) must respect this
  exception behavior.  For instance, method invocations $I_1$ and
  $I_2$ are not considered to commute with each other if execution
  $I_1; I_2$ throws an exception, but $I_2; I_1$ does not.

\end{enumerate}
