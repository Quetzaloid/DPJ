\section{The DPJ Runtime%
\label{sec:runtime}}
\cutname{runtime.html}

This section gives an overview of the classes \kwd{ArraySlice} and
\kwd{Partition} in the package \kwd{DPJRuntime}.  These classes are
part of the runtime supplied with DPJ and are useful for manipulating
arrays.  For full documentation of these classes, and for coverage of
the other DPJ runtime classes, see the HTML documentation included in
the DPJ release at \kwd{DPJ/Implementation/Runtime/dpjdoc}.

To use the DPJ runtime, you have to do the following:
%
\begin{enumerate}
%
\item Put \kwd{import DPJRuntime.*} at the top of your code (or just
  import the class or classes you want to use), as for regular Java.
%
\item Compile the runtime classes located at
  \kwd{Implementation/Runtime/dpj} in the DPJ release together with
  any code that uses the runtime.  For example, if class file
  \kwd{Foo.java} depends on the DPJ runtime, then you need to do
  something like this:
%
\begin{description}
\item \kwd{dpjc Foo.java \$\{DPJ\_ROOT\}/Implementation/Runtime/dpj/*java}
\end{description}
%
\emph{It is not sufficient to put the runtime classes in your class
  path.}  If you do that, you will get a lot of errors.  The reason is
that DPJ bytecode doesn't yet properly support separate compilation of
DPJ annotations; the DPJ compiler needs the DPJ source for all
annotated code to process the annotations.
%
\item When running your code, put the runtime classes located at
  \kwd{Implementation/Runtime/classes} in your class path, as for
  regular Java.
%
\end{enumerate}

\subsection{Array Classes%
\label{sec:runtime:array}}
\cutname{runtime.array.html}

The DPJ runtime provides a generic array class \kwd{Array} that is
defined like this:
%
\begin{dpjlisting}
package DPJRuntime;

public arrayclass Array<type T, region R> {
  T in R;
}
\end{dpjlisting}
%
It provides a similar array class for each primitive type, specialized
to that type.

The DPJ runtime also provides an index-parameterized array
\kwd{IPArray} that is identical to the \kwd{Array} class shown above,
except that the cell region specifier is \kwd{R:[index]} instead of
\kwd{R}.  Specialized versions of \kwd{IPArray} are also provided for
each primitive type.

These array classes handle many of the common cases involving arrays
and regions.  For other cases, it is easy to define custom array
classes.  In particular, because Java generics and arrays don't work
well together, it's often useful to define a custom array class
specialized to a particular class type, rather than using a generic
array class.

\subsection{ArraySlice%
\label{sec:runtime:array-slice}}
\cutname{runtime.array-slice.html}

In parallel algorithms that operate on arrays, especially
divide-and-conquer algorithms, it is often necessary to split an array
into parts and operate separately on the separate parts.  DPJ provides
array slices as an operation of a library class, called
\kwd{ArraySlice}.  There are two kinds of \kwd{ArraySlice} classes:
the generic \kwd{ArraySlice} class, and a set of \kwd{ArraySlice}
classes specialized to primitive types.

\bfhead{Generic ArraySlice} The generic \kwd{ArraySlice} class
represents an array of objects.  It has one type parameter and one RPL
parameter:
%
\begin{dpjlisting}
class ArraySlice<type T, region R>
\end{dpjlisting}
%
The type parameter is the element type of the array, and the RPL
parameter is the RPL of the array storage.

\ithead{Creating an ArraySlice} There are three ways to create a
\kwd{ArraySlice} object.  The first is to call a constructor that
takes just a length argument.  This operation creates a new
\kwd{ArraySlice} with the specified length, type, and RPL.  For
example, the following code creates an \kwd{ArraySlice} storing 10
\kwd{Integer} objects, and the storage is in RPL \kwd{r}:
%
\begin{dpjlisting}
region r;
ArraySlice<Integer,r> A = new ArraySlice<Integer,r>(10);
\end{dpjlisting}
%

Creating a fresh DPJ array is useful, but sometimes you have an array
and you want to make an \kwd{ArraySlice} out of it.  So the second way
to create an \kwd{ArraySlice} is to call a constructor that takes an
array class as an argument.  For example, the following code creates a
Java array of 10 \kwd{Integer} objects, and wraps it in a
\kwd{ArraySlice}:
%
\begin{dpjlisting}
region r;
ArrayInt<r> a = new ArrayInt<r>(10);
ArraySlice<Integer,r> A = new ArraySlice<Integer,r>(a);
\end{dpjlisting}
%
Note that the type and RPL of the array being passed in (here,
\kwd{Integer} and \kwd{r}) must match the type and RPL argument of the
\kwd{ArraySlice} type; if not, a compile error occurs.

The third way to create an \kwd{ArraySlice} is to make a
\emph{subslice} (i.e., a smaller slice) of an existing
\kwd{ArraySlice}.  This is explained below.

\ithead{Accessing elements} \kwd{ArraySlice} has \kwd{put} and
\kwd{get} operations similar to the ones in \kwd{java.util.ArrayList}.
For example, if \kwd{A1} and \kwd{A2} are \kwd{ArraySlice}s, then the
following code fragment gets element 0 out of \kwd{A1} and stores it
into element 0 of \kwd{A2}:
%
\begin{dpjlisting}
A2.put(0, A1.get(0));
\end{dpjlisting}
%
If the \kwd{ArraySlice} was created by wrapping a Java array, then the
\kwd{put} operation modifies the wrapped array.  The effect of a
\kwd{get} operation is a read on the region of the \kwd{ArraySlice}, and
the effect of a \kwd{put} operation is a write to the region of the
\kwd{ArraySlice}.

Accesses are bounds-checked.  Any attempt to access a position less
than 0 or greater than the array length minus one throws an
\kwd{ArrayIndexOutOfBoundsException}.

\ithead{Subslices} The real usefulness of \kwd{ArraySlice} is its
support for \emph{subslices}.  To create a subslice of an
\kwd{ArraySlice}, you call the \kwd{subslice} instance method with a
start and length argument.  For example, the following code creates an
\kwd{ArraySlice} and then extracts the subslice of length 2 starting
at position 5 (here we use the default RPL of \kwd{Root}):
%
\begin{dpjlisting}
ArraySlice<Integer> A = new ArraySlice<Integer>(10);
ArraySlice<Integer> B = A.subslice(5,2);
\end{dpjlisting}

There are two nice things about subslices.  First, creating a subslice
takes minimal time and space overhead.  Nothing is copied, and no new
storage allocated to hold any array elements.  A small object is
created that stores the start position, length, and a reference to the
underlying array.

Second, as far as its API is concerned, a subslice is
indistinguishable from a freshly created \kwd{ArraySlice}.  For example,
the subslice created above is zero-indexed, it has length 2, and
attempts to access indices other than 0 and 1 throw an exception.
However, the subslice also provides a view into the original array.
For example, the following code stores 1 into position 0 of \kwd{B},
which is the same as position 5 of \kwd{A}:
%
\begin{dpjlisting}
B.put(0,1);
\end{dpjlisting}
%
You can get the start position out of an \kwd{ArraySlice} by reading the
field \kwd{start}, and you can get the length by reading the field
\kwd{length}.

These features allow methods that operate on array subranges to be
zero-indexed, without worrying about the index parameters of the
subranges.  For example, here is some simple recursive code that
increments every position of an \kwd{ArraySlice} by 1:
%
\begin{dpjlisting}
<type T, region R>void increment(ArraySlice<T,R> A) {
  if (A.length == 0) return;
  if (A.length == 1) {
    A.put(0, A.get(0)+1);
  }
  int mid = A.length / 2;
  increment(A.subslice(0, mid));
  increment(A.subslice(mid, length-mid));
}
\end{dpjlisting}
%
Without the subslice feature, this code would have to be written by
passing the index ranges as parameters to the \kwd{increment} method,
which is ugly.

\bfhead{Specialized ArraySlice} The DPJ runtime has a version of
\kwd{ArraySlice} specialized to each primitive type.

\subsection{Partition%
\label{sec:runtime:partition}}
\cutname{runtime.partition.html}

For parallel divide-and-conquer algorithms on arrays, it is often
important to create disjoint collections of subslices.  For example, a
parallel sorting algorithm might repeatedly divide an array into
disjoint halves (or quarters, etc.) and operate recursively in
parallel on the pieces.  To support effect checking for this kind of
algorithm, the DPJ runtime includes a class \kwd{Partition} for
representing disjoint collections of slice of the same array.  Each of
the slices in the collection is called a \emph{segment} of the
partition.  As with \kwd{ArraySlice}, there is a generic version, and
there are specialized versions.

\bfhead{Generic Partition}  The generic \kwd{Partition}
class represents an array of slices of an \kwd{ArraySlice}.  It has
one type parameter and one RPL parameter:
%
\begin{dpjlisting}
class Partition<type T, region R>
\end{dpjlisting}
%
The type parameter is the element type of the \kwd{ArraySlice} being
partitioned, and the RPL parameter is the RPL of the array storage.

\ithead{Creating a Partition} There are several ways to
create a \kwd{Partition}; for a full list, see the HTML
documentation.  Here are two useful ways.  First, \kwd{Partition}
has a constructor that takes an \kwd{ArraySlice} to partition, an index
at which to partition, and a \kwd{boolean} value that says whether to
exclude or include the element at the index position.  For example, if
\kwd{A} is an \kwd{ArraySlice<Integer>} of length 10, then
%
\begin{dpjlisting}
new ArraySlice<Integer>(A, 5, true)
\end{dpjlisting}
%
partitions \kwd{A} into the segments $[0,4]$ and $[6,9]$ (excluding
position 5), while
%
\begin{dpjlisting}
new ArraySlice<Integer>(A, 5, false)
\end{dpjlisting}
%
partitions \kwd{A} into the segments $[0,4]$ and $[5,9]$ (including
position 5).  This constructor is useful for parallel
divide-and-conquer algorithms with a fanout of 2.

Second, \kwd{Partition} has a static factory method
\kwd{stridedPartition} that takes an \kwd{ArraySlice} to partition and a
stride at which to partition.  For example, if \kwd{A} is a
\kwd{ArraySlice<Integer>} of length 10, then
%
\begin{dpjlisting}
new Partition<Integer>(A, 2)
\end{dpjlisting}
%
creates a \kwd{Partition<Integer>} with five segments, each of
length 2.  This feature is useful for parallel divide-and-conquer
algorithms with a fanout of greater than two, as well as flat
partitions (such as tiling an array).

\ithead{Accessing segments} The field \kwd{length} stores the number
of segments in the partition.  It is \kwd{final}, so reading it has no
effect (\S~\ref{sec:effects:stmt-exp}).  The method \kwd{get} takes an
integer index \kwd{idx} and returns the segment corresponding to that
index (and throws an exception if the index is out of range).  The
type of the segment is \kwd{ArraySlice<T,this:[idx]:*>}.

The index-parameterized type returned by \kwd{get} allows the
different segments to be operated on in parallel without interference,
similarly to an index-parameterized array
(\S~\ref{sec:array-classes:ipa}).  For example, the following code uses
\kwd{Partition} to parallelize the simple recursive \kwd{increment}
shown in \S~\ref{sec:runtime:array}:
%
\begin{dpjlisting}
<type T, region R>void parallelIncrement(ArraySlice<T,R> A) 
  writes R:* 
{
  if (A.length == 0) return;
  if (A.length == 1) {
    // Effect is 'writes R'
    A.put(0, A.get(0)+1);
  }
  int mid = A.length / 2;
  final Partition<T,R> segs = new Partition<T,R>(A,mid)
  cobegin {
    // Effect is 'writes segs:[0]:*'
    parallelIncrement(segs.get(0));
    // Effect is 'writes segs:[1]:*'
    parallelIncrement(segs.get(1));
  }
}
\end{dpjlisting}
%
For more examples of how to use \kwd{Partition}, see \tutorial.

The variable \kwd{this} in the type gets substituted by the recevier
expression at the call site (\S~\ref{sec:types:exp:invoke}), so it is
usually most useful to call \kwd{get} through a \kwd{final} local
variable.  Including the variable in the RPL ensures that the compiler
doesn't erroneously infer disjointness for two different partitions of
the same array.  For example, in the following code,
\kwd{segs1.get(1)} and \kwd{segs2.get(0)} are not disjoint (they
overlap at $[2,7]$):
%
\begin{dpjlisting}
ArraySlice<Integer> A = new ArraySlice<Integer>(10);
// Create segments [0,1] and [2,9]
Partition<Integer> segs1 = new Partition(A, 2);
// Create segments [0,7] and [8,9]
Partition<Integer> segs2 = new Partition(A, 8);
\end{dpjlisting}

\bfhead{Specialized Partition} As with \kwd{ArraySlice}, there are
versions of \kwd{Partition} specialized to each primitive type.


