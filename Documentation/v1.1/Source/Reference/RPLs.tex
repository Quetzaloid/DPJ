\section{Region Path Lists (RPLs)%
\label{sec:rpls}}
\cutname{rpls.html}

This section describes \emph{region path lists}, or RPLs.  RPLs are
the fundamental tool for describing sets of memory locations in DPJ.
RPLs can be \emph{fully specified} or \emph{partially specified}.  A
fully-specified RPL names a \emph{region}, which represents a set of
memory locations on the heap. The correspondence between memory
locations and regions is given by the way that the programmer has
assigned RPLs to class fields (\S~\ref{sec:classes:field-region-spec})
and array class cells (\S~\ref{sec:array-classes}) and bound RPLs to
region parameters in class types (\S~\ref{sec:types:class:instant})
and method invocations (\S~\ref{sec:classes:params:method}). A
partially-specified RPL names a \emph{set of regions} and is useful
for expressing hierarchical effects (e.g., ``all regions under this
one'').

RPLs are naturally nested.  In fact the motivation for name ``region
path list'' is that an RPL is a sequence of names that describes a
path from the root in a tree of regions.  Together with partially
specified RPLs, this nesting structure provides a great deal of power
in comparing sets of memory locations to determine whether they are
overlapping or disjoint.  These properties are the key to specifying
and checking effects.  In particular, checking \emph{subeffects}
(e.g., that a method's effect summary covers its effects,
\S~\ref{sec:classes:methods:summaries}) requires reasoning about
inclusion of RPLs, while checking \emph{noninterfering effects} (e.g.,
that the branches of a \kwd{cobegin} statement do not interfere,
\S~\ref{sec:parallel:cobegin}) requires reasoning about disjointness
of RPLs.

The rest of this section proceeds as follows.  \S~\ref{sec:rpls:basic}
describes the simplest form of RPL, which is just a name, such as a
region name declared at class scope
(\S~\ref{sec:classes:region-names}).  \S~\ref{sec:rpls:sequences}
describes how to construct RPLs using \emph{sequences of names}, which
put a nesting structure on RPLs.  \S~\ref{sec:rpls:param} describes
the use of region parameters (\S~\ref{sec:classes:params}) with RPLs.
These parameters allow RPLs to vary with the class type or method
invocation.  \S~\ref{sec:rpls:partial} explains how to write
\emph{partially specified RPLs}, which describe sets of fully
specified RPLs.  \S~\ref{sec:rpls:comparing} explains the rules for
comparing RPLs for properties such as inclusion and disjointness.

Finally, a word about the uses of RPLs.  This section is about the
\emph{definition} of RPLs, and the relations among them (inclusion,
disjointness, etc.).  The various ways in which RPLs can be
\emph{used} (i.e., how they actually appear in a DPJ program) are
covered in other sections.  To provide context for this section, we
list all the possible uses here, together with the sections discussing
those uses:
%
\begin{itemize}
%
\item RPLs can appear in field RPL specifiers, to assign class fields
  to regions (\S~\ref{sec:classes:field-region-spec}).
%
\item RPLs can appear in effect summaries
  (\S~\ref{sec:effects:summaries}).  These RPLs describe the effects
  of a code statement.
%
\item RPLs can appear as arguments to parametric class types
  (\S~\ref{sec:types:class:instant}).  These RPLs provide actual
  regions for the parameters in the instantiation of a class or array.
%
\item RPLs can appear in method invocations, as arguments to method
  region parameters (\S~\ref{sec:classes:params:method}).  These RPLs
  provide actual regions for the parameters in an invocation of a
  method.
%
\end{itemize}
%
Throughout this section, we provide example uses to illustrate the
definitions in context.

\subsection{Basic Region Names%
\label{sec:rpls:basic}}
\cutname{rpls.basic.html}

This section describes the simplest form of RPL, which is a single
name.  There are five kinds of region names: 
%
\begin{enumerate}
%
\item
\kwd{Root}, a special name that is always in scope.
%
\item A region name declared at class scope
  (\S~\ref{sec:classes:region-names}).
%
\item An array index region \kwd{[}\emph{exp}\kwd{]}, where \emph{exp}
  is an integer expression.
%
\item A region name declared in a method-local scope
  (\S~\ref{sec:classes:methods:local-regions}).
%
\item A \kwd{final} local variable or \kwd{this}.  
%
\end{enumerate}
%
The first three kinds of names are available everywhere (except that
class region names may have access qualifiers, see
\S~\ref{sec:classes:region-names}).  The last two kinds of names are
available only in the scope where the region names or variables
are active.

\subsubsection{The Name \kwd{Root}%
\label{sec:rpls:basic:root}}
\cutname{rpls.basic.root.html}

The region name \kwd{Root} is always in scope, and can be used
anywhere as an RPL.  For example, the following code uses \kwd{Root}
in a field region specifier (\S~\ref{sec:classes:field-region-spec})
and to instantiate a class type (\S~\ref{sec:types:class:instant}):
%
\begin{dpjlisting}
class RootExample<region R> {
  RootExample<Root> x in Root;
}
\end{dpjlisting}
%

The name \kwd{Root} has a special meaning: it is the root of the
region tree.  Every RPL is nested under \kwd{Root}.  For more
information about the nesting relation on RPLs, see
\S~\ref{sec:rpls:comparing:nest}.

\subsubsection{Class Region Names%
\label{sec:rpls:basic:class}}
\cutname{rpls.basic.class.html}

A region name declared as a class member
(\S~\ref{sec:classes:region-names}) is available for use as an RPL in
any scope where it is visible.  For example, the following code
declares a region name \kwd{ClassRegion} at class scope, then uses it
in a field region specifier (\S~\ref{sec:classes:field-region-spec})
and to instantiate a class type (\S~\ref{sec:types:class:instant}):
%
\begin{dpjlisting}
class ClassRegionExample<region R> {
  region ClassRegion;
  int ClassRegionExample<ClassRegion> x in ClassRegion;
}
\end{dpjlisting}
%
Because class region names can be package- and class-qualified, like
Java static fields, the following code is also legal:
%
\begin{dpjlisting}
class Class1 {
  region ClassRegion;
}
class Class2 {
  int x in Class1.ClassRegion;
}
\end{dpjlisting}

When a class region name $R$ is used as an RPL, it is shorthand for
the name sequence \kwd{Root}:$R$, as described in
\S~\ref{sec:rpls:sequences}.  For example, the following definition of
\kwd{Class2} is equivalent to the one given above:
%
\begin{dpjlisting}
class Class2 {
  int x in Root:Class1.ClassRegion;
}
\end{dpjlisting}


\subsubsection{Array Index RPLs%
\label{sec:rpls:basic:array}}
\cutname{rpls.basic.array.html}

The name \kwd{[}$e$\kwd{]}, where $e$ is an integer expression,
functions as an RPL, called an \emph{array index RPL}.  The array
index RPL represents a region indexed by the number that $e$
evaluates to at runtime (so there is a different region for each
natural number).  It is useful in conjunction with
\emph{index-parameterized arrays} (\S~\ref{sec:array-classes:ipa}) and
with DPJ's built-in features for array partitioning
(\S~\ref{sec:runtime:partition}).

As with class region names used as RPLs
(\S~\ref{sec:rpls:basic:class}), an array index RPL \kwd{[}$e$\kwd{]}
is short for \kwd{Root:[}$e$\kwd{]}.

\subsubsection{Local Region Names%
\label{sec:rpls:basic:local}}
\cutname{rpls.basic.local.html}


A region name declared inside a method body
(\S~\ref{sec:classes:methods:local-regions}) is available for use as
an RPL in the scope where it is visible.  For example, the following
code declares a region name \kwd{LocalRegion} in a local scope, and
uses it to instantiate a type:
%
\begin{dpjlisting}
class LocalRegionExample<region R> {
  void method() {
    region LocalRegion;
    LocalRegionExample<LocalRegion> lre = 
      new LocalRegionExample<LocalRegion>();
  }
}
\end{dpjlisting}
%

As with class region names and array index regions, an RPL consisting
of a single local region name implicitly starts with \kwd{Root}.  For
example, the class definition above is equivalent to the following:
%
\begin{dpjlisting}
class LocalRegionExample<region R> {
  void method() {
    region LocalRegion;
    LocalRegionExample<Root:LocalRegion> lre = 
      new LocalRegionExample<Root:LocalRegion>();
  }
}
\end{dpjlisting}
%


\subsubsection{\kwd{final} Local Variables and \kwd{this} as Region Names%
\label{sec:rpls:basic:var}}
\cutname{rpls.basic.var.html}

The following variables may be used as an RPL:
%
\begin{itemize}
%
\item A \kwd{final} local variable or method parameter with a class
  type.
%
\item The variable \kwd{this}, in a non-static context.
%
\end{itemize}
%
This usage is called a \emph{variable RPL}.  For example, the
following code uses \kwd{this} as an RPL in a field region specifier
(\S~\ref{sec:classes:field-region-spec}), as an argument to a class
type (\S~\ref{sec:types:class:instant}), and in an effect summary
(\S~\ref{sec:effects:summaries}):
%
\begin{dpjlisting}
class VarRegionExample<region R> {
  VarRegionExample<this> x in this;
  void setX(VarRegionExample<this> x) 
    writes this 
  {
    this.x = x;
  }
}
\end{dpjlisting}
%

Variable RPLs allow \emph{runtime objects to function as regions}, as
in type systems based on \emph{ownership}.  The variable stands in for
the object stored to the variable at runtime.  Because the variable
must be \kwd{final} or \kwd{this} (which cannot be assigned to), the
variable always refers to the same object.  This feature is
particularly useful in conjunction with the \kwd{Partition} class
(\S~\ref{sec:runtime:partition}), because it provides a way to
instantiate a partition with the region corresponding to the array
being partitioned.

As in ownership systems, a variable RPL is nested under the first RPL
argument of its type, which is called the \emph{owner RPL} of the type
(\S~\ref{sec:types:class:owner-rpl}).  For example, in the code above,
the variable \kwd{this} has type \kwd{VarRegionExample<R>}, so it is
nested under the RPL \kwd{R}.  For more information about the nesting
relationship on RPLs, see \S~\ref{sec:rpls:comparing:nest}.


\subsection{Sequences of Basic Names%
\label{sec:rpls:sequences}}
\cutname{rpls.sequences.html}

RPLs are built up from colon-separated sequences of names; this gives
them a natural nesting structure.  For example, if \kwd{A} and \kwd{B}
are names, then \kwd{A}, \kwd{B}, \kwd{A:B}, \kwd{B:A}, \kwd{A:A:B},
and so forth are all RPLs.  The nesting structure is given by the
syntax: for example, \kwd{A:B} and \kwd{A:A:B} are both nested under
\kwd{A}.  Note that nesting does not imply inclusion; \kwd{A} and
\kwd{A:B} are distinct regions.  In particular the effect \kwd{writes
  A} does \emph{not} cover (or imply) the effect \kwd{writes A:B}.
However, the nesting structure creates a hierarchy of regions that is
useful in conjunction with partially specified RPLs
(\S~\ref{sec:rpls:partial}), because then we can use it to say things
like ``all RPLs under this one'' or ``all RPLs under this one that end
in \kwd{B}.''  Separating nesting from inclusion like this makes RPLs
more complicated, but it also allows greater precision.

An RPL is valid if it is conforms to the following rules:
%
\begin{itemize}
%
\item It is composed of basic names (\S~\ref{sec:rpls:basic}),
  separated by colons.
%
\item If \kwd{Root} (\S~\ref{sec:rpls:basic:root}) or a variable name
  (\S~\ref{sec:rpls:basic:var}) appears in the sequence, it must
  appear first (so, in particular, both cannot appear).
%
\end{itemize}
%
The names in the sequence are called the \emph{elements} of the RPL.
If an RPL does not start with \kwd{Root} or a variable element, then
it is treated as implicitly starting with \kwd{Root}.  For example,
\kwd{A} is the same as \kwd{Root:A}, and \kwd{[0]} is the same as
\kwd{Root:[0]}.

The following are examples of valid RPLs, assuming that \kwd{A} and
\kwd{B} are declared region names in scope:
%
\begin{dpjlisting}
A        // Declared name
Root:A   // Same as A
A:B      // Sequence of names
Root:A:B // Same as A:B
[0]      // Array index region
Root:[0] // Same as [0]
this     // Variable region
this:A   // Variable region with A appended
\end{dpjlisting}

The following are examples of RPLs that are \emph{not} valid, because
they do not conform to the rules above:
%
\begin{dpjlisting}
A:Root    // Root must appear first
this:Root // Root must appear first
A:this    // this must appear first
Root:this // this must appear first
\end{dpjlisting}

\subsection{Parameterized RPLs%
\label{sec:rpls:param}}
\cutname{rpls.param.html}

Class and method region parameters (\S~\ref{sec:classes:params}) work
with RPLs in a natural way.  First, a region parameter standing alone
is always a valid RPL, as shown in the examples in
\S\S~\ref{sec:classes:params:class}
and~\ref{sec:classes:params:method}.  Second, an RPL may be
constructed from a region parameter followed by a colon-separated
sequence of elements, similar to a sequence of basic names
(\S~\ref{sec:rpls:sequences}).  If a parameter appears in an RPL, then
it must appear first, and no \kwd{Root} or \kwd{final} local variable
may appear in the RPL (because those, when they appear, must also be
first).

The following are examples of valid parameterized RPLs, assuming that
\kwd{R} is a region parameter in scope and \kwd{A} and \kwd{B} are
declared region names in scope:
%
\begin{dpjlisting}
R      // Region parameter alone
R:A    // Region parameter with name appended
R:A:B  // Region parameter with names appended
R:[0]  // Region parameter with index appended
\end{dpjlisting}

The following are examples of RPLs that are \emph{not} valid, because
they do not conform to the rules above:
%
\begin{dpjlisting}
A:R      // R must appear first
Root:R   // R must appear first
R:Root   // Root must appear first
R:this   // this must appear first
\end{dpjlisting}

The meaning of a parameterized RPL is that at runtime, the parameter
will be substituted away, generating a region that has \kwd{Root} or
an object as its first element.  For example, consider the following
class definition:
%
\begin{dpjlisting}
class ParameterExample<region R> {
  region r;
  int x in R:r
}
\end{dpjlisting}
%
The class \kwd{ParameterExample} has one region parameter \kwd{R}, and
one field \kwd{x} placed in RPL \kwd{R} using a field RPL specifier
(\S~\ref{sec:classes:region-names}).  Suppose we now instantiate that
class to a type (\S~\ref{sec:types:class:instant}), by binding
\kwd{Root} to \kwd{R}:
%
\begin{dpjlisting}
ParameterExample<Root> pa = new ParameterExample<Root>();
\end{dpjlisting}
%
This type says that the object bound to \kwd{pa} at runtime has its
field \kwd{x} in RPL \kwd{Root:r} (substituting \kwd{Root} for \kwd{R}
in \kwd{R:r}).  For more information about how this subsitution works,
see \S~\ref{sec:types:exp}.

\subsection{Partially Specified RPLs%
\label{sec:rpls:partial}}
\cutname{rpls.partial.html}

In order for the nested structure of RPLs to be useful, the programmer
must be able to name a \emph{set} of RPLs such as ``all RPLs nested
under this one.''  To support this kind of naming, DPJ has a special
element \kwd{*} that stands in for any sequence of RPL elements.
There is also a wildcard array index element \kwd{[?]}, where the
\kwd{?} stands in for any natural number.  This is useful for naming
sets of array regions.

\subsubsection{The \kwd{*} RPL Element%
\label{sec:rpls:partial:star}}
\cutname{rpls.partial.star.html}

An RPL may be written with \kwd{*} as one if its elements.  The RPL
must otherwise follow the rules in \S\S~\ref{sec:rpls:sequences}
and~\ref{sec:rpls:param}: for example, if a parameter appears in the
RPL, it must be first.  It is permissible for an RPL to start with
\kwd{*}; in this case, there is an implicit \kwd{Root} before the
\kwd{*}.  In particular, the RPL \kwd{*} is equivalent to \kwd{Root:*}
and refers to all regions.

The meaning of an RPL containing \kwd{*} is that it stands in for all
legal RPLs that could be constructed by substituting sequences of zero
or more elements for the \kwd{*}.  For example, \kwd{A:*:B} stands in
for \kwd{A:B}, \kwd{A:A:B}, \kwd{A:B:B}, \kwd{A:A:A:B}, etc.  However,
\kwd{A:*:B} does \emph{not} stand in for \kwd{A:Root:B}, because that
is not a valid RPL (\kwd{Root} must appear first).  See
\S\S~\ref{sec:rpls:comparing:nest}
and~\ref{sec:rpls:comparing:include} for more information about the
nesting and inclusion rules for RPLs using \kwd{*}.

Here are some examples of valid RPLs constructed with the \kwd{*}
element, assuming that \kwd{A} and \kwd{B} are declared region names
in scope:
%
\begin{dpjlisting}
*          // All RPLs
Root:*     // Same as *
*:A        // All RPLs under Root ending in A
A:*        // All RPLs under Root:A
Root:A:*   // Same as A:*
this:*     // All RPLs under this
this:*:A   // All RPLs under this ending in A
\end{dpjlisting}
%
Although more complex RPLs are possible, typically in user code at
most one \kwd{*} appears in an RPL, as shown in the examples above.

Here are some examples of RPLs that are \emph{not} valid, because they
do not conform to the rules stated in \S\S~\ref{sec:rpls:sequences}
and~\ref{sec:rpls:param} (\kwd{R} is a region parameter):
%
\begin{dpjlisting}
*:R        // R must appear first
*:this     // this must appear first
A:*:Root   // Root must appear first
\end{dpjlisting}


\subsubsection{The \kwd{[?]} RPL Element%
\label{sec:rpls:partial:qmark}}
\cutname{rpls.partial.qmark.html}

The \emph{wildcard array index element} \kwd{[?]} may be used in an
RPL anywhere that an array index element \kwd{[$e$]} may appear
(\S~\ref{sec:rpls:basic:array}).  It stands in for any array index
element.  For example:
%
\begin{dpjlisting}
[?]        // Matches [e] for any [e]
Root:[?]   // Same as [?]
[?]:A      // Matches [e]:A for any e
this:[?]   // Matches this:[e] for any e
\end{dpjlisting}

\subsection{Local RPLs%
\label{sec:rpls:local}}
\cutname{rpls.local.html}

A \emph{local RPL} is an RPL that contains a local region name
(\S~\ref{sec:rpls:basic:local}) as an element.  It may be
parameterized and/or partially specified.  For example:
%
\begin{dpjlisting}
class LocalRPLs<region R> {
  region ClassRegion;
  void method() {
    region LocalRegion;
    // LocalRegion is a local RPL; it is equivalent to Root:LocalRegion
    LocalRPLs<LocalRegion> x = null;
    // R:LocalRegion:* is a local RPL
    LocalRpls<ClassRegion:LocalRegion:*> y = null;
    // ClassRegion is not a local RPL
    LocalRpls<ClassRegion> z = null;
  }
}
\end{dpjlisting}
%
Local RPLs are important because they define \emph{local effects}
(\S~\ref{sec:effects:local}).  A local effect is visible only in a
method and its callees, so doesn't have to be reported to the caller.
For example, in the code above, any effects on \kwd{LocalRegion} or
\kwd{R:LocalRegion} could be omitted from the effect summary of
\kwd{method}.

\subsection{Comparing RPLs%
\label{sec:rpls:comparing}}
\cutname{rpls.comparing.html}

To reason about effects, both the programmer and the compiler have to
make judgments like ``this RPL is included in that one'' or ``this RPL
is disjoint from that one'' (where ``disjoint'' means that the
corresponding sets of memory locations are nonintersecting).  The
programmer has to do this in order to reason about which code sections
may be run in parallel safely, i.e., without interference.  And the
compiler has to do it to check that the programmer got it right.

To support this kind of reasoning, there are four relations on RPLs:
%
\begin{enumerate}
%
\item \emph{Equivalence}, meaning that two RPLs refer to the same set
  of regions.
%
\item \emph{Nesting}, meaning that one RPL is a descendant of another
  in the region tree.  For example, \kwd{A:B} is nested under \kwd{A}.
%
\item \emph{Inclusion}, meaning that the set of memory locations
  represented by one RPL includes the set of locations represented by
  another.  For example, \kwd{A:B} is included in \kwd{A:*}.  Note
  that \emph{\kwd{A:B} is not included in \kwd{A}}.  The \kwd{*} is
  necessary to get the inclusion.  This is to increase the precision
  of effect specifications with RPLs.
%
\item \emph{Disjointness}, meaning that the set of memory locations
  represented by one RPL has no common elements with the set of memory
  locations represented by another RPL.  For example, if \kwd{A} and
  \kwd{B} are distinct names, then they represent disjoint regions.
  Or \kwd{R1} and \kwd{R2} are disjoint if they are parameters
  declared to be disjoint (\S~\ref{sec:classes:params:disjoint}).
%
\end{enumerate}

\subsubsection{Equivalence%
\label{sec:rpls:comparing:equiv}}
\cutname{rpls.comparing.equiv}

\bfhead{Equivalent RPLs} The meaning of the equivalence relation on
RPLs is that the sets of regions represented by the RPLs are the same.
Two RPLs are equivalent if they have the same number of elements, and
the corresponding pairs of elements are equivalent, as defined below.
This test is done after adding the implicit \kwd{Root} to the front of
RPLs that begin with a declared name element, array index element,
\kwd{*}, or \kwd{[?]}.

For example, the following pairs of RPLs are equivalent:
%
\begin{itemize}
%
\item \kwd{Root:A} and \kwd{A}
%
\item \kwd{this:[i]} and \kwd{this:[i]}, if \kwd{i} is a \kwd{final}
  local variable or method parameter
%
\item \kwd{R:[i+1]} and \kwd{R:[i+1]}, if \kwd{i} is a
\kwd{final} local variable or method parameter
%
\end{itemize}
%
The following pairs of RPLs are not equivalent:
%
\begin{itemize}
%
\item \kwd{Root:A} and \kwd{Root:A:B}
%
\item \kwd{this:[i]} and \kwd{this:[i]}, if \kwd{i} is not a
  \kwd{final} local variable or method parameter
%
\item \kwd{this:[i]} and \kwd{this:[j]}
%
\item \kwd{R:[i+1]} and \kwd{R:[i+1]}, if \kwd{i} is not a \kwd{final}
  local variable or method parameter
%
\item \kwd{R:[i+1]} and \kwd{R:[i+2]}
%
\end{itemize}

\bfhead{Equivalent RPL elements} The following rules govern
equivalence of RPL elements.

\ithead{Basic names other than array index elements} Two RPL elements
are equivalent if they refer to the same basic name
(\S~\ref{sec:rpls:basic}), except for array index elements.

\bfhead{The elements \kwd{*} and \kwd{[?]}} Two RPL elements are
equivalent if they are both \kwd{*} or \kwd{[?]}.

\ithead{Array index elements} Two array index elements \kwd{[$e_1$]}
and \kwd{[$e_2$]} equivalent if $e_1$ and $e_2$ always evaluate to the
same value (i.e., are ``always equal'').

Since the compiler can't always know when two expressions evaluate to
the same value, it applies the following conservative rules to
identify pairs of always-equal expressions:
%
\begin{itemize}
%
\item $e_1$ and $e_2$ are always-equal if they refer to the same
  compile-time integer constant.
%
\item $e_1$ and $e_2$ are always-equal if they are the same local
  variable or method parameter, and the variable or parameter is
  declared \kwd{final}.
%
\item $e_1$ and $e_2$ are always-equal if they represent the same
  binary operation, and the corresponding subexpressions are
  always-equal.  For example, \kwd{i+1} and \kwd{i+1} are
  always-equal, if \kwd{i} is a \kwd{final} local variable.
\end{itemize}
%
Otherwise, the compiler conservatively assumes that the expressions
are not always-equal.


\subsubsection{Nesting%
\label{sec:rpls:comparing:nest}}
\cutname{rpls.comparing.nest}

The meaning of the nesting relation on RPLs is that one RPL is a
descendant of another in the region tree.  Nesting is important
because, in conjunction with partially specified RPLs, it defines
inclusion of RPLs (\S~\ref{sec:rpls:comparing:include}).  Four rules
govern nesting of RPLs.

\bfhead{Nesting under \kwd{Root}}  Any RPL is nested under
\kwd{Root}.

\bfhead{Nesting by syntax} One RPL is nested under another if the
second is a prefix of the first.  For example, \kwd{A:B} is nested
under \kwd{A}, \kwd{R:A} is nested under \kwd{R}, and \kwd{this:[0]}
is nested under \kwd{this}.

\bfhead{Nesting by inclusion} If one RPL is included in another, then
the first RPL is nested under the second.  For example, \kwd{A:B} is
included in \kwd{A:*} (see \S~\ref{sec:rpls:comparing:include}), so
\kwd{A:B} is nested under \kwd{A:*}.

\bfhead{Nesting by ownership} An RPL that starts with a variable
element (\S~\ref{sec:rpls:basic:var}) is nested under the owner RPL of
the variable (\S~\ref{sec:types:class:owner-rpl}).  For example, in
the following code, the method parameter \kwd{param} has type
\kwd{NestByOwnership<R>}, so it is nested under \kwd{R} when used as
an RPL:
%
\begin{dpjlisting}
class NestByOwnership<region R> {
  <region R>method(final NestByOwnership<R> param) {
    // param is used as an RPL here; it is nested under R
    NestByOwnership<param> localVar = new NestByOwnership<param>();
  }
}
\end{dpjlisting}

\subsubsection{Inclusion%
\label{sec:rpls:comparing:include}}
\cutname{rpls.comparing.include}

The meaning of the inclusion relation on RPLs is that the set of
regions represented by one RPL is included (in the sense of set
inclusion) in the set of regions represented by another RPL.
Inclusion of RPLs is important in checking type comparisons
(\S\S~\ref{sec:types:class:compare},~\ref{sec:types:array:compare})
and subeffects (\S~\ref{sec:effects:subeffects}).  The following rules
govern inclusion of RPLs.

\bfhead{Inclusion by equivalence}  One RPL is included in another if
the RPLs are equivalent (\S~\ref{sec:rpls:comparing:equiv}).

\bfhead{Inclusion by trailing \kwd{*}} One RPL is included in another
if the second ends with \kwd{*}, and the first is nested under the
second after the trailing \kwd{*} is removed.  For example,
\kwd{A:B:C} is included in \kwd{A:*}, because \kwd{A:B:C} is nested
under \kwd{A} (i.e., \kwd{A:*} with the \kwd{*} removed).

\bfhead{Inclusion by trailing elements other than \kwd{*}} One RPL is
included in another if the last element of the first is included in
the last element of the second, and inclusion holds for the RPLs after
stripping the last elements of both.  One RPL element is included in
another if the elements are equivalent
(\S~\ref{sec:rpls:comparing:equiv}) or the first element is any array
index element or \kwd{[?]}, and the second is \kwd{[?]}.  For example:
%
\begin{itemize}
%
\item \kwd{A:B} is included in \kwd{A:*:B}, because \kwd{B} is
  included in \kwd{B}, and \kwd{A} is included in \kwd{A:*}
%
\item \kwd {A:[0]} is included in \kwd{A:*:[?]}, because \kwd{[0]} is
  included in \kwd{[?]}, and \kwd{A} is included in \kwd{A:*}.
%
\item \kwd{A:*:B} is not included in \kwd{A:B}, because \kwd{A:*} is
  not included in \kwd{A}.
%
\item \kwd {A:[?]} is not included in \kwd{A:[0]}, because \kwd{[?]}
  is not included in \kwd{[0]}.
%
\end{itemize}

\bfhead{Inclusion by constrant} Two RPLs may be constrained to be
included, one in the other.  In the current language specification,
the only way this can happen is when capturing a type
(\S~\ref{sec:types:exp}).  There is no way for the programmer to
directly write an inclusion constraint on RPLs.

\subsubsection{Disjointness%
\label{sec:rpls:comparing:disjoint}}
\cutname{rpls.comparing.disjoint.html}

The meaning of the disjointness relation on RPLs is that the sets of
regions represented by the RPLs have no common elements.  This
relation is important in checking noninterference
(\S~\ref{sec:effects:nonint}).

\bfhead{Disjoint RPLs}  The following rules govern disjointness of
RPLs.

\ithead{Distinctions from the left} Two RPLs are disjoint if they
start with a sequence of one or more equivalent elements, and then
their elements are disjoint, where disjoint elements are defined
below.  An application of this rule is called a ``distinction from the
left,'' because it uses a difference in the RPLs starting on the left
to infer disjointness.  This rule works because RPLs form a tree, so
two RPLs that start the same then diverge must be on different
branches of the tree.

The following are examples of distinctions from the left, assuming
\kwd{A} and \kwd{B} are class region names
(\S~\ref{sec:rpls:basic:class}):
%
\begin{itemize}
%
\item \kwd{A} and \kwd{B} are disjoint, because they are short for
  \kwd{Root:A} and \kwd{Root:B}, so they start with the same element
  \kwd{Root} and then diverge.  \kwd{A:*} and \kwd{B:*} are also
  disjoint, for the same reason.
%
\item If \kwd{R} is a region parameter, then \kwd{R:A} and \kwd{R:B}
  are disjoint, as are \kwd{R:A:*} and \kwd{R:B:*}.
%
\item \kwd{this:A} and \kwd{this:B} are disjoint, as are
  \kwd{this:A:*} and \kwd{this:B:*}.
%
\end{itemize}
%
Note that there must be at least one equivalent element for a
distinction to from the left.  For example, region parameters \kwd{R1}
and \kwd{R2} are not disjoint unless they are constrained to be
disjoint (see below).

\ithead{Distinctions from the right} Two RPLs are disjoint if they
end with disjoint elements, where disjoint elements are defined below.
An application of this rule is called a ``distinction from the
right,'' because it uses a difference in the RPLs starting on the
right to infer disjointness.  This rule works because two RPLs that
end in disjoint elements can never refer to the same regions, even
after substituting for parameters, \kwd{*}, and \kwd{[?]}.

The following are examples of distinctions from the right, assuming
\kwd{A} and \kwd{B} are class region names
(\S~\ref{sec:rpls:basic:class}):
%
\begin{itemize}
%
\item \kwd{Root:*:A} and \kwd{Root:*:B} are disjoint RPLs.
%
\item \kwd{R:*:A} and \kwd{R:*:B} are disjoint RPLs.
%
\item \kwd{this:*:A} and \kwd{this:*:B} are disjoint RPLs.
%
\end{itemize}

\ithead{Disjointness by constraint} Two RPLs are disjoint if they are
each included in another RPL (\ref{sec:rpls:comparing:include}) and
the including RPLs are constrained to be disjoint
(\ref{sec:classes:params:disjoint}).  For example, in the following
code, \kwd{R1:r} and \kwd{R2:r} are disjoint RPLs, because \kwd{R1:r}
is included in \kwd{R1:*}, \kwd{R2:r} is included in \kwd{R2:*}, and
\kwd{R1:*} and \kwd{R2:*} are constrained to be disjoint.
%
\begin{dpjlisting}
class DisjointConstraints<region R1, R2 | R1:* # R2:*> {
  region r;
  void method() {
    DisjointConstraints<R1:r> dc1 =
    new DisjointConstraints<R1:r>();
    DisjointConstraints<R2:r> dc2 =
    new DisjointConstraints<R2:r>();
  }
}
\end{dpjlisting}

Note that \kwd{R1:r} and \kwd{R2:r} would \emph{not} be disjoint if
the constraint were the weaker \kwd{R1 \# R2}; the \kwd{*} is
required.  This is because nesting does not imply inclusion.  For
example, setting $\kwd{R1}=\kwd{A}$ and $\kwd{R2}=\kwd{A:B}$ satisfies
\kwd{R1 \# R2}.  But then $\kwd{R1:B}=\kwd{A:B}$ is not disjoint from
\kwd{R2}, after substituting \kwd{A} for \kwd{R1} and \kwd{A:B} for
\kwd{R2}.  However, $\kwd{R1}=\kwd{A}$ and $\kwd{R2}=\kwd{A:B}$ does
not satisfy \kwd{R1:* \# R2:*}, because \kwd{A:B} is included in
\kwd{A:*}.


\bfhead{Disjoint RPL elements} Two RPL elements are disjoint if one
is a name and the other is an array index (including \kwd{[?]}); or
they are different names; or they are the array index elements
\kwd{[$e_1$]} and \kwd{[$e_2$]}, where $e_1$ and $e_2$ always evaluate
to different values at runtime (i.e., are ``never-equal
expressions'').

Since the compiler can't always know when two expressions evaluate to
the same value, it applies the following conservative rules to
identify pairs of never-equal expressions:
%
\begin{itemize}
%
\item $e_1$ and $e_2$ are never-equal if they refer to different
  compile-time constants.  For example, 0 and 1 are never-equal.
%
\item Index variables $i_1$ and $i_2$ are never-equal if they
  represent the values attained by the same index variable $i$ on
  different iterations of a \kwd{foreach} loop
  (\S~\ref{sec:parallel:foreach}).
%
\item $e_1$ and $e_2$ are never-equal if they represent the same
  binary operation, one of the corresponding subexpressions is
  always-equal (\S~\ref{sec:rpls:comparing:equiv}), and the other is
  never-equal.  For example, if \kwd{i} is a \kwd{final} local
  variable, then \kwd{i+1} and \kwd{i+2} are never-equal.
%
\end{itemize}
%
Otherwise, the compiler conservatively assumes that the expressions
are not never-equal.

