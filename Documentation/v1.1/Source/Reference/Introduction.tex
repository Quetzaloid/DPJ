\section{Introduction%
\label{sec:introduction}}
\cutname{introduction.html}

This document is a reference guide to the Deterministic Parallel Java
programming language, Version 1.1 (DPJ v1.1).  It explains in detail
the new language features that DPJ adds to Java, and how they work.
It is intended to be accessible to readers with no previous knowledge
of DPJ, though familiarity with Java is assumed.

Many cross-references appear throughout the document.  To refer to
sections, we use the section symbol \S: for example,
\S~\ref{sec:effects:basic} refers to the section numbered
\ref{sec:effects:basic}.

Readers of this document may also wish to consult the following:
%
\begin{itemize}
%
\item \tutorial\ provides a tutorial introduction to DPJ, and
  explains how to write some common parallel patterns in DPJ.  The DPJ
  programmer should probably read that document first.  It
  cross-references this document, so you can look up particular
  features that you want to learn more about.
%
\item \installmanual\ explains how to get started installing the
  \kwd{dpjc} compiler, using the compiler to compile DPJ programs, and
  running DPJ programs.
%
\end{itemize}

The rest of the document is organized as follows:
%
\begin{itemize}
%
\item \S~\ref{sec:classes}, \emph{Classes and Interfaces}, describes
  DPJ's extensions to Java classes and interfaces.  These include
  features for declaring region names, assigning region names to class
  fields, summarizing method effects, and writing classes and methods
  with region variables.
%
\item \S~\ref{sec:rpls}, \emph{Region Path Lists}, describes region
  path lists, or RPLs.  These are the general form of a region name in
  DPJ.  Regions partition the heap and help express heap effects.
  RPLs are hierarchically structured, and the structure allows many
  fine distinctions between sets of regions to be expressed.
%
\item \S~\ref{sec:types}, \emph{Types}, describes DPJ's extensions to
  Java's class and array types.  The main difference is that both
  class and array types can have region arguments.  There are also new
  rules for type comparisons, casts, etc. to support the extensions.
%
\item \S~\ref{sec:effects}, \emph{Effects}, describes DPJ's effect
  system, which is closely integrated with the regions in the type
  system.  Effects describe operations on the heap.  The programmer
  declares the effects of methods, and the compiler infers the rest of
  the effects.  The compiler checks to make sure that the effects of
  parallel tasks are mutually noninterfering.
%
\item \S~\ref{sec:parallel}, \emph{Parallel Control Flow}, describes
  DPJ's constructs for creating parallel tasks.
%
\item \S~\ref{sec:runtime}, \emph{The DPJ Runtime}, explains the classes in
  the DPJ runtime that are useful for manipulating arrays.
%
\item \S~\ref{sec:exceptions}, \emph{Exception Behavior}, explains
  what happens when an exception is thrown in DPJ.
%
\end{itemize}


