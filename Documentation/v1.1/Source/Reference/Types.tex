\section{Types%
\label{sec:types}}
\cutname{types.html}

DPJ extends the Java type system by adding RPLs (\S~\ref{sec:rpls}) to
class and array types.  The RPLs in the types support the effect
system: different class and array objects can be created with
different RPLs, and the compiler can use the RPLs to infer the effect
of an operation on the object (i.e., accessing a class field or array
cell, or invoking a method).

This section discusses DPJ's extensions to the Java type system:
\S~\ref{sec:types:class} discusses class and interface types,
\S~\ref{sec:types:array} discusses array types, and
\S~\ref{sec:types:exp} discusses the rules for determining the type of
an expression (e.g., what type is returned by calling a method).

\subsection{Class and Interface Types
\label{sec:types:class}}
\cutname{types.class.html}

This section discusses DPJ's class types (including array classes) and
interface types.  We will use the term ``class'' throughout this
section, with the understanding that the concepts apply identically to
array classes and interfaces, unless otherwise noted.
\S~\ref{sec:types:class:instant} discusses the instantiation of a
class to a type by supplying RPLs for its parameters.
\S~\ref{sec:types:class:compare} explains how the DPJ compiler
compares two class types for compatibility, for example in checking an
assignment statement.  \S~\ref{sec:types:class:cast} discusses casting
one type to another in DPJ.  \S~\ref{sec:types:class:owner-rpl}
explains \emph{owner RPLs}, which determine the nesting relation for
variable RPLs (\S~\ref{sec:rpls:basic:var}).

\subsubsection{Writing Class Types%
\label{sec:types:class:instant}}
\cutname{types.class.instant}

In DPJ, as in generic Java, a class with parameters defines a family
of types, one for each set of arguments to the parameters.  However,
in addition to type parameters, DPJ classes can have region parameters
(\S~\ref{sec:classes:params:class}).  When writing a DPJ class type,
parameter arguments must be supplied.  They can be supplied implicitly
or explicitly.

\bfhead{Implicit parameter arguments}  As in generic Java, a DPJ class
with parameters may always be used as a type by just naming the class,
without providing any explicit parameter arguments.  For example, the
following code is valid:
%
\begin{dpjlisting}
class ImplicitArgs<type T, region R> {
  // ImplicitArgs is a valid type here
  ImplicitArgs field = null;
}
\end{dpjlisting}
%

When the type arguments are omitted for a class with parameters, the
following occurs:
%
\begin{itemize}
%
\item The type parameters, if any, have no argument.  That is, the
  type functions as a \emph{raw type} in generic Java.
%
\item The RPL parameters, if any, become bound to \kwd{Root}.
%
\end{itemize}
%
In the example written above, \kwd{field} is of class type
\kwd{ImplicitArgs}, with \kwd{Root} bound to \kwd{R} and no
argument for \kwd{T}.

Implicit parameter arguments are useful in two ways.  First they help
with porting legacy Java code to DPJ.  For example, adding a region
parameter to a preexisting class will not break any code that uses the
class.  Second, because region arguments are used to compute the
effects on fields of a class (\S~\ref{sec:effects:stmt-exp}), they are
usually not important for code that is never invoked in a parallel
context (for example, in a sequential initialization phase).  For this
code, the programmer can avoid writing the arguments.

\bfhead{Explicit parameter arguments} As in Java, if explicit
arguments to class parameters are given, then they appear in angle
brackets after the class name.  The arguments must appear in the same
order in which the parameters appear in the class definition; so, in
particular, any type arguments must precede any RPL arguments.  Any of
the type arguments may be preceded by the keyword \kwd{type}, and any
of the RPL arguments may be preceded by the keyword \kwd{region}; but
the keywords are optional, as the compiler can infer which are the
types and which are the RPLs from the class definition.

For example:
%
\begin{dpjlisting}
class ExplicitArgs<type T, region R> {
  // Instantiate an ExplicitArgs type with T=Object, R=Root
  ExplicitArgs<Object,Root> field;
}
\end{dpjlisting}
%
This code defines a class \kwd{ExplicitArgs} with one type parameter
\kwd{T} and one RPL parameter \kwd{R}.  The field \kwd{field} has
class type \kwd{ExplicitArgs} with $\kwd{T}=\kwd{Object}$ and
$\kwd{R}=\kwd{Root}$.

The number of type arguments must exactly match the number of type
parameters.  However, fewer RPL arguments can be given than the number
of RPL parameters; any missing RPL arguments (matching the arguments
that are there from the left) are implicitly bound to \kwd{Root}.  For
example, in the code above, we could have written the type of field
\kwd{field} as \kwd{ExplicitArgs<Object>}.  Here is another example:
%
\begin{dpjlisting}
class ImplicitRPLArg<region R1, R2> {
  region r;
  ImplicitRPLArg<r> field;
}
\end{dpjlisting}
%
The class \kwd{ImplicitRPLArg} has two parameters \kwd{R1} and
\kwd{R2}, but the type \kwd{ImplicitRPLArg<r>} of \kwd{field} has only
one explicit RPL argument, \kwd{r}.  \kwd{r} is bound to the first
parameter \kwd{R1}, and the second parameter has no argument, so it is
bound to \kwd{Root}.  This type is the same as
\kwd{ImplicitRPLArg<r,Root>}.

\subsubsection{Class Type Comparisons%
\label{sec:types:class:compare}}
\cutname{types.class.compare.html}

The notion of \emph{type comparison} is important in Java.  For
example, if class \kwd{B} extends class \kwd{A}, and variable \kwd{x}
has type \kwd{A}, then you can assign an object of type \kwd{B} to
\kwd{x}; but it is not permissible to assign an object of some type
\kwd{C} that does not extend \kwd{A} (or extend another class that
extends \kwd{A}, etc.).

DPJ extends Java's type comparison rules to account for the RPL
information in the class types.  The rules ensure that the actual type
of an object always corresponds to the type that appears in a program
variable storing a reference to the object.  This property in turn
allows the compiler to reason soundly about effects by looking at the
types of variables.  If the ``wrong'' type of object could be assigned
to a variable, then this kind of reasoning would not work.

To state the comparison rules for DPJ, we introduce the concept of the
\emph{DPJ erasure} of a type.  This is the type obtained by ignoring
all RPL parameters and arguments.  For example, suppose class \kwd{C}
has a type parameter \kwd{T} and a region parameter \kwd{R}.  Then the
DPJ erasure of \kwd{C<Object,Root>} is \kwd{C<Object>}.

The following rules determine whether a variable of class type $T_1$
can be assigned to a variable of class type $T_2$ in DPJ.

\bfhead{Types that instantiate the same class} $T_1$ may be assigned
to $T_2$ if the two types instantiate the same class, the DPJ erasure
of $T_1$ may be assigned to the DPJ erasure of $T_2$ in ordinary Java,
and the RPL arguments of $T_1$ are included
(\S~\ref{sec:rpls:comparing:include}) in the corresponding arguments
of $T_2$.

Here are some examples of compatible types that instantiate the same
class:
%
\begin{itemize}
%
\item \kwd{C<Root>} may be assigned to \kwd{C<Root:*>}, because the
  DPJ erasures are identical, and \kwd{Root:*} includes \kwd{Root}.
%
\item \kwd{C<Object,Root>} may be assigned to \kwd{C<Object,Root:*>}
  for the same reason.
%
\item \kwd{C<C<Object,Root>,Root>} may be assigned to \kwd{C<? extends
  C<Object,Root:*>,Root:*>}.  The reasoning here is a bit more
  complicated.  First, look at the DPJ erasures of the types, that is
  \kwd{C<C<Object,Root>{>}} and \kwd{C<? extends C<Object,Root:*>{>}}.
  According to the rules for Java generic wildcards, the first type
  may be assigned to the second if \kwd{C<Object,Root>} may be
  assigned to \kwd{C<Object,Root:*>}.  To figure that out, we have to
  apply the rule recursively.  First look at the DPJ erasures:
  assigning \kwd{C<Object>} to \kwd{C<Object>} is OK.  Now look at the
  RPLs: \kwd{Root:*} includes \kwd{Root}.  So that test checks out.
  Now look at the RPL arguments in the original types: \kwd{Root:*}
  includes \kwd{Root}.  So that test checks out as well.
%
\end{itemize}
%

Here are some examples of incompatible types that instantiate the same
class, assuming \kwd{r1} and \kwd{r2} are region names:
%
\begin{itemize}
%
\item \kwd{C<r1>} may not be assigned to \kwd{C<r2>}, because \kwd{r1}
is not included in \kwd{r2}.
%
\item \kwd{C<Object,r1>} may not be assigned to \kwd{C<Object,r2>} for
  the same reason.
%
\item \kwd{C<C<Object,r1>,Root>} may not be assigned to \kwd{C<?
  extends C<Object,r2>,Root} because \kwd{C<Object,r1>} may not be
  assigned to \kwd{C<Object,r2>}.
%
\end{itemize}

\bfhead{Types that instantiate different classes}  $T_1$ may be
assigned to $T_2$ if $T_2$ is a \emph{direct supertype} or
\emph{indirect supertype} of $T_1$.

\ithead{Direct supertypes} $T_2$ is a direct supertype of $T_1$ if a
type with $T_2$'s class or interface appears in an \kwd{extends} or
\kwd{implements} clause of $T_1$'s class or interface, and that type
is assignable to $T_2$ after substituting arguments for parameters
according to $T_1$.

For example:
%
\begin{itemize}
%
\item Assume class declarations \kwd{class B<region R> extends A<R>}
  and \kwd{class A<region R>}.  Then \kwd{A<Root>} is a direct
  supertype of \kwd{B<Root>}, because \kwd{A<R>} appears in the
  \kwd{extends} clause of \kwd{B}, and the type obtained by
  substituting \kwd{Root} for \kwd{R} from the type \kwd{B<Root>} is
  \kwd{A<Root>}.
%
\item With the same assumptions as in (1), \kwd{A<Root:*>} is a direct
  supertype of \kwd{B<Root>}, because \kwd{A<Root>} is assignable to
  \kwd{A<Root:*>}.
%
\item Assume class declaration \kwd{class B<type T, region R> extends
  A<B<Root>,R>}.  Then \kwd{A<B<Root>,Root:*>} is a direct supertype
  of \kwd{B<Object,Root>}.
%
\end{itemize}
%

\ithead{Indirect supertypes} $T_2$ is an indirect supertype of $T_1$
if there is a chain of direct supertypes connecting $T_1$ to $T_2$.
For example, if we have class declarations \kwd{class C<region R>
  extends B<R>}, \kwd{class B<region R> extends A<R>} and \kwd{class
  A<region R>}, then \kwd{A<Root>} is an indirect supertype of
\kwd{C<Root>}, because \kwd{A<Root>} is a direct supertype of
\kwd{B<Root>}, and \kwd{B<Root>} is a direct supertype of
\kwd{C<Root>}.

\subsubsection{Class Type Casts%
\label{sec:types:class:cast}}
\cutname{types.class.cast.html}

One DPJ type may be cast to another if the cast would be allowed for
the DPJ erasures of the types (\S~\ref{sec:types:class:compare}) in
ordinary Java.  For example, assuming a class declared as \kwd{class
  C<region R>} the following cast is legal:
%
\begin{dpjlisting}
C<r1> x = (C<r1>) new C<r2>();
\end{dpjlisting}
%
This code creates an object of type \kwd{C<r2>} on the right-hand
side, then casts it to type \kwd{C<r1>} before assigning it to a
variable \kwd{x} of type \kwd{C<r1>}.  The DPJ erasure of both the
\kwd{new} expression and the target type is \kwd{C}, so the cast is
allowed.  Without the cast, \kwd{C<r2>} is not assignable to
\kwd{C<r1>}, because \kwd{r2} does not include \kwd{r1}.

However, the following cast is not legal, assuming a class declared as
\kwd{class C<type T, region R>}:
%
\begin{dpjlisting}
// Compile error!
C<C<Object,r1>,r1 x = 
  (C<C<Object,r1>,r1>) new C<C<Object,r2>,r2>();
\end{dpjlisting}
%
That is because the DPJ erasure of the type of the \kwd{new}
expression is \kwd{C<Object,r2>}; the DPJ erasure of the target type
is \kwd{C<Object,r1>}; and these two types are not compatible.

As in ordinary Java, casts allow assignments that the compiler cannot
check for correctness.  That is, in general for a statement of the
form
%
\begin{dpjlisting}
T x = (T) y
\end{dpjlisting}
%
there is no way to ensure at compile time that the type of the object
\kwd{y} is really consistent with the type \kwd{T} of variable
\kwd{x}.  Further, for efficiency reasons, DPJ has no checks for
catching bad assignments at runtime.  Therefore, a DPJ program with
casts in it is a potentially nondeterministic program!  Casts should
be used very carefully and only as a last resort, when there is no
other way to express the program.

\subsubsection{Owner RPLs%
\label{sec:types:class:owner-rpl}}
\cutname{types.class.owner-rpl.html}

Owner RPLs define the nesting relationship between \kwd{final} local
variables used as RPLs (\S~\ref{sec:rpls:basic:var}) and other RPLs.
A variable RPL is nested under the owner RPL of its class type
(\S~\ref{sec:rpls:comparing:nest}).

If a class has RPL parameters, then the owner RPL of a type
instantiating the class is the argument to the first parameter.  For
example, assuming a class declaration \kwd{class C<region R1,R2>}, and
region names \kwd{r1} and \kwd{r2}, the owner RPL of the type
\kwd{C<r1,r2>} is \kwd{r1}.

If a class does not have RPL parameters, then the owner RPL of a type
instantiating the class is \kwd{Root}.  For example, assuming a class
declaration \kwd{class C<type T>}, the owner RPL of the type
\kwd{C<Object>} is \kwd{Root}.
%

\subsection{Array Types%
\label{sec:types:array}}
\cutname{types.array.html}

This section discusses DPJ's support for Java-style array types, i.e.,
types of the form \kwd{T[]}.  For the most part, array classes
(\S~\ref{sec:classes:array}) should be more convenient to use than
Java-style array types.  However, for backwards compatibility,
Java-style array types are still supported.

\S~\ref{sec:types:array:writing} explains how to use array types in
DPJ.  \S~\ref{sec:types:array:compare} explains how the DPJ compiler
compares two array types for compatibility, for example in checking an
assignment statement.  \S~\ref{sec:types:array:cast} discusses casting
one array type to another in DPJ.

\subsubsection{Using Array Types%
\label{sec:types:array:writing}}
\cutname{types.array.writing.html}

A DPJ array type is the same as an ordinary Java array type, i.e., it
has the form \kwd{T[]} for some type \kwd{T} (which may itself be an
array type).  The only difference is that \kwd{T} may be a class with
region parameter arguments, or it may be an array type constructed
from such a class.  

Creation of array objects via \kwd{new} is exactly the same as in
Java.  For example, \kwd{new String[10]} creates a new array of 10
\kwd{String} objects.

Access to an array cell via an index operation always touches the
region \kwd{Root}.  If an array in a different region is desired, an
array class must be used (\S~\ref{sec:classes:array}).

\subsubsection{Array Type Comparisons%
\label{sec:types:array:compare}}
\cutname{types.array.compare}

The following rules determine when $T_1$ may be assigned to $T_2$,
where $T_1$ and $T_2$ are both array types.

\bfhead{Comparing arrays of primitive types} $T_1$ may be assigned to
$T_2$ if the element types of $T_1$ and $T_2$ are both the same
primitive type.

\bfhead{Comparing arrays of class and array types} $T_1$ may be
assigned to $T_2$ if the element types of $T_1$ and $T_2$ are
\emph{array-compatible} class or array types.  Two class types are
array-compatible if the element type of $T_1$ can be assigned to the
element type of $T_2$ using the rules in
\S\S~\ref{sec:types:class:compare},~\ref{sec:types:array:compare}, but
requiring equivalence (\S~\ref{sec:rpls:comparing:equiv}) instead of
inclusion of RPLs.  For example:
%
\begin{enumerate}
\item \kwd{C<R>[]} may be a assigned to \kwd{C<R>[]}, because
  \kwd{C<R>} is array compatible with itself.
\item \kwd{C<R>[]} may not be assigned to \kwd{C<R:*>[]},
  because \kwd{C<R>} and \kwd{C<R:*>} are not array-compatible.
\end{enumerate}
%
The extra requirement of array-compatibility is introduced to avoid
problems like this:
%
\begin{dpjlisting}
class C<region R> { ... }
region r1, r2;
// Create an array of 10 C<r1>
C<r1>[] arr = new C<r1>[10];
// Not really allowed, but would be without array compatibility
C<*>[] = badArr;
// Inconsistent types!  Assigning C<r2> to a cell of arr
badArr[0] = new C<r2>();
\end{dpjlisting}

\subsubsection{Array Type Casts%
\label{sec:types:array:cast}}
\cutname{types.array.cast.html}

Casts of array types have the same rules as casts of class types
(\S~\ref{sec:types:class:cast}).  The cast is allowed if it would be
allowed for the DPJ erasures of the types in ordinary Java.  The DPJ
erasure of an array type is computed by taking the erasure of the
class type appearing to the left of all brackets.  For example, the
DPJ erasure of \kwd{C<R>[]} is \kwd{C[]}.

\subsection{Typing Expressions%
\label{sec:types:exp}}
\cutname{typing.exp.html}

In Java, every expression has a type.  For example, if a method
returns \kwd{int}, then an expression invoking that method has type
\kwd{int}.  (Some statements have types too, but those statements
always enclose expressions.)  The types allow the compiler to enforce
consistency of assignments; for example, to make sure that an
\kwd{int} is never assigned to a variable of type \kwd{String}.

In DPJ, every expression has a type and an effect.
\S~\ref{sec:effects:stmt-exp} discusses the effects of statements and
expressions.  Here we discuss DPJ's extensions to Java's rules for
determinining the type of an expression.

\subsubsection{Field Access%
\label{sec:types:exp:field}}
\cutname{types.exp.field.html}

A field access expression in Java has the general form
%
\begin{description}
\item \emph{receiver-exp}\kwd{.}\emph{field-name}
\end{description}
%
where \emph{receiver-exp} is a class name $C$ or expression of class
type $C$, and \emph{field-name} is the name of a field of class $C$
(or a superclass of $C$).  If a bare field name appears, then the
implied receiver is either \kwd{C.}\emph{field-name} where \kwd{C} is
the enclosing class (for a static field), or
\kwd{this.}\emph{field-name} (for an instance field).  Here we give
the rules for the general form.

For instance fields (i.e., \emph{receiver-exp} is an expression of
class type), the compiler carries out the following steps to determine
the type of an expression \emph{receiver-exp}\kwd{.}\emph{field-name}:
%
\begin{enumerate}
%
\item Determine the type \emph{receiver-type} of \emph{receiver-exp}, using the rules in
  this section together with the ordinary rules for Java types.
%
\item Look up the type \emph{field-type} of \kwd{field} based on the
  class $C$ named in \emph{receiver-type}.
%
\item Compute the \emph{capture} (\S~\ref{sec:types:exp:capture}) of
  \emph{receiver-type} to generate the type
  \emph{captured-receiver-type}.
%
\item Make the following subsitutions in \emph{field-type} to generate
  the answer:
%
\begin{enumerate}
%
\item Substitute the type and RPL arguments of
  \emph{captured-receiver-type} for the corresponding type and RPL
  parameters of class $C$.
%
\item If \emph{receiver-exp} is a \kwd{final} local variable or
  \kwd{this}, then substitute the variable for \kwd{this}.  Otherwise,
  substitute a capture parameter (\S~\ref{sec:types:exp:capture})
  included in \emph{owner-rpl}\kwd{:*}, where \emph{owner-rpl} is the
  owner RPL of the variable (\S~\ref{sec:types:class:owner-rpl}).
%
\end{enumerate}
%
\end{enumerate}
%
For static fields (i.e., \emph{receiver-exp} is a class name), only
steps 1 and 2 are required.

Here is some example code for which the capture in step 3 is a no-op,
and the capture in step 4(b) isn't needed.
\S~\ref{sec:types:exp:capture} gives examples involving capture.
%
\begin{numbereddpjlisting}
class FieldTypingExample<region R> {
  region r;
  FieldTypingExample<R> field1 in R;
  void FieldTypingExample<r> method1() {
    return (new FieldTypingExample<r>).field1;
  }        
  FieldTypingExample<this> field2 in this;
  void FieldTypingExample<arg> 
  method2(final FieldTypingExample<R> arg) {
    return arg.field;
  }
}
\end{numbereddpjlisting}
%

Let's see how to compute the type of the expression in the
\kwd{return} statement in line 5.  
%
\begin{enumerate}
%
\item The receiver expression is a \kwd{new} expression of type
  \kwd{FieldTypingExample<r>}.
%
\item The field name is \kwd{field1}, the class $C$ is
  \kwd{FieldTypingExample}, and the declared type of the field is
  \kwd{FieldTypingExample<R>} (line 3).
%
\item Nothing to do (see \S~\ref{sec:types:exp:capture}).  
%
\item Substituting \kwd{r} given in the receiver type
  expression for \kwd{R} in the field type, we have that the answer is
  \kwd{FieldTypingExample<r>}.
%
\end{enumerate} 

Now let's see how to compute the type of the expression in the
\kwd{return} statement in line 10.  
%
\begin{enumerate}
%
\item The receiver expression is \kwd{arg}, which has type
  \kwd{FieldTypingExample<R>}.  
%
\item Now the field is \kwd{field2}, and its type is
  \kwd{FieldTypingExample<this>} (line 7).
%
\item Again, nothing to do.
%
\item \kwd{arg} is a \kwd{final} local variable (line 9), so we can
  substitute it for \kwd{this} in the type of \kwd{field2}.  Therefore
  the answer is \kwd{FieldTypingExample<arg>}.
%
\end{enumerate}

\subsubsection{Array Access%
\label{sec:types:exp:array}}
\cutname{sec:types:exp:array}

An array access expression has the form
\emph{array-exp}\kwd{[}\emph{index-exp}\kwd{]}, where \emph{array-exp}
is an expression of array class type (\S~\ref{sec:types:class}) or
array type (\S~\ref{sec:types:array}), and \emph{index-exp} is an
expression of integer type.

\bfhead{Array class type} If \emph{array-exp} has array class type,
then do the following:
%
\begin{enumerate}
%
\item Perform the same conversion as for field access
  (\S~\ref{sec:types:exp:field}), treating the cell type of the array
  class as the field type.
%
\item Substitute \emph{receiver-exp} for \kwd{index} in the result.
%
\end{enumerate}
%
For example, given the following code
%
\begin{dpjlisting}
class Data<region R> {
  static arrayclass Array<region R> {
    Data<R:[i]> in R:[i];
  }
}
region r;
Data.Array<r> a = new Data.Array<r>(10);
\end{dpjlisting}
%
The access expression \kwd{a[0]} has type \kwd{Data<r:[0]>}.

\bfhead{Array type} If \emph{array-exp} has array type, then delete
one set of brackets to get the type, as in ordinary Java.

\subsubsection{Method Invocation%
\label{sec:types:exp:invoke}}
\cutname{types.exp.invoke.html}

\bfhead{Explicit type and RPL arguments}  To compute the type of a
method invocation expression
%
\begin{description}
\item \emph{receiver-exp}\kwd{.<}\emph{type-args}\kwd{,}\emph{rpl-args}\kwd{>}\emph{method-name}\kwd{(}\emph{args}\kwd{)}
\end{description}
%
The compiler carries out the following steps:
%
\begin{enumerate}
%
\item Determine the type \emph{receiver-type} of \emph{receiver-exp}
  and the types \emph{arg-types} of \emph{args}, using the rules in
  this section together with the ordinary rules for Java types.
%
\item Look up the method in the ordinary Java way, using the method
  name, \emph{receiver-type}, and \emph{arg-types}.  Use the method to
  find formal value parameter types and return type.
%
\item Compute the \emph{capture} (\S~\ref{sec:types:exp:capture}) of
  \emph{receiver-type} to generate the type
  \emph{captured-receiver-type}.
%
\item Make the following substitutions in each of the formal argument
  types then check that the actual argument types \emph{arg-types} are
  assignable (\S\S~\ref{sec:types:class:compare}
  and~\ref{sec:types:array:compare}) to the formal argument types:
%
\begin{enumerate}
%
\item Substitute the type and RPL arguments of
  \emph{captured-receiver-type} for the type and RPL arguments of its
  class.
\item Capture \emph{type-args} and \emph{rpl-args} and substitute the
  captured types and RPLs for the type and RPL parameters of the
  method.
\item For every argument in \emph{args} of a type assignable to
  \kwd{int}, substitute the actual argument expression for the
  corresponding formal parameter of the method.
\item If \emph{receiver-exp} is a \kwd{final} local variable or
  \kwd{this}, then substitute the variable for \kwd{this}.  Otherwise
  substitute a capture parameter (\S~\ref{sec:types:exp:capture})
  included in \emph{owner-rpl}\kwd{:*}, where \emph{owner-rpl} is the
  owner RPL of the variable (\S~\ref{sec:types:class:owner-rpl}).
%
\end{enumerate}
%
\item Make the same substitutions as in step 4 in the return type to
  compute the answer.
%
\end{enumerate}

Here is some example code for which the capture in steps 3 and 4(b)
are no-ops, and the capture in step 4(d) isn't needed.
\S~\ref{sec:types:exp:capture} gives examples involving capture.
%
\begin{numbereddpjlisting}
abstract class MethodInvocationExample<region R1, R2> {
  abstract <region R3>
    MethodInvocationExample<R3, [i]>callee(int i);
  MethodInvocationExample<Root, [0]>caller() {
    return this.<Root>callee(0);
  }
}
\end{numbereddpjlisting}
%
Lines 1--2 define a method \kwd{callee} with one RPL parameter
\kwd{R3} and one formal parameter \kwd{int i}.  It returns type
%
\begin{description}
\item \kwd{MethodInvocationExample<R3,[i]>}.  
\end{description}
%
Notice that the return type is written in terms of the method
parameter (both the RPL and value parameter); these parameters have to
be substituted away to generate a meaningful type at the call site.

At the call site (line 5), the compiler does the following.  
%
\begin{enumerate}
%
\item The receiver type is \kwd{MethodRegionInvocationExample<R1,R2>},
  and the argument type is \kwd{int}.
%
\item The invoked method is \kwd{callee}, defined in lines 1--2.  
%
\item Nothing to do (see \S~\ref{sec:types:exp:capture}).
%
\item Binding 0 to \kwd{int} is OK.  
%
\item The return type is \kwd{<MethodInvocationExample<R3,[i]>}.  The
  method arguments are $\kwd{R3}=\kwd{Root}$ and $\kwd{i}=0$.
  Substituting arguments for parameters gives a return type of
  \kwd{MethodInvocationExample<Root,[0]>}.
%
\end{enumerate}

\bfhead{Inferred type and RPL arguments} If there are no explicit
type or RPL arguments, the compiler infers them as discussed in
\S~\ref{sec:classes:params:method} and proceeds as stated above, using
the inferred arguments in step 4(b).  As in ordinary Java, if there is
no \emph{receiver-exp}, then the implied receiver expression is
\kwd{this}.

\subsubsection{Captured Types%
\label{sec:types:exp:capture}}
\cutname{types.exp.capture.html}

The \emph{capture} of a generic type is important in generic Java; it
prevents bad assignments through wildcard types.  DPJ uses the same
concept in connection with partially-specified RPLs, which are a kind
of wildcard, as they can stand in for several different regions.  To
motivate the problem, consider this example:
%
\begin{numbereddpjlisting}
class CaptureExample<region R> {
  CaptureExample<R> field;
  void method() {
    region r1, r2;
    // Create a new CaptureExample<R:r1>
    CaptureExample<R:r1> ce = new CaptureExample<R:r1>();
    // Assign it to CaptureExample<R:*>; this is allowed
    CaptureExample<R:*> ceStar = ce;
    // This should not be allowed!
    ceStar.field = new CaptureExample<R:r2>();
  }
}
\end{numbereddpjlisting}
%
The assignment in line 10 is a problem: the \emph{actual} type of the
object stored in \kwd{ceStar} is \kwd{CaptureExample<R:r1>}.  That's
because the assignment in line 8 didn't change the type of the object;
it just assigned the same object to a different reference with a
partially specified type.  So the actual type of the \kwd{field} field
of that object is \kwd{CaptureExample<R:r1>}.  That means it can't
hold a \kwd{CaptureExample<R:r2>}; that would violate the consistency
of typing at runtime.  However, the type of the receiver expression
\kwd{ceStar} is \kwd{CaptureExample<R:*>}.  If we computed the type of
\kwd{ceStar.field} by just substituting the actual for formal
arguments in the type of the receiver expression, we would get
\kwd{CaptureExample<R:*>} for the type of \kwd{ceStar.field}, and the
bad assignment would be allowed.

So we don't do that.  Instead, we introduce a new parameter, called a
\emph{capture parameter}, in the type of the receiver expression.  It
stands in for the unknown actual region of an object stored in a
variable with partially specified type.  The introduction of the new
parameter is called \emph{capturing} a type.  It occurs in steps 3 and
4(b) of typing field access (\S~\ref{sec:types:exp:field}) and steps 3
and 4(d) of typing method invocation (\S~\ref{sec:types:exp:invoke}).
As discussed in \S~\ref{sec:classes:params:method}, capturing RPLs can
also occur when inferring arguments to method region parameters.

In this example, the captured type of the receiver expression is
\kwd{CaptureExample<P>}, where \kwd{P} is the capture parameter.
\kwd{P} is constrained to be included in \kwd{R:*}, because that is
all we know about the actual region from the type
\kwd{CaptureExample<R:*>}.  This is the only way parameters can be
inclusion-constrained in DPJ (\S~\ref{sec:rpls:comparing:include}).
In the example above, now the bad assignment is disallowed, because
the type \kwd{CaptureExample<R:r2>} is not assignable to the type
\kwd{CaptureExample<P>}, where \kwd{P} is the capture parameter.  Nor
is the type \kwd{CaptureExample<R:*>} assignable to
\kwd{CaptureExample<P>}.  However, \kwd{CaptureExample<P>} is
assignable to \kwd{CaptureExample<R:*>}, because of the inclusion
constraint.

More generally, this is how the compiler computes the capture of a type
in DPJ:
%
\begin{enumerate}
\item Take the normal Java capture of the type, substituting for any
  wildcard generic type arguments but keeping the same RPL
  arguments, if any.
\item For each RPL that is partially specified (i.e., contains \kwd{*}
  or \kwd{[?]}), replace that RPL with a capture parameter constrained
  to be included in the RPL.
\end{enumerate}
%
Notice that if there are no generic wildcards and no partially
specified RPL arguments, then the capture operation does nothing to
the type.

Capturing types is mostly an internal compiler mechanism.  Programmers
never have to do it, and they shouldn't even have to worry about it,
except to deal with capture errors when they occur.  For example, the
above code, if compiled, would generate a type error like ``expected
type \kwd{CaptureExample<capture of R:*>}, found type
\kwd{CaptureExample<R:r2>}.''  That error will probably be mysterious
unless you understand how type capture works.

Here is another common way that capture errors can occur, involving
method invocations:
%
\begin{numbereddpjlisting}
abstract class CaptureMethodExample<region R> {
  abstract void callee(CaptureMethodExample<R> x);
  void caller(CaptureMethodExample<*> y) {
    // Compile error!
    y.callee(y);
  }
}
\end{numbereddpjlisting}
%
Line 4 causes an error, because it is attempting to assign \kwd{y},
which has type 
%
\begin{description}
\item \kwd{CaptureMethodExample<*>} 
\end{description}
%
to type 
%
\begin{description}
\item \kwd{CaptureMethodExample<capture of *>}
\end{description}
%
which is the type of the formal parameter \kwd{x} of \kwd{callee},
after capturing the type of the receiver expression.

Usually you can work around these errors by adding a parameter.  For
example, the code above could be rewritten as follows:
%
\begin{numbereddpjlisting}
abstract class CaptureMethodExample<region R> {
  abstract void callee(CaptureMethodExample<R> x);
  <region R>void caller(CaptureMethodExample<R> y) {
    // OK
    y.callee(y);
  }
}
\end{numbereddpjlisting}
%
By adding a method region parameter \kwd{R}, we ``capture the type
ourselves.''  Now the code explicitly says that the region in the type
of \kwd{y} is the same as the region of the type of \kwd{x}: whatever
it is, it is bound to the same parameter \kwd{R} in both cases.



