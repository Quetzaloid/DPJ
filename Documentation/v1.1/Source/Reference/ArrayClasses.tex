\section{Array Classes%
\label{sec:array-classes}}
\cutname{array-classes.html}

DPJ v1.0 added region parameters directly to Java's syntax for array
types.  This is fine for simple cases, but it can quickly get
unwieldy.  Therefore, DPJ v1.1 adds a new feature called an
\emph{array class}.  Syntactically an array class is similar to an
ordinary Java class.  However, instead of specifying an ordinary Java
class object with fields and methods, it specifies a Java array object
by assigning a type and region to each cell of the array.

\subsection{Defining Array Classes
\label{sec:array-classes:defining}}
\cutname{classes.array.defining}

An array class definition is like an ordinary class definition with a
single field.  The differences are: (1) the definition begins with the
keyword \kwd{arrayclass} instead of \kwd{class}; (2) the ``field'' has
no name (since it actually defines the type of an array cell) and no
access specifier; and (3) in addition to defining the cell the class
definition may declare regions, but it may have no other members.

For example, the following code defines an array whose cells have type
\kwd{int} and are located in region \kwd{Root}:
%
\begin{dpjlisting}
arrayclass ArrayInt { int; }
\end{dpjlisting}
%
And the following code declares a region \kwd{r} and defines an array
whose cells have type \kwd{int} and are located in region
\kwd{ArrayInt2.r}:
%
\begin{dpjlisting}
arrayclass ArrayInt2 {
  region r;
  int in r;
}
\end{dpjlisting}

Type and region parameters work just as for ordinary Java classes
(\S~\ref{sec:classes:params}).  For example, the following code
declares an array class with one type parameter \kwd{T}, one region
parameter \kwd{R}, and cells of type \kwd{T} in region \kwd{R}:
%
\begin{dpjlisting}
arrayclass Array<type T, region R> {
  T in R;
}
\end{dpjlisting}

The following code is illegal, because it attempts to define an
instance method for an array class, which is not allowed:
%
\begin{dpjlisting}
arrayclass Illegal {
  int;
  int m() { return 0; }
}
\end{dpjlisting}
%
The following code is also illegal, because it doesn't provide a type
for the array cells:
%
\begin{dpjlisting}
arrayclass Illegal2 {
  region r;
}
\end{dpjlisting}

\subsection{Creating and Using Array Class Objects
\label{sec:array-classes:creating}}
\cutname{classes.array.creating}

Every array class has an implicit constructor that takes a single
integer length argument.  Calling the constructor creates a new array
with the specified type, region, and length.  For example, using the
array classes defined in the previous section, we could write the
following code:
%
\begin{dpjlisting}
// Create an array of 10 int in Root
ArrayInt a1 = new ArrayInt(10);
// Create an array of 10 int in ArrayInt2.r
ArrayInt2 a2 = new ArrayInt2(10);
// Create an array of 10 Integer in r
region r;
Array<Integer,r> a3 = new Array<Integer,r>(10);
\end{dpjlisting}
%
In each case, the effect of the constructor call is to create an
ordinary Java array.  For example, the compiler will translate
\kwd{new ArrayInt(10)} to \kwd{new int[10]} before generating code.

Once created, an array class object works just like an ordinary Java
array object: its fields can be read and written with the index
operator \kwd{[$\ldots$]}, it has an implicit \kwd{length} field, and
it works with the ``enhanced for'' construct of Java 5.  For example,
the following code creates and populates array of 10 \kwd{int} values,
and then prints them out:
%
\begin{dpjlisting}
ArrayInt a = new ArrayInt(10):
for (int i = 0; i < a.length; ++i)
  a[i] = i;
for (int i : a)
  System.out.println(i);
\end{dpjlisting}
%

\subsection{Index-Parameterized Array Classes
\label{sec:array-classes:ipa}}
\cutname{classes.array.ipa}

For parallel computations on arrays, it is often necessary to place
different array cells in different regions, so they can be updated in
parallel.  In DPJ v1.1, you do this by writing an array region
\kwd{[index]} in the scope of the array class definition, where
\kwd{index} is an implicit field that is in scope in every array class
definition (DPJ v1.0 used a different syntax).  The region
\kwd{[index]} stands in for the index region associated with each
cell. For example:
%
\begin{dpjlisting}
// Array of int such that cell i is in region [i]
arrayclass IPArrayInt {
  int in [index];
}
\end{dpjlisting}
%
Now we can do disjoint initialization of an array using a parallel
\kwd{foreach} loop (see \S~\ref{sec:parallel:foreach}):
%
\begin{dpjlisting}
IPArrayInt arr = new IPArrayInt(N);
foreach (int i in 0, N) {
  // Write to each distinct arr[i] in parallel
  arr[i] = i;
}
\end{dpjlisting}
%
The compiler knows this is safe, because the writes to \kwd{arr[i]}
touch a different region for each iteration of the parallel loop.

Index-parameterized arrays also support updates through references to
different objects, by using \kwd{[index]} in the RPL argument of the
element type of the array.  Every object stored in a distinct array
cell gets its own region, parameterized by the index of the cell.  For
example, suppose we have the following simple class definitions:
%
\begin{dpjlisting}
class Data<region R> {
  int field in R;
  static arrayclass Array<region R> {
    Data<R:[index]> in R:[index];
  }
}
\end{dpjlisting}
%
Then we can create the following index-parameterized array:
%
\begin{dpjlisting}
Data.Array arr = new Data.Array(N);
\end{dpjlisting}
%
The type of \kwd{arr} is an array such that (1) cell \kwd{i} of the
array is in region \kwd{[i]}; and (2) cell \kwd{i} of the array has
type \kwd{Data<[i]>}.  Now we can do disjoint initialization of the
array, as before:
%
\begin{dpjlisting}
foreach (int i in 0, N)
  arr[i] = new Data<[i]>();
\end{dpjlisting}
%
Notice that the type \kwd{Data<[i]>} on the left-hand side of the
assignment matches the type of \kwd{arr[i]}, as it must; see
\S~\ref{sec:types:exp:array}.

We can also update the objects disjointly through the references in
the array:
%
\begin{dpjlisting}
foreach (int i in 0, N) {
  // Effect is 'writes [i]'
  ++arr[i].field;
}
\end{dpjlisting}
%
This is a very useful pattern in shared-memory parallel programming.
See \tutorial\ for more realistic examples.


\subsection{Arrays of Arrays}

Array class definitions may be composed to define arrays of arrays.
For example, to define an array class equivalent to the ordinary Java
array type \kwd{int[][]}, one could do this:
%
\begin{dpjlisting}
arrayclass MatrixInt {
  ArrayInt;
}
arrayclass ArrayInt {
  int;
}
\end{dpjlisting}
%
Then we could write the following code:
%
\begin{dpjlisting}
// Create array of 10 ArrayInt
MatrixInt arr = new MatrixInt(10);
// Fill it in with 10 ArrayInts
for (int i = 0; i < 10; ++i)
  arr[i] = new ArrayInt(10);
// Initialize the ArrayInts
for (int i = 0; i < 10; ++i)
  for (int j = 0; j < 10; ++j)
    arr[i][j] = 10 * i + j;
\end{dpjlisting}

Of course if we decompose the arrays into regions, then we can update
them in parallel.  Here is another example that uses
index-parameterized arrays to do just that:
%
\begin{dpjlisting}
class MatrixExample {
  static arrayclass MatrixInt<region R> {
    ArrayInt<R:[index]> in R:[index];
  }
  static arrayclass ArrayInt<region R> {
    int in R:[index];
  }
  public static final int N = 10;

  public static void main(String[] args) {
    region r;
    MatrixInt<r> matrix = new MatrixInt<r>(N);
    foreach (int i in 0, N) {
      // Effect is 'writes [i]'
      matrix[i] = new ArrayInt<r:[i]>(N);
      foreach (int j in 0, N) {
        // Effect is 'writes [i]:[j]'
        matrix[i][j] = N*i+j;
      }
    }
  }
}
\end{dpjlisting}
%
Because of the region decomposition, the compiler can use the effects
to prove that the parallel tasks are noninterfering.  For the details
of how the compiler would check this example, see
\S~\ref{sec:effects:stmt-exp} (computing effects of statements) and
\S~\ref{sec:effects:nonint} (noninterfering effects).

\subsection{Generic Arrays}

Unfortunately, Java does not let you create arrays of class types that
have generic parameters without an explicit cast.  For example, the
following code will not work properly:
%
\begin{dpjlisting}
class Data<type T> {
  T field;
  arrayclass Array {
    Data<Integer>;
  }
  void m() {
    Array a = new Array(10);
  }
}
\end{dpjlisting}
%
This code passes the DPJ compiler, but it will not compile correctly
to Java bytecode, because the generated code causes a ``generic array
creation'' error.

To get around this limitation, you need to use a cast:
%
\begin{dpjlisting}
class Data<type T> {
  T field;
  arrayclass Array {
    Data<Integer>;
  }
  void m() {
    // Rewrite of the last line above so it compiles
    Array a = (Array) ((Object) new Object[10]);
  }
}
\end{dpjlisting}
%
This code is ugly, but it works.  Also, the ugliness is localized to
the point of array creation.  Once the array is created and assigned,
everything else works as it should.

